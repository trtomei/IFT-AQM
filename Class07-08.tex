%!TEX TS-program = xelatex
%!TEX encoding = UTF-8 Unicode

\documentclass[12pt]{article}
\usepackage{geometry}                % See geometry.pdf to learn the layout options. There are lots.
\geometry{a4paper,top=2cm}
\usepackage[parfill]{parskip}    % Activate to begin paragraphs with an empty line rather than an indent
\usepackage{graphicx}
\usepackage{amsmath}
\usepackage{amssymb}
\usepackage{mathtools}
\usepackage{physics}
\newcommand{\be}{\begin{equation}}
\newcommand{\ee}{\end{equation}}
\usepackage[thicklines]{cancel}

\usepackage{fontspec,xltxtra,xunicode}
\defaultfontfeatures{Mapping=tex-text}

\title{Advanced Quantum Mechanics\\Class 07--08}
%\author{The Author}
\date{August 30, 2022}                                           % Activate to display a given date or no date

\setcounter{section}{5}
\setcounter{subsection}{4}
\setcounter{equation}{49}

\begin{document}
\maketitle

\subsection{Reduced State Operator}

Consider \(\hat{\rho}\) acting in space \(H_{1} \otimes H_{2}\).
What is the state vector of particle 1?
Answer: let us examine a physical property \(C\)
which depends only on particle 1: \(\hat{C}=\hat{A} \otimes I_{2}\).

We want to find a \(\hat{\rho}_{1}\) such that
\be
\ev{A} = \Tr(\hat{\rho}_{1}\hat{A})
\ee
%%% \setcounter
\setcounter{equation}{51}
In \(H_{1} \otimes H_{2},\langle\hat{C}\rangle=\left\langle\hat{A} \otimes I_{2}\right\rangle\) is given by:
\be
\begin{aligned}
\left\langle\hat{A} \otimes I_{2}\right\rangle
&=\Tr\left[\left[\hat{A} \otimes I_{2}\right] \hat{\rho}\right]
=\sum_{n, m}\langle n m|\hat{A} \hat{\rho}| n m\rangle\\
&=\sum_{n, m} \sum_{n, m^{\prime}}\langle\underbrace{n m|\hat{A}| n^{\prime} m^{\prime}\rangle}%
_{\delta_{mm^\prime}A_{nn^\prime}}
\langle n^{\prime}m^{\prime}|\hat{\rho}| n m\rangle\\
&=\sum_{n, n^{\prime}} A_{nn^{\prime}}%
\underbrace{\sum_{m}\langle n^{\prime}m|\hat{\rho}| n m\rangle}%
_{\mathrlap{\text{Traced over particle 2 ($m$ indices): }\left(\rho^{(1)}\right)_{n^{\prime}n}}}
=\sum_{n,n^{\prime}} A_{nn^\prime} \rho_{n^{\prime}n}^{(1)}\\
%&=\sum_{n}\langle n|\hat{A} \hat{\rho}^{(1)}| n\rangle
%%% 2 OK
%=\Tr_{1}\left(\hat{A} \rho^{(1)}\right)
\end{aligned}
\label{eq:g52}
\ee
We can rearrange the $\rho^{(1)}$ expression to make it more clear:
\[
\begin{aligned}
\left(\rho^{(1)}\right)
&=\sum_{m}\left\langle n^{\prime} m|\hat{\rho}| n m\right\rangle
= \sum_{m}     \langle n^{\prime} \otimes m |\hat{\rho}| 
\underbrace{n \otimes m \rangle}%
_{\ket{n}\otimes\ket{m}}
\\
&=\sum_{m}\bra{n^\prime}\otimes\bra{m}\hat{\rho}\ket{m}\otimes\ket{n}\\
&=\bra{n^\prime}\otimes
\underbrace{\left(\sum_{m}\bra{m}\hat{\rho}\ket{m}\right)}%
_{\Tr_2 \hat{\rho}}
\otimes\ket{n}\\
\end{aligned}
\]
so we define the \emph{reduced} state operator of particle 1:
\be
\boxed{
\hat{\rho}^{(1)} = \Tr_2 \hat{\rho}
}\to\,\text{partial trace, trace on }H_2
\ee
Substitute back in Eq.~\ref{eq:g52}:
%%% Thiago: cambalacho setcounter
\[
\tag{\ref{eq:g52}$^\prime$}
\begin{aligned}
\left\langle\hat{A} \otimes I_{2}\right\rangle 
&= \sum_{n,n^\prime} A_{n,n\prime} \rho_{n^{\prime}n}^{(1)}
 = \sum_{n,n^\prime} \bra{n}\hat{A}\ket{n^\prime}\bra{n^\prime}\Tr_2\hat{\rho}\ket{n}\\
&= \sum_n \bra{n}\hat{A} \sum_{n^\prime}\op{n^\prime}{n^\prime}\Tr_2\hat{\rho}\ket{n}
 = \sum_n \bra{n}\hat{A}\Tr_2\hat{\rho}\ket{n}\\
&= \Tr_1\left(\hat{A}\rho^{(1)}\right)\checkmark
\end{aligned}
\]

Probability of finding eigenvalue $a_n$ of $\hat{A}$: in $H_1 \otimes H_2$: $\hat{A} \to \hat{A} \otimes \mathbf{I}_2$
\[
\begin{aligned}
\ev{\hat{A}} 
&= \Tr_1
\underbrace{\left(\hat{A}\rho^{(1)}\right)}%
_{\mathrlap{\text{evaluate this in the eigenstates of }\hat{A}}}
\equiv \sum_n a_n p(a_n)\\
&=\sum_n \bra{\varphi_n}\hat{A}\rho^{(1)}\ket{\varphi_n} 
=\sum_n a_n \underbrace{\bra{\varphi_n}\rho^{(1)}\ket{\varphi_n}}%
_{p(a_n)}
\end{aligned}
\]
%
\[
\begin{aligned}
p(a_n) 
&= \bra{\varphi_n}\rho^{(1)}\ket{\varphi_n}
= \Tr_1 \hat{P}_n \rho^{(1)}\\
&= \Tr_1 \hat{P}_n \Tr_2 \hat{\rho}
= \Tr_1 \left[\Tr_2\left(\hat{P}_n \otimes\mathbf{I}_2\right)\hat{\rho}\right]
\end{aligned}
\]
%%% 3 OK
and therefore
\be
p(a_n) = \Tr_1\Tr_2 \left[\left(\hat{P}_n \otimes \mathbf{I}_2\right)\hat{\rho}\right]
\ee
\emph{Example:} $\ket{\psi}$ generic pure state of $H_1 \otimes H_2$:
\be
\ket{\psi} = \sum_{i=1}^{N} \sum_{j=1}^{M} C_{ij} \ket{\varphi_i \otimes \chi_j}
\ee
\be
\hat{\rho} = \op{\psi}{\psi}
\ee
We write the reduced state operator $\hat{\rho}^(1)$:
\[
\rho^{(1)} = \Tr_2 \hat{\rho} =  \Tr_2 \op{\psi}{\psi} = \sum_{i,k=1}^{N} \sum_{j,l=1}^{M} C_{ij} C_{kl}^*
\times \Tr_2 \op{\varphi_i \chi_j}{\varphi_k \chi_l}
\]
but
\[
\Tr_2 \op{\varphi_i \chi_j}{\varphi_k \chi_l} = 
\op{\varphi_i}{\varphi_k} 
\underbrace{\Tr\op{\chi_j}{\chi_l}}%
_{\text{see below}}
\]
and using that
\[
\Tr\op{a}{b} = \sum_n\bra{n}\ket{a}\bra{b}\ket{n}
=\bra{b}\sum_n\ket{n}\bra{n}\ket{a}
=\bra{b}\ket{a}
\]
we have that
\[
\Tr\op{\chi_j}{\chi_l} = \bra{\chi_l}\ket{\chi_j} = \delta_{lj}\to\text{orthonormal} 
\]
so finally
\be
\rho^{(1)} = \sum_{i,k=1}^{N} \sum_{j=1}^{M} C_{ij} C_{kj}^* \op{\varphi_i}{\varphi_k} 
\ee
In the case that
\(
\ket{\psi} = \sum_{i=1}^{N} C_{i} \ket{\varphi_i \otimes \chi_i}:
\)
\be
\rho^{(1)} = \sum_{i} |C_{i}|^2 
\underbrace{\op{\varphi_i}{\varphi_i}}%
_{\mathrlap{\text{again } \bra{\chi_j}\ket{\chi_i} = \delta_{ij}}}
\ee

Also, if $\ket{\psi} = \ket{\varphi\otimes\chi}$, then $\rho^{(1)}$ itself also describes a
pure state:
\be
\begin{gathered}
\rho^{(1)} = \Tr_2 \op{\varphi\chi}{\varphi\chi} = \ket{\varphi}
\underbrace{\Tr\op{\chi}{\chi}}_{1}
\bra{\varphi}\\
%%% 4 OK
\ket{\psi} = \ket{\varphi\otimes\chi} \Rightarrow \rho^{(1)} = \op{\varphi}{\varphi}
\end{gathered}
\ee

\emph{Now}, if $\ket{\psi}$ \emph{is not a tensor product},
$\rho^{(1)}$ (or $\rho^{(2)}$, of course) does not describe
a pure state $\to$ \emph{not possible} to attribute a
well-defined state to particle 1 (or 2).
\[
\ket{\psi} = \frac{1}{\sqrt{2}}\left(\ket{+-}-\ket{-+}\right)
\]
\be
\begin{aligned}
\Tr_2\hat{\rho} 
&= \frac{1}{2}\Tr_2\left(
\op{+-}{+-} - 
\op{+-}{-+} -
\op{-+}{+-} +
\op{-+}{-+}
\right)\\
&=\frac{1}{2}\left(
\op{+}{+} -0 -0 + \op{-}{-} 
\right)
=\frac{1}{2}\left(
\op{+}{+} + \op{-}{-} 
\right)\\
&=\frac{1}{2}
\begin{pmatrix}
1 & 0\\0 & 1
\end{pmatrix}
\to\text{unpolarized state}
\end{aligned}
\label{eq:g60}
\ee

\emph{Conclusion:} Even when the two-particle state is
a pure state, the state of an
individual particle is in general
a mixture.

Eq.~\eqref{eq:g60}: an extreme case of a mixture, maximal
disorder $\equiv$ minimal information.
Disorder $\to$ measure of information? YES!
%%% 5 OK

\subsubsection{Entanglement entropy}

Also know as von Neumann entropy (compare with Gibbs / Boltzmann entropy):
\be
S_E = -\Tr(\hat{\rho}\ln\hat{\rho})
\ee
in case of spin 1/2:
\be
\left\{
\begin{aligned}
&S_E \text{ maximum: }              & \ln2\\
&S_E \text{ minimum (pure state): } & 0 
\end{aligned}
\right.
\ee

\begin{enumerate}
\item Pure state: $\rho^2 = \rho$
\be
2\ln\rho = \ln\rho \to \ln\rho = 0  \Rightarrow S_E = 0
\ee
\item %
\[
\begin{gathered}
\rho        = \frac{1}{2}\begin{pmatrix}1 & 0\\0 & 1\end{pmatrix} \to
\ln\rho     = \begin{pmatrix}\ln 1/2 & 0\\0 & \ln1/2\end{pmatrix}\\
\rho\ln\rho = \begin{pmatrix}1/2 & 0\\0 & 1/2\end{pmatrix} 
\begin{pmatrix}\ln 1/2 & 0\\0 & \ln1/2\end{pmatrix} = 
\begin{pmatrix}1/2 \ln 1/2 & 0\\0 & 1/2 \ln1/2\end{pmatrix}
\end{gathered}
\]
therefore
\be
S_E = -\Tr\begin{pmatrix}1/2 \ln 1/2 & 0\\0 & 1/2 \ln 1/2\end{pmatrix} = -\ln1/2 =\ln2
\ee
\end{enumerate}

\emph{Note that:} Hilbert space has dim.\,$N$, the
$\hat{\rho}$ corresponding to maximal disorder:
\be
\rho        = \frac{1}{N}\begin{pmatrix}1 & 0\\0 & 1\end{pmatrix} \Rightarrow S_E = \ln N
\ee

\subsection{Time dependence of $\hat{\rho}$}

%%% \setcounter
\setcounter{equation}{64}
First, consider a pure state:
\be
\hat{P}_\varphi(t) = \op{\varphi(t)}
\ee
%
\be
\begin{aligned}
%%% 6 OK
i\hbar\hat{P}_\varphi(t)
&=i\hbar\frac{d}{dt}\left(\op{\varphi(t)}\right)\\
&=i\hbar\underbrace{\ket{\frac{d\varphi(t)}{dt}}}%
_{\ket{\frac{1}{i\hbar}\hat{H}\varphi}}
\bra{\varphi(t)} + 
  i\hbar\ket{\varphi(t)}\underbrace{\bra{\frac{d\varphi(t)}{dt}}}%
_{\bra{\frac{1}{i\hbar}\hat{H}\varphi}}\\
&= i\hbar\left[
\frac{1}{i\hbar}
\left(
\hat{H}(t)\op{\varphi(t)} - \op{\varphi(t)}\hat{H}(t) 
\right)
\right]\\
&=[\hat{H}(t),\hat{P}_\varphi(t)]
\end{aligned}
\ee

We use this result in the derivation of $d\hat{\rho}/dt$:
\[
\begin{aligned}
i\hbar \frac{d\hat{\rho}}{dt}
&= \sum_\alpha p_\alpha \frac{d}{dt}\left(\op{\varphi_\alpha(t)}\right)
= \sum_\alpha p_\alpha [\hat{H}(t),\op{\varphi_\alpha}]\\
&= \left[\hat{H}(t),\sum_\alpha p_\alpha \op{\varphi_\alpha}\right]
= \left[\hat{H}(t),\hat{\rho}(t)\right]
\end{aligned}
\]
therefore
\be
\boxed{
i\hbar \frac{d\hat{\rho}}{dt} = \left[\hat{H}(t),\hat{\rho}(t)\right]
}-
\ee
so $\hat{\rho}(t)$ has 
Hamiltonian evolution (unitary evolution),
for which \(S_{E}\) is time-independent, as can
be shown very easily.

For a $\hat{\rho}$ corresponding to maximal disorder $\to$ that $\hat{\rho}$
is a dynamical invariant, \textit{i.e.} $[\hat{H}(t),\hat{\rho}(t)] = 0$.

%%% AULA 8: 1 OK
\subsubsection{Time evolution of $\rho$ spin 1/2 in magnetic field $\vec{B}$}

Let $\vec{B} = B\hat{z}$. Then
\be
\hat{H} = -\vec{\mu} \cdot \vec{B}=-\gamma B \hat{S_{z}}=-\gamma \frac{B \hbar}{2} \sigma_{z}
\ee
Recall that the state operator for a two-level system like a spin 1/2 is
\[
\rho = \frac{1}{2}(1+\vec{b}\cdot\vec{\sigma})
\]
so,

\be
\begin{aligned} 
i \hbar \frac{d \hat{\rho}}{d t} 
&=[\hat{H}, \hat{\rho}]=-\frac{\gamma B \hbar}{2}\left[\sigma_{z}, \frac{1}{2}(1+\vec{b} \cdot \vec{\sigma})\right] \\ 
&=-\frac{\gamma B \hbar}{4}\left[\sigma_{z}, b_{x} \sigma_{x}+b_{y} \sigma_{y}\right]\left\{\left[\sigma_{i}, \sigma_{j}\right]=2 i \varepsilon_{i j k} \sigma_{k}\right.\\ 
&=-\frac{\gamma B \hbar}{4}\left(2 i b_{x} \sigma_{y}-2 i b_{y} \sigma_{x}\right)\\
&=-\frac{\gamma B}{2} i \hbar\left(b_{x} \sigma_{y}-b_{y} \sigma_{x}\right)
\end{aligned}
\ee
therefore
\[
\begin{aligned}
\cancel{i \hbar} \frac{1}{2} 
&\frac{d \vec{b}}{d t} \cdot \vec{\sigma}=
-\frac{\gamma B}{2} \cancel{i \hbar} \underbrace{\left(b_{x} \sigma_{y}-b_{y} \sigma_{x}\right)}%
_{(\vec{b} \times \vec{\sigma})_{z}} \Rightarrow\\
&\frac{d \vec{b}}{d t} \cdot \vec{\sigma} = \gamma \vec{B}\cdot(\vec{b} \times \vec{\sigma})
= -\gamma \vec{B}\times\vec{b}\cdot\vec{\sigma}
\end{aligned}
\]
and finally
\be
\boxed{
\frac{d \vec{b}}{d t} = -\gamma \vec{B}\times\vec{b}
} 
\ee
that is the classical equation of motion of a magnetic moment.

\subsubsection{Creating coherences}

Consider spins in contact with a fluctuating medium
at temperature \(T\). What is \(\vec{b}\) in such a situation? Need
to know \(\hat{\rho}\): Boltzmann distribution
\be
\hat{\rho}=\frac{1}{Z} e^{-\beta \hat{H}}, \quad
\beta=\frac{1}{k_{B} T}, \quad 
Z=\Tr e^{-\beta \hat{H}}
\ee
%%% AULA 8: 2 OK

\be
\begin{gathered}
e^{-\beta \hat{H}}=e^{\beta \hbar \omega_{0} / 2 \sigma_{z}} \approx 1+\beta \frac{\hbar \omega_{0}}{2} \sigma_{z}\\
\text{where } \omega_0 = \gamma\beta \text{ and we use the weak field approx. (high $T$)}\\
\Tr e^{-\beta \hat{H}} \approx 2
\end{gathered}
\ee
hence
\be
\hat{\rho} \simeq 1 / 2\left(I+\beta \frac{\hbar \omega_{0}}{2} \sigma_{z}\right) \Rightarrow b_{z}=\hbar \omega_{0} / 2
\ee

Note that:
\[
\begin{aligned}
\hat{\rho} \sim e^{\lambda \beta \hbar \omega_{0} / 2 \sigma_z} (\op{+} + \op{-})\\
\underbrace{e^{ \beta \hbar \omega_{0} / 2}}_{p_+} \op{+} + 
\underbrace{e^{-\beta \hbar \omega_{0} / 2}}_{p_-} \op{-}
\end{aligned}
\]
therefore
\be
p_{+}=e^{\beta \hbar \omega_{0} / 2} \simeq 1+\beta \hbar \omega_{0} / 2 \quad, \quad 
p_{-} \simeq 1-\beta \hbar \omega_{0} / 2
\ee
so
\be
\begin{gathered}
p_+ - p_- \equiv \delta p = \beta \hbar \omega_{0} \Rightarrow b_z = 1/2 \delta p\\
\rho(t=0) \simeq 1/2 (I + 1/2 \delta p \sigma_z)
\end{gathered}
\ee

Apply a transverse periodic field, \(\vec{B}_{1}(t)=B_{1}\left(\hat{x} \cos \omega t-\hat{y} \sin \omega t\right)\)
\(\rightarrow\) at resonance, \(\omega=\omega_{0}\), the effect is equivalent to a
rotation of the spin by angle \(-\theta\) (about $\hat{x}$), with \(\theta=\omega_{1} t\), where \(\omega_{1}=\gamma B_{1}\) (Rabi frequency).
\be
\begin{aligned} 
\hat{\rho}(t) & \rightarrow \rho_{\theta}=
\hat{U}\left[R_{x}(-\theta)\right] \hat{\rho}(0) \hat{U}^{\dagger}\left[R_{x}(-\theta)\right] \\ 
& \hat{U}\left[R_{x}(-\theta)\right]=e^{+i \theta \sigma_{x} / 2} 
\end{aligned}
\ee
therefore
%%% AULA 8: 3 OK
\be
\begin{aligned}
&e^{i \theta \sigma_{x} / 2} \sigma_{z} e^{-i \theta \sigma_{x} / 2}
=\left(\cos \theta / 2+i \sigma_{x} \sin \theta / 2\right) \sigma_{z}
 \left(\cos \theta / 2-i \sigma_{x} \sin \theta / 2\right)\\
&=\cos ^{2} \theta / 2 \sigma_{z}-
i \cos \theta / 2 \sin \theta / 2
\underbrace{\left(\sigma_{z} \sigma_{x} -\sigma_{x} \sigma_{z}\right)}%
_{2i\sigma_y}+
\underbrace{\sigma_{x} \sigma_{z} \sigma_{x}}%
_{-\sigma_z}
\sin ^{2} \theta / 2 \\
&=\left(\cos ^{2} \theta / 2-\sin ^{2} \theta / 2\right) \sigma_{z}+2 \cos \theta / 2 \sin \theta / 2 \sigma_{y}\\
&=\cos \theta \sigma_{z}+\sin \theta \sigma_{y}
\end{aligned}
\ee
and finally
\be
\hat{\rho}(t)=1 / 2\left[I+1 / 2 \delta p\left(\sigma_{z} \cos \omega_{1} t+\sigma_{y} \sin \omega_{1} t\right)\right]
\ee
For a time interval of $t = \pi/(2\omega_1)$:
\be
\hat{\rho}\left(t e^{\pi / 2 \omega_{1}}\right)=
1 / 2\begin{pmatrix}1 & -\frac{i}{2} \delta p \\ \frac{i}{2} \delta p & 1\end{pmatrix}
\ee
where we can see that we \emph{created coherences}: started from a
diagonal $\hat{\rho}$ and after a time $t = \pi/(2\omega_1)$
obtained a $\hat{\rho}$ with off-diagonal
terms.

Return to equilibrium : introduce a relaxation-time
factor in $b \rightarrow b e^{-t / T_{r}}$ (Exercise 6.5.6, where the relaxation time is called $T_2$).

\subsection{General form of postulates}


\emph{Postulate Ia}: state of a quantum system is represented
%%% AULA 8: 4 OK
by a state operator acting in a Hilbert space $H$;
\(\hat{\rho}\) is a positive operator with unit trace.


\emph{Postulate IIa}: probability $p_{\chi}$ of finding the quantum
systen in state $|\chi\rangle$ is 
\be
p_{\chi}=\Tr(\hat{\rho}\op{\chi})=\Tr\left(\hat{\rho} \hat{P}_{\chi}\right)
\ee


\emph{Postulate IIIa} =  Postulate III: to a physical property
$A$ there is an associated Hermitian operator \(\hat{A}\)
in $H$.

\emph{Postulate IVa}: the time evolution of the state operator
is given by
\be
i\hbar \frac{d \hat{\rho}(t)}{d t}=[\hat{H}(t), \hat{\rho}(t)]
\ee
but this holds only for \emph{closed} systems.

Supplementary WFC postulate becomes: when \(A\) is
measured, the result is one of the eigenvalues $a_n$ of \(\hat{A}\)
and the state operator right after the measurement is
\be
\hat{\rho} \rightarrow \hat{\rho}_{n}=
\frac{\hat{P}_{n}\left(\hat{\rho} \hat{P}_{n}\right)}
{\Tr\left(\hat{\rho} \hat{P}_{n}\right)}
\ee






\end{document}