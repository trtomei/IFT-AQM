%!TEX TS-program = xelatex
%!TEX encoding = UTF-8 Unicode

\documentclass[12pt]{article}
\usepackage{geometry}                % See geometry.pdf to learn the layout options. There are lots.
\geometry{a4paper,top=2cm}
\usepackage[parfill]{parskip}    % Activate to begin paragraphs with an empty line rather than an indent
\usepackage{graphicx}
\usepackage{amsmath}
\usepackage{amssymb}
\usepackage{mathtools}
\usepackage{physics}
\newcommand{\be}{\begin{equation}}
\newcommand{\ee}{\end{equation}}
\usepackage[thicklines]{cancel}
\usepackage[colorlinks=true,citecolor=blue,linkcolor=blue,urlcolor=blue]{hyperref}
\usepackage{booktabs}
\usepackage{csquotes}
\usepackage{qcircuit}
\usepackage{circledsteps}
\usepackage{nicefrac}
\usepackage{fontspec,xltxtra,xunicode}
\usepackage{xcolor}
\usepackage{simplewick}
\defaultfontfeatures{Mapping=tex-text}

\newcommand{\polv}{\ensuremath{\updownarrow}}
\newcommand{\polh}{\ensuremath{\leftrightarrow}}
\newcommand{\poldr}{\rotatebox[origin=c]{45}{\ensuremath{\leftrightarrow}}}
\newcommand{\poldl}{\rotatebox[origin=c]{-45}{\ensuremath{\leftrightarrow}}}
\newcommand{\bigzero}{\mbox{\normalfont\Large\bfseries 0}}
\newcommand{\vecrp}{\ensuremath{\vec{r}^{\,\prime}}}
\newcommand{\vecnr}{\ensuremath{\vec{\nabla}_{\!r}}}

\title{Advanced Quantum Mechanics\\Class 19 (a)}
%\author{The Author}
\date{April 27, 2023}                                           % Activate to display a given date or no date

\setcounter{section}{4}
\setcounter{subsection}{7}
\setcounter{equation}{107}

\begin{document}
\maketitle

%%% 01 OKAY

\subsection{Addition of two angular momenta \(\vec{J_{1}}\) and \(\vec{J_{2}}\)}

Let 
\(\vec{J_{1}} \in \varepsilon\left(j_{1}\right)\) and \(\vec{J_{2}} \in \varepsilon\left(j_{2}\right)\)
and
\be
\hat{\vec{J}}=\vec{J}_{1} \otimes \mathbf{1}_{2}+\mathbf{1}_{1} \otimes \hat{J}_{2} \in \varepsilon=\varepsilon\left(j_{1}\right) \otimes \varepsilon\left(j_{2}\right)
\ee

\(\hat{\vec{J}}_{1}\) and \(\hat{\vec{J}}_{2}\) commute in \(\varepsilon=\varepsilon\left(j_{1}\right) \otimes \varepsilon\left(j_{2}\right)\) :
\be
\left[\hat{J}_{1 i}, \hat{J}_{2 j}\right]=0, \quad \text{for all } i, j
\ee
In $\varepsilon$, $\vec{J}$ satisfies the angular momentum
commutation relations:
\be
\begin{aligned} 
{\left[\hat{J}_{i}, \hat{J}_j\right] } 
&=\left[\hat{J}_{1 i}+\hat{J}_{2 i}, \hat{J}_{1 j}+\hat{J}_{2 j}\right] \\ 
&=i \hbar \varepsilon_{i j k}\left(\hat{J}_{1 k}+\hat{J}_{2 k}\right)=i \hbar \varepsilon_{i j k} \hat{J}_{k} \end{aligned}
\ee
%%% 02 OKAY
From this algebra alone:
\be
\begin{aligned}
\hJtwo|j m\rangle
&=j(j+1) \hbar^{2}|j m\rangle \\ 
\hat{J}_{0}|j m\rangle
&=m \hbar|j m\rangle,\quad j \leqslant m \leqslant j
\end{aligned}
\ee
\emph{But}, we need to elaborate a little more to get
its connections with the states $\ket{j_1m_1}$ and $\ket{j_2m_2}$ -- can have at least two orthonormal basis in $\varepsilon:$

\begin{enumerate}
\item
\be
\ket{j_1 m_1} \otimes \ket{j_2 m_2} = \ket{j_1 j_2 m_1 m_2}
\ee
This basis corresponds to the complete set of
commuting operators
\be
%\left\{\vec{J}_{1}^{2}, \vec{J}_{2}^{2}, \hat{J}_{1z}, \hat{J}_{2z}\right\}
\left\{\hJtwo_1, \hJtwo_2, \hat{J}_{1z}, \hat{J}_{2z}\right\}
\ee
%
\item
\be
\ket{j_1 j_2 j m} \equiv \ket{jm}
\ee
which corresponds to
\be
%\left\{\vec{J}_{1}^{2}, \vec{J}_{2}^{2}, \vec{J}^{2}, \hat{J}_{z}\right\}
\left\{\hJtwo_1, \hJtwo_2, \hJtwo, \hat{J}_{z}\right\}
\ee
\end{enumerate}
The analysis of these bases leads to the following results:
\begin{enumerate}
\item the possible values of $j$ are
\be
\left|j_{1}-j_{2}\right|,\left|j_{1}-j_{2}\right|+1, \cdots, j_{1}+j_{2}-1, j_{1}+j_{2}
\ee
%
\item the dimension of $\varepsilon$ is 
\be
\operatorname{dim}\varepsilon = (2j_1+1)(2j_2+1)
\ee
\end{enumerate}

%%% 03 OKAY

\subsubsection{Change of basis}

One can switch from one basis to the other, \textit{e.g.}
from the $\ket{j_1 j_2 m_1 m_2}$ to the $\ket{jm}$:
\[
\begin{aligned}
|j m\rangle
&=\sum_{m_{1} m_{2}}\left|j_{1} j_{2} m_{1} m_{2}\right\rangle\left\langle j_{1} j_{2} m_{1} m_{2} |j m\right\rangle\\
&=\sum_{m_{1} m_{2}}\left\langle j_{1} j_{2} m_{1} m_{2} | j m\right\rangle\left|j_{1} j_{2} m_{1} m_{2}\right\rangle
\end{aligned}
\]
where the sum cannot be over all $m_1$,$m_2$.
In fact
\[
\begin{aligned}
\hat{J}_{0}|j m\rangle
&=m \hbar \sum_{m_{1} m_{2}}\left\langle j_{1} j_{2} m_{1} m_{2} | j m\right\rangle\left|j_{1} j_{2} m_{1} m_{2}\right\rangle\\
\left(\hat{J}_{10}+\hat{J}_{20}\right)|j m\rangle
&=\sum_{m_{1} m_{2}}\left(m_{1}+m_{2}\right) \hbar\left\langle j_{1} j_{2} m_{1} m_{2} | j m\right\rangle\left|j_{1} j_{2} m_{1} m_{2}\right\rangle
\end{aligned}
\]
whence
\be
\sum_{m_1 m_2}
[(m-(m1+m2)]\op{\ldots}\ket{j_1j_2m_1m_2} = 0
\Rightarrow m = m_1+m_2
\ee
%%% 04 OKAY
so finally
\be
\boxed{
\ket{jm} = \sum_{m_1+m_2=m}
C^{j_1j_2}_{m_1,m_2;jm} %%% Clebsh-Gordan (CG) coefficients
\ket{j_1j_2m_1m_2}
}
\ee
where we see the Clebsch-Gordan (CG) coefficients
\be
C^{j_1j_2}_{m_1,m_2;jm} = 
\bra{j_1j_2m_1m_2}\ket{jm} =
\bra{jm}\ket{j_1j_2m_1m_2}^*
\ee
%
\begin{align}
&\text{- nonzero for } |j_1-j_2| \leqslant j \leqslant j_1+j_2 \text{ and } m = m_1+m_2\\
&\text{- convention: } C^{j_1j_2}_{m_1,m_2;jm=j} \text{\emph{ are real } > 0}
\end{align}

\emph{Exercise:} with the above convention, show that the CG are \emph{real}.

\emph{Example:} sum of two spin-1/2
\be
j_1 = j_2 = 1/2,
\quad
j = 0,1
\begin{cases}
\begin{aligned}
C^{1/2\,1/2}_{1/2\,1/2;11} &\,=  C^{1/2\,1/2}_{-1/2\,-1/2;1-1} &\quad=& \quad 1\\
C^{1/2\,1/2}_{1/2\,-1/2;10} &\,=  C^{1/2\,1/2}_{1/2\,-1/2;00} &\quad=& \quad 1/\sqrt{2}\\ 
C^{1/2\,1/2}_{-1/2\,1/2;10} &\,=  -C^{1/2\,1/2}_{-1/2\,1/2;00} &\quad=& \quad 1/\sqrt{2}
\end{aligned}
\end{cases}
\ee
so
\be
T\,
\begin{cases}
\begin{aligned}
\ket{1,\phantom{-}1} &= \ket{1/2,1/2,1/2,1/2}\\
\ket{1,\phantom{-}0} &= \frac{1}{\sqrt 2}
\left[\ket{1/2,1/2,1/2,-1/2} + \ket{1/2,1/2,-1/2,1/2}\right]\\
\ket{1,-1} &= \ket{1/2,1/2,-1/2,-1/2}\\
\end{aligned}
\end{cases}
\ee
and
\be
S:\,
\ket{0,\phantom{-}0} = \frac{1}{\sqrt 2}
\left[\ket{1/2,1/2,1/2,-1/2} - \ket{1/2,1/2,-1/2,1/2}\right]
\ee
$T$: triplet, $S$: singlet.

%%% 05 (SKIP)

%%% 06 OKAY

\emph{Exercise:} consider the case $j_1 = 1$, $j_2 = 1/2$.
Possible $j$'s: $j$ = 1/2, 3/2.
$j$=1/2, $m$ = $\pm$1/2;  
$j$=3/2, $m$ = $\pm$1/2, $\pm$3/2
$\Rightarrow$ total of six states.
Use the CG table\footnote{\url{https://pdg.lbl.gov/2022/reviews/rpp2022-rev-clebsch-gordan-coefs.pdf}}
to construct these six states.

To conclude, we note that since the CG coefficients
are the elements of a unitary real matrix with
indices $(m_1m_2)$ and $(jm)$, they satisfy the following
orthogonality relations:
\be
\sum_{m_{1}=-j_{1}}^{j_{1}} \sum_{m_{2}=-j_{2}}^{j_{2}} C_{m_{1} m_{2} ; j m}^{j_{1} j_{2}} C_{m_{1} m_{2} ; j^\prime m^\prime}^{j_{1} j_{2}}=\delta_{j j^{\prime}} \delta_{m m^\prime}
\ee
\be
\sum_{j=\left|j_{1}-j_{2}\right|}^{j_{1}+j_{2}} \sum_{m=-j}^{+j} C_{m_{1} m_{2} ; j m}^{j_{1} j_{2}} C_{m_{1} m_{2} ; j m}^{j_{1} j_{2}}=\delta_{m_{1} m_{1}^{\prime}} \delta_{m_{2} m_{2}^\prime}
\ee
After this review, we are equipped to construct all irreducible representations $D^{(j)}$ from the $D^{(1/2)}$
representation.

%% 07 OKAY

\subsubsection{Kronecker product of representations}
\setcounter{equation}{125}

Let us consider the unitary representation of a spatial
rotation in the space \(\varepsilon=\varepsilon\left(j_{1}\right) \otimes \varepsilon\left(j_{2}\right)\)
\be
\hat{U}[\mathcal{R}(\theta, \phi)]=e^{-i / \hbar \phi \hat{J}_{z}} e^{-i / \hbar \theta \hat{J}_{y}}=e^{\ldots \hat{J}_{1} \ldots} \times e^{\ldots \hat{J}_{2} \ldots}
\ee
where $\vec{J} = \vec{J_1} + \vec{J_2}$.
The corresponding Wigner matrix is
\[
\begin{gathered}
\left\langle j m|\hat{U}[R]| j m^{\prime}\right\rangle=D_{m m^{\prime}}^{(j)}[R]\\
=
\sum_{m_1+m_2=m}\sum_{m_1^\prime+m_2^\prime=m^\prime}\bra{jm}\ket{j_1j_2m_1m_2}\\
\times
\underbrace{\bra{j_1j_2m_1m_2}\hat{U}[R]\ket{j_1j_2m_1^\prime m_2^\prime}}%
_{\Circled{1}}
\bra{j_1j_2m_1^\prime m_2^\prime}\ket{jm^\prime}
\end{gathered}
\]
where
\[
\Circled{1} = 
\bra{j_1j_2m_1m_2} 
e^{\ldots \hat{J}_{1} \ldots} \times e^{\ldots \hat{J}_{2} \ldots}
\ket{j_1j_2m_1^\prime m_2^\prime}
\]
is the Kronecker product of representations. Finally
\be
D^{(j)}_{mm^\prime}[R] =
\sum_{m_1m_2}\sum_{m_1^\prime m_2^\prime}
C^{j_1j_2}_{m_1m_2;jm}
C^{j_1j_2}_{m_1^\prime m_2^\prime;jm}
D^{(j_1)}_{m_1m_1^\prime}[R]
D^{(j_2)}_{m_2m_2^\prime}[R]
\label{eq:g127}
\ee
One gets a transparent interpretation of this product
writing it in terms of matrices instead of in terms
%%% 08 OKAY
of matrix elements, namely:
\begin{itemize}
\item The tensor product of the matrices \(D^{(j_{1})}[R]\) and \(D^{\left(j_{2}\right)}[R]\)
gives rise to a matrix \(\Delta[R]\) in the space \(\varepsilon\left(j_{1}\right) \otimes \varepsilon\left(j_{2}\right)\)
whose matrix elements are as
\be
\Delta_{m_{1} m_{2}, m_{1}^{\prime} m_{2}^{\prime}}[R]=D_{m_{1} m_{1}^{\prime}}^{\left(j_{1}\right)}[R] \,D_{m_{2} m_{2}^{\prime}}^{\left(j_{2}\right)}[R]
\ee
%
\item Eq.~\eqref{eq:g127} is then in the form of a change of
basis performed by a unitary (actually an orthogonal) matrix \(C\) with elements given by the CG
coefficients:
\be
\Delta[R] \rightarrow \Delta^\prime[R]=C \Delta[R] C^{-1}
\label{eq:g129}
\ee
%
\item Note that the matrix \(C\) \emph{does not} depend
on the parameters of the rotation matrix \((\theta, \phi)\).
%
\item The really important feature of the unitary
transformation in Eq.~\eqref{eq:g129} is that the
matrix \(\Delta[R]\) is brought into a block-ortogonal form.
More specifically, when considering the
matrices \(D^{(j)}=C D^{\left(j_{1}\right)} \otimes D^{\left(j_{2}\right)} C^{-1}\) for all
\(j\) in the range \(\left|j_{1}-j_{2}\right| \leqslant j \leqslant j_{1}+j_{2}\) in a
%%% 09 OKAY
single big matrix, that matrix is of the form
\be
\small
\begin{gathered}
\begin{matrix}
\hspace{5em} j_1+j_2 \hspace{3em} & j_1+j_2-1 \hspace{3em} & \cdots \hspace{3em} & |j_1-j_2|
\end{matrix}\\
\begin{matrix}
j_1+j_2 \\ j_1+j_2-1 \\ \vdots \\ |j_1-j_2|
\end{matrix}
\overbrace{\begin{pmatrix}
\left[D^{\left(j_{1}\right)} \otimes D^{\left(j_{2}\right)}\right]^{j=j_{1}+j_{2}} & \bigzero & \cdots   & \bigzero\\
\bigzero & \left[D^{\left(j_{1}\right)} \otimes D^{\left(j_{2}\right)}\right]^{j=j_{1}+j_{2}-1} & \cdots & \bigzero\\
\vdots & \vdots & \ddots & \vdots\\
\bigzero & \bigzero & \bigzero & \ddots\\
\end{pmatrix}}
\end{gathered}
\ee
where the zero-blocks represent matrices
with elements of the kind
\be
\langle j=j_{1}+j_{2}|
\underbrace{\cdots}%
_{\mathrlap{\text{commutes with } \hJtwo}\quad}
| j \neq j_{1}+j_{2}\rangle=0
\ee
%
\item One denotes this diagonalization as producing
a sum of irreducible representations
\be
D^{\left(j_{1}\right)} \otimes D^{\left(j_{2}\right)}=D^{\left(j_{1}+j_{2}\right)} \oplus
 D^{\left(j_{1}+j_{2}-1\right)} \oplus 
 \cdots \oplus 
 D^{\left(|j_{1}-j_{2}|\right)}
\ee
%
\item Why are they irreducible; that is, whey cannot
they be broken up into smaller blocks?
$\Rightarrow$ 
Each of such blocks is a matrix with
elements of an operator
\be
e^{-i / \hbar \phi \tilde{J}_{z}} e^{-i / \hbar \theta \tilde{J_{y}}}
\label{eq:g133}
\ee
%%% 10 OKAY
with \(\tilde{j}\) being or \(j_{1}+j_{2}\) or \(j_{1}+j_{2}-1\), or \ldots .
When Eq.~\eqref{eq:g133} is expanded in powers of \(\tilde{J}_{y}\), the
expansion terminates with \((\tilde{J}_{y} \sim \tilde{J}_{+}-\tilde{J}_{-})^{2 \tilde{j}}\).
Therefore, the operator \eqref{eq:g133} has just enough
raising \((\tilde{J}_{+})\)and lowering \((\tilde{J}_{-})\) operators so that
in acting on any state $\ket{\tilde{j}\tilde{m}}$ it will produce
a linear combination of all \((2 \tilde{j}+1)\) states and
thus populating the \emph{entire block}.\\
Moreover, \(D^{(j_1)} \otimes D^{(j_2)}\) was brought to a
block-diagonal form by a matrix \(C\)
that \emph{does not} depend on the parameters
of the rotation \(R\), \textit{i.e.} it does not depend
on \(\theta\) and \(\phi\).
\end{itemize}

To conclude this discussion, we remark that Eq.~\eqref{eq:g127} can be inverted,
\(D^{(j_{1})} \otimes D^{(j_{2})}\) can be expressed in terms of \(D^{(j)}\):
\be
D_{m_{1} m_{1}^{\prime}}^{\left(j_{1}\right)} D_{m_{2} m_{2}^{\prime}}^{\left(j_{2}\right)}=\sum_{m, m^{\prime}=-j}^{+j} 
C_{m_{1} m_{1}^{\prime} ; j m}^{j_{1} j_{2}} 
C_{m_{2} m_{2}^{\prime} ; j m^{\prime}}^{j_{1} j_{2}}
D_{m m^{\prime}}^{(j)}
\ee
where in this sum 
\(\left|j_{1}-j_{2}\right| \leqslant j \leqslant j_{1}+j_{2}\)

Next, we explore properties of physical quantities
under spatial rotation: this will be important for
constraining dynamics under symmetry requirements.

%%% 11 OKAY
\subsection{Spherical tensors}

(a.k.a. irreducible tensors).

Let $\hat{S}$ be an operator that is invariant under a
spatial rotation; namely, let $\ket{\varphi_R}$ be
\be
\ket{\varphi_R} = \hat{U}(R)\ket{\varphi}
\ee
then, invariance of $\hat{S}$ under rotation means
\be
\ev*{\hat{S}}{\varphi_R} = \ev*{\hat{U}^{-1}\hat{S}\hat{U}} {\varphi}= \ev*{\hat{S}}{\varphi}
\ee
which implies (for, \textit{e.g.} rotation angle $\theta$ around $\hat{n}$)
\be
e^{i / \hbar \theta \hat{n} \cdot \hat{\vec{J}}} \hat{S} e^{-i / \hbar \theta \hat{n} \cdot \hat{\vec{J}}}=\hat{S}
\ee
therefore
\be
\left[\hat{S},\hat{\vec{J}}\,\right] = 0
\ee
$\hat{S}$ is said to be a \emph{scalar} under rotations.

Similarly, one can obtain the behavior of a \emph{vector}
$\hat{\vec{V}} = (\hat{V_x},\hat{V_y},\hat{V_z})$ under a spatial rotation by
demanding that its expected value under an
arbitrary state vector transforms like its classical
%%% 12 OKAY
counterpart, \textit{i.e.}
\be
\ev*{\hat{V_i}}{\varphi_R} = [R_{\hat n}]_{ij}\ev*{\hat{V_j}}{\varphi}
\ee
so, with a rotation $\theta$ around $\hat{n}$
\be
e^{i / \hbar \theta \hat{n} \cdot \hat{\vec{J}}} \hat{V_i} e^{-i / \hbar \theta \hat{n} \cdot \hat{vec{J}}}=
[R_{\hat n}(\theta)]_{ij}\hat{V_j}
\ee
$\theta \to 0$:
\be
[R_{\hat n}(\theta)]_{ij}\hat{V_j}
\simeq V_i + \theta \varepsilon_{i j k} \hat{n}_{j} \hat{V}_{k}
\ee
\[
\begin{gathered}
\left(I+i / \hbar \theta \hat{n}_{k} \hat{J}_{k}\right) \hat{V}_{i}\left(I-i / \hbar \theta \hat{n}_{l} \hat{J}_{l}\right) \simeq V_{i}+\theta \varepsilon_{i j k} \hat{n}_{j} \cdot \hat{V}_{k} \to\\
i / \hbar\left(\hat{n}_{k} \hat{J}_{k} V_{i}-\hat{V}_{i} \hat{n}_{l} \hat{J}_{l}\right)=\varepsilon_{i j k} n_{j} \hat{V}_{k} \to\\
\hat{J}_{j} \hat{V}_{i}-V_{i} \hat{J}_{j}=-i \hbar \varepsilon_{i j k} V_{k} \xrightarrow{{i\leftrightarrow j}}
\end{gathered}
\]
\be
\boxed{
\left[\hat{J_{i}}, \hat{V_{j}}\right]=i \hbar \varepsilon_{i j k} V_{k}
}
\ee
\be
\text{Examples of } \hat{\vec{V}}: \hat{\vec{R}}, \hat{\vec{P}}, \underbrace{\vec{E}, \vec{B}}_{\text{e.m. fields}}
\ee

%%% 13 OKAY
There exist physical quantities with more than one
spatial index $i$; an example is the quadrupole moment
of a change distribution \(\rho(\vec{r})\), which carries indices $i,j$:
\be
Q_{i j}=\int d^{3} r\left(3 r_{i} r_{j}-|\vec{r}|^{2} \delta_{i}\right) \rho(\vec{r})
\ee
Such objects are known as \emph{tensors}. A generic
tensor operator \(\hat{T}\) has \emph{cartesian} components \(i, j, k, \ldots\) :
\be
\hat{T}_{i j k\ldots}
\ee
Their transformation properties under a spatial rotation are
obtained by postulating that their expectation
values $\ev*{\hat{T}_{ijk\ldots}}{\varphi_R}$ transform like their classical
counterparts, \textit{i.e.} for each \emph{cartesian} index one associates
a rotation matrix \(R\) like for the cartesian index of
a vector, \textit{i.e.}
\be
\ev*{\hat{T}_{ijk\ldots}}{\varphi_R} = R_{ii^\prime}R_{jj^\prime}R_{kk^\prime}\cdots
\ev*{\hat{T}_{i^\prime j^\prime k^\prime\ldots}}{\varphi}
\ee
Since $\ket{\varphi}$ is arbitrary:
\be
e^{i/\hbar \hat{n}\cdot\hat{\vec{J}}}
\hat{T}_{ij\ldots}
e^{-i/\hbar \hat{n}\cdot\hat{\vec{J}}}
=
R_{ii^\prime}R_{jj^\prime}\ldots\hat{T}_{i^\prime j^\prime\ldots}
\ee
The analysis of such on expression is facilitated by
employing ``spherical'' components \(\Rightarrow\) \emph{spherical tensors}.

%%% 14 OKAY

$T_q^{(k)}$: spherical tensor (or irreducible tensor, or Racah tensor).
\begin{itemize}
\item $k$: \emph{rank} (or order), $T_{kq}$ has $2k+1$ \emph{components} $q$
\item $q$: $-k, -k+1, \ldots, k-1, k$
\item $k$: integer or half-integer
\item $\hat{T}_q^{(k)}$ transforms under a spatial rotation as
\be
\hat{T}_q^{(k)\prime} = \hat{U}_{\hat{n}}^{-1} \hat{T}_q^{(k)} \hat{U}_{\hat{n}} = \sum_{q^\prime} D_{q^\prime q}^{(k)} \hat{T}_{q^\prime}^{(k)}
\label{eq:g148}
\ee
\end{itemize}
with $D_{q^\prime q}^{(k)}$ given by
\[
\begin{aligned}
D_{q^\prime q}^{(k)}
&= \bra{kq^\prime}\hat{U}_{\hat{n}}[R]\ket{kq}\\
&= \bra{kq^\prime}e^{-i\hbar\theta\hat{n}\cdot\vec{J}}[R]\ket{kq}\\
\end{aligned}
\]
where $\ket{kq}$ are the eigenstates of $\hJtwo$ and $\hat{J_0}$
\begin{align}
\hJtwo \ket{kq} &= k(k+1) \hbar^2 \ket{kq}\\
\hat{J_0} \ket{kq} &= q\hbar \ket{kq}
\end{align}
\textit{i.e.} $k \sim j $ and $q \sim m$. 
For an infinitesimal $\theta$
\be
[\hat{n}\cdot\hat{\vec{J}}, \hat{T}_q^{(k)}]
= \sum_{q^\prime} \hat{T}_q^{(k)}
\bra{kq^\prime}\hat{n}\cdot\hat{\vec{J}}\ket{kq}
\ee
in terms of the components of $\hat{\vec{J}} = (\hat{J}_\pm, \hat{J}_0)$, this is:
\begin{align}
[\hat{J}_0, \hat{T}_q^{(k)}]   &= \hbar q\,\hat{T}_q^{(k)}\\
[\hat{J}_\pm, \hat{T}_q^{(k)}] &= \hbar \sqrt{k(k+1)-q(q \pm 1)}\,\hat{T}_{q\pm1}^{(k)}\label{eq:g152}
\end{align}

%%% 15 OKAY

Note that the definition in Eq.~\eqref{eq:g148} that the states $\ket{\tau,jm}$ are themselves spherical tensors, as
\be
\begin{aligned}
\ket{jm}^\prime = \hat{U}[R]\ket{jm}
&=\sum_{m^\prime}\op{jm^\prime}\hat{U}[R]\ket{jm}\\
&=D_{m^{\prime}m}^{(j)}[R]\ket{jm^\prime}
\end{aligned}
\ee
\emph{Note also} that Eq.~\eqref{eq:g152} provides a selection rule:
\[
\begin{aligned}
\bra{\tau,jm}[\hat{J}_0, \hat{T}_q^{(k)}]\ket{j^\prime m^\prime}
&=\hbar q \bra{\tau,jm}\hat{T}_q^{(k)}\ket{j^\prime m^\prime}\\
&= \hbar(m-m^\prime)\bra{\tau,jm}\hat{T}_q^{(k)}\ket{j^\prime m^\prime}
\end{aligned}
\]
therefore
\be
m-m^\prime=q
\ee
important, \textit{e.g.} in electromagnetic transitions in atoms.
\end{document}
