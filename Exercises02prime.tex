%!TEX TS-program = xelatex
%!TEX encoding = UTF-8 Unicode

\documentclass[12pt]{article}
\usepackage{geometry}                % See geometry.pdf to learn the layout options. There are lots.
\geometry{a4paper,top=1.8cm,bottom=2cm,right=2.7cm}
\usepackage[parfill]{parskip}    % Activate to begin paragraphs with an empty line rather than an indent
\usepackage{graphicx}
\usepackage{amsmath}
\usepackage{amssymb}
\usepackage{mathtools}
\usepackage{physics}
\newcommand{\be}{\begin{equation}}
\newcommand{\ee}{\end{equation}}
\usepackage[thicklines]{cancel}
\usepackage{url}
\usepackage{booktabs}

\usepackage{fontspec,xltxtra,xunicode}
\defaultfontfeatures{Mapping=tex-text}

\newcommand{\polv}{\ensuremath{\updownarrow}}
\newcommand{\polh}{\ensuremath{\leftrightarrow}}
\newcommand{\poldr}{\rotatebox[origin=c]{45}{\ensuremath{\leftrightarrow}}}
\newcommand{\poldl}{\rotatebox[origin=c]{-45}{\ensuremath{\leftrightarrow}}}

\title{Advanced Quantum Mechanics\\Exercises Set 2$^{\,\prime}$\\(Fill the gaps)\vspace{-0.5em}}
%\author{The Author}
\date{Due date: June 30\textsuperscript{th}, 2023}

\begin{document}
\maketitle


\textbf{Exercise 1:} Show that a state \(|\Psi\rangle\) of a bipartite system is a product state if and only if it has Schmidt number $n_s$ =  1.


\textbf{Exercise 2:} Show that a state \(|\Psi(1,2)\rangle\) of a bipartite system is a product state if and only if \(\hat{\rho}^{(1)}=\)
\(\operatorname{Tr}_{2}|\Psi(1,2)\rangle\) (and thus also \(\left.\hat{\rho}^{(2)}=\operatorname{Tr}_{1}|\Psi(1,2)\rangle\right)\) is a pure state.


\textbf{Exercise 3:} Let \(\hat{\rho}^{(1)}\) be a state operator in Hilbert space \({H}_{1}\) written in terms of normalized state vectors
\(\left\{\left|\varphi_{i}\right\rangle, i=1, \cdots, n\right\}\) as
\be
\hat{\rho}^{(1)}=\sum_{i=1}^{n} p_{i}\left|\varphi_{i}\right\rangle\left\langle\varphi_{i}\right| \quad \text{ with }\quad p_{i} \geq 0 \quad \text{ and } \quad \sum_{i=1}^{n} p_{i}=1
\ee
Use a set of mutually orthogonal and normalized vectors \(\left\{\left|\chi_{i}\right\rangle, i=1, \cdots, n\right\}\) from a Hilbert space
\({H}_{2}\) to construct an entangled pure bipartite state \(|\Psi(1,2)\rangle\) in \({H}_{1} \otimes {H}_{2}\) such that
\be
\hat{\rho}^{(1)}=\operatorname{Tr}_{2} \hat{\rho}^{(12)}, \quad \text { with } \quad \hat{\rho}^{(12)}=|\Psi(1,2)\rangle\langle\Psi(1,2)|
\ee
\(\hat{\rho}^{(12)}\) is then an extension of \(\hat{\rho}^{(1)}\).


\textbf{Exercise 4:} Show that two purifications \(|\Psi(1,2)\rangle\) and \(|\Phi(1,2)\rangle\) in \({H}_{1} \otimes {H}_{2}\) differ by a unitary operator.


\textbf{Exercise 5:} Let \(\hat{\rho}^{(12)}\) be a general state operator (\emph{not necessarily pure}) in a tensor-product Hilbert space
\({H}_{1} \otimes {H}_{2}\). Show that if \(\hat{\rho}^{(1)}\) is a pure state, then \(\hat{\rho}^{(12)}\) is a product \(\hat{\rho}^{(1)} \otimes \hat{\rho}^{(2)}\), with \(\hat{\rho}^{(2)}\) in \({H}_{2}\)
(where \(\hat{\rho}^{(2)}\) can be mixed). Hint: Choose an arbitrary purification of \(\hat{\rho}^{(12)}\) and make use of
the result of Exercise 2.


\textbf{Exercise 6 (Monogamy of entanglement):} 
Suppose that \(\hat{\rho}^{(12)}=|\Psi(1,2)\rangle\langle\Psi(1,2)|\) is an entangled state of
particles 1 and 2 . Use the result of Exercise 5 to show that an extension \(\hat{\rho}^{(123)}\) of \(\hat{\rho}^{(12)}\) would imply
that particles 1 and 2 are both uncorrelated with particle 3, that is, if particles 1 and 2 are
entangled, each of then cannot be entangled with particle 3.

Monogamy of entanglement is valid even when \(\hat{\rho}^{(12)}\) is not pure; but we will not deal with
entanglement for general quantum states (\textit{i.e.} not necessarily pure).

\end{document}

\textbf{Exercise 7 -- No faster-than-light communication, no-cloning theorem:}

\emph{Preamble:} the aim of this exercise is to show that the possibility of cloning an unknown state would
allow faster-than-light information transmission. To exemplify what we mean by that, let us
suppose that Alice and Bob share a two-level entangled state; for definiteness, consider an
ensemble of two spin-1/2 particles prepared in the maximally entangled \(\left|\Psi_{-}\right\rangle\) Bell state (recall
an exercise of a previous list):

\be
\left|\Psi_{-}\right\rangle=\frac{1}{\sqrt{2}}(|+\otimes-\rangle-|-\otimes+\rangle)
\ee
where \(|+\rangle\) and \(|-\rangle\) are the eigenstates of the \(\sigma_{z}\) Pauli matrix. One of the particles is analysed
in Alice's laboratory and the other in Bob's laboratory; both laboratories can be light-years
apart.

Alice wants to send a message to Bob through a binary string, combinations of zeros (0)
and ones (1). Alice's idea to send the message is to perform measurements on her spins,
measuring \(\sigma_{x}\) for sending the digit 0 and \(\sigma_{z}\) for sending digit 1 (of course, for this to work in
principle Alice and Bob must share at least as many entangled pairs as digits in the message).
If Alice measures \(\sigma_{x}\), Bob's state collapses to \(|\pm 1, \hat{x}\rangle\) (he receives the digit 0), if she measures \(\sigma_{z}\),
Bob's state collapses to \(|\pm 1, \hat{z}\rangle\) (he receives digit 1 ). The problem with Alice's idea is that the
states \(|\pm 1, \hat{x}\rangle\) and \(|\pm 1, \hat{z}\rangle\) are not orthogonal; hence Bob cannot obtain any information about
Alice's measurements (and therefore the sequence of digits 0 and 1). You will prove all of this
below.

Now, if Bob could make as many copies as he wishes of his states, he could distinguish
between \(|\pm 1, \hat{x}\rangle\) and \(|\pm 1, \hat{z}\rangle\). Why? Because an arbitrary qubit, \(|\varphi\rangle=\lambda|+\rangle+\mu|-\rangle\), can be
reconstructed (i.e. \(\lambda\) and \(\mu\) determined) once we have a large number of copies of the state
upon which to perform measurements. Therefore, Bob could determine whether the state is
\(|\pm 1, \hat{x}\rangle\) or \(|\pm 1, \hat{z}\rangle\), even when Alice's and Bob's measurements have no causal connection,\textit{i.e.}
even when their measurements are spacelike separated, a situation characterizing faster-than-
light communication.

The point is: independently of Alice's choice of axis to measure the spin of the particle in
her laboratory, Bob obtains with probabilities equal to 1/2 the values \(\pm 1\) in his measurements in
his laboratory, whatever axis he uses. Therefore, Bob always obtains randomly \(\pm 1\), indicating
that no information has been transmitted from Alice (\textit{i.e.} the axis she used) to Bob.

\textbf{(a)} First, express \(\left|\Psi_{-}\right\rangle\)in the basis \(\{|+, \hat{n}\rangle,|-, \hat{n}\rangle\}\) of spin-1/2 projection on an arbitrary axis \(\hat{n}-\) see Eq.~(3.56) of Le~Bellac. You should find (up to a global phase):
\be
\left|\Psi_{-}\right\rangle=\frac{1}{\sqrt{2}}(|+\otimes-\rangle-|-\otimes+\rangle)=\frac{1}{\sqrt{2}}(|+, \hat{n} \otimes-, \hat{n}\rangle-|-, \hat{n} \otimes+, \hat{n}\rangle)
\ee

\textbf{(b)} Let \(\hat{a}\) indicate Alice's axis and \(\hat{b}\) Bob's axis. Eq. (4) shows that when Alice measures the
spin projection along the direction \(\hat{a}\), she finds \(\pm 1\) with probability 1/2, and Bob's particle
state is then one of the two qubits
\be
|\text{Bob}\rangle=|\mp, \hat{a}\rangle
\ee
where the \(\mp\) sign corresponds to the \(\pm 1\) Alice's result -- note that Bob obtains these states
with probability 1/2. Given this, show that the probability of Bob finding \(+1\) or \(-1\) when he
measures in his laboratory the spin along an axis \(\hat{b}\) is equal to 1/2. This proves that there is
no information exchange between Alice and Bob, as mentioned above.

\textbf{(c)} Finally, suppose that Bob could make clones of states \(|\text{Bob}\rangle=\), \textit{i.e.} make as many copies as
he wishes of his particle's state. Is it clear to you that this would allow him to instantly know
the axis Alice used, even if their measurements are spacelike separated?
\end{document}