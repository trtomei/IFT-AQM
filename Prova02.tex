%!TEX TS-program = xelatex
%!TEX encoding = UTF-8 Unicode

\documentclass[12pt]{article}
\usepackage{geometry}                % See geometry.pdf to learn the layout options. There are lots.
\geometry{a4paper,top=2cm}
\usepackage[parfill]{parskip}    % Activate to begin paragraphs with an empty line rather than an indent
\usepackage{graphicx}
\usepackage{amsmath}
\usepackage{amssymb}
\usepackage{physics}
\usepackage{polyglossia}
\setdefaultlanguage{portuguese}

\usepackage{fontspec,xltxtra,xunicode}
\defaultfontfeatures{Mapping=tex-text}

\title{Mecânica Quântica Avançada\\%
Prova 02\\%
27 de setembro -- 03 de Novembro}
%\author{The Author}
\date{}

\begin{document}
\maketitle
\vspace*{-4em}

\textbf{Exercício 1:} 

O grupo $SU(2)$ é o grupo de matrizes unitárias de determinante 1;

a. Mostre que se $U \in SU(2)$, então $U$ tem a forma
\[
U=\left(\begin{array}{cc}a & b \\ -b^{*} & a^{*}\end{array}\right), \quad|a|^{2}+|b|^{2}=1.
\]

b. Mostre que na vizinhança da identidade, nós podemos escrever
\[
U=I-\mathrm{i} \tau \quad \text{ com } \quad \tau=\tau^{\dagger}
\]
e que $\tau$ é expressado como função das matrizes de Pauli como
\[
\tau=\frac{1}{2} \sum_{i=1}^{3} \theta_{i} \sigma_{i}, \quad \theta_{i} \rightarrow 0.
\]

c. Vamos tomar $\theta=\left(\sum_{i} \theta_{i}^{2}\right)^{1 / 2}$ e $\theta_{i}=\theta \hat{n}_{i}$, onde $\hat{n}$ é um vetor de norma 1. Supondo que os $\theta_1$ são finitos nós definimos $U_{\hat{n}}(\theta)$ como
\[
U_{\hat{n}}(\theta)=\lim _{N \rightarrow \infty}\left[U_{\hat{n}}\left(\frac{\theta}{N}\right)\right]^{N}.
\]
Mostre que 
\[
U_{\hat{n}}(\theta)=\mathrm{e}^{-\mathrm{i} \theta \vec{\sigma} \cdot \hat{n} / 2}.
\]

d. Seja $\vec{V}$ um vetor de $\mathbb{R}^3$ e $\mathcal{V}$ uma matriz hermitiana de traço zero:
\[
\mathcal{V}=\left(\begin{array}{cc}V_{z} & V_{x}-\mathrm{i} V_{y} \\ V_{x}+\mathrm{i} V_{y} & -V_{z}\end{array}\right)=\vec{\sigma} \cdot \vec{V}.
\]
Qual o determinante de $\mathcal{V}$? Seja $\mathcal{W}$ a matriz
\[
\mathcal{W}=U \mathcal{V} U^{-1}
\]
onde $U \in SU(2)$.
Mostre que $\mathcal{W}$ é da forma $\vec{\sigma} \cdot \vec{W}$ e que $\vec{W}$ é obtido de $\vec{V}$ por uma rotação.

e. Vamos definir $\vec{V}(\theta)$ como
\[
\vec{\sigma} \cdot \vec{V}(\theta)=U_{\hat{n}}(\theta)[\vec{\sigma} \cdot \vec{V}] U_{\hat{n}}^{-1}(\theta), \quad \vec{V}(\theta=0)=\vec{V}.
\]
Mostre que
\[
\frac{\mathrm{d} \vec{V}(\theta)}{\mathrm{d} \theta}=\hat{n} \times \vec{V}(\theta).
\]
Mostre que $\vec{V}(\theta)$ é obtido de $\vec{V}$ por uma rotação de um ângulo $\theta$ ao redor de $\hat{n}$. Este resultado estabelece uma
correspondência entre as matrizes $\mathcal{R}_{\hat{n}}(\theta)$ de $SO(3)$ e as matrizes $U_{\hat{n}}(\theta)$ of $SU(2)$. Esta correspondência é um-pra-um ou dois pra um?


\textbf{Exercício 2:} 

Os boosts de Galileu formam um subgrupo de um grupo maior,
de 10 dimensões chamado grupo de Galileu de transformações de espaço-tempo:

\[
\begin{aligned}
\vec{x} \rightarrow \vec{x}^{\,\prime}&=R \vec{x}+\vec{a}+\vec{v} t \\ 
t \rightarrow t^{\prime}&=t+s
\end{aligned}
\]

onde além do deslocamento \(\vec{a}\) e da velocidade do boost \(\vec{v}\) já estudados, também temos
uma rotação espacial \(R\) e um deslocamento temporal \(s\). Seja \(g=(R, \vec{a}, \vec{v}, s)\) uma transformação desse tipo.

Mostre que a lei de composição para \(g_{3}=g_{2} g_{1}\), com \(g_{3}=\left(R_{3}, a_{3}, v_{3}, s_{3}\right)\) é:
\[
\begin{aligned} 
R_{3} &=R_{2} R_{1} \\ 
\vec{a}_{3} &=\vec{a}_{2}+R_{2} \vec{a}_{1}+\vec{v}_{2} s_{1} \\ 
\vec{v}_{3} &=\vec{v}_{2}+R_{2} \vec{v}_{1} \\ 
s_{3} &=s_{2}+s_{1} 
\end{aligned}
\]


\textbf{Exercício 3:}

a. Use as relações
\[
\begin{aligned}
\langle j^{\prime} m^{\prime}|\vec{J}^{2}| j m\rangle &=j(j+1) \delta_{j^{\prime} j} \delta_{m^{\prime} m} \\
\langle j^{\prime} m^{\prime}|J_{0}| j m\rangle &=m \delta_{j^{\prime} j} \delta_{m^{\prime} m} \\
\langle j^{\prime} m^{\prime}|J_{\pm}| j m\rangle &=[j(j+1)-m m^{\prime}]^{1 / 2} \delta_{j^{\prime} j} \delta_{m^{\prime}, m \pm 1}
\end{aligned}
\]
para encontrar os operadores $S_{x}, S_{y}$, and $S_{z}$ para spin $1 / 2$

b. Novamente usando essas relações, calcule as representações de matriz $3 \times 3$ 
de $J_{x}, J_{y}$, e $J_{z}$ para momento angular $j=1$.

c. Mostre que para $j=1$, $J_{x}, J_{y}$, and $J_{z}$ estão relacionados aos geradores infinitesimais $T_{x}, T_{y}$, e $T_{z}$ 
\[
T_{x}=\left(\begin{array}{ccc}0 & 0 & 0 \\ 0 & 0 & -\mathrm{i} \\ 0 & \mathrm{i} & 0\end{array}\right), \quad 
T_{y}=\left(\begin{array}{ccc}0 & 0 & \mathrm{i} \\ 0 & 0 & 0 \\ -\mathrm{i} & 0 & 0\end{array}\right), \quad 
T_{z}=\left(\begin{array}{ccc}0 & -\mathrm{i} & 0 \\ \mathrm{i} & 0 & 0 \\ 0 & 0 & 0\end{array}\right)
\]
por uma transformação unitária que leva as componentes Cartesianas do vetor unitário $\hat{r}$ às suas componentes esféricas
\[
\begin{array}{c}
\hat{r}=(\sin \theta \cos \phi, \sin \theta \sin \phi, \cos \theta) \\ 
Y_{1}^{0}=\sqrt{\frac{3}{4 \pi}} \hat{r}_{0}, \quad Y_{1}^{\pm}=\mp \sqrt{\frac{3}{4 \pi}} \frac{\hat{r}_{x} \pm \mathrm{i} \hat{r}_{y}}{\sqrt{2}}=\sqrt{\frac{3}{4 \pi}} \hat{r}_{\pm 1}
\end{array},
\]
\textit{i.e.} $J_{i}=U^{\dagger} T_{i} U$ com
\[
U=\frac{1}{\sqrt{2}}\left(\begin{array}{ccc}-1 & 0 & 1 \\ -\mathrm{i} & 0 & -\mathrm{i} \\ 0 & \sqrt{2} & 0\end{array}\right)
\]

d. Calcule a matriz de rotação $d^{(1)}(\theta)$:
\[
d^{(1)}(\theta)=\exp \left(-\mathrm{i} \theta J_{y}\right)
\]
e verifique que ela tem a forma.
\[
d^{(1)}(\theta)=\left(\begin{array}{ccc}\frac{1}{2}(1+\cos \theta) & -\frac{1}{\sqrt{2}} \sin \theta & \frac{1}{2}(1-\cos \theta) \\ \frac{1}{\sqrt{2}} \sin \theta & \cos \theta & -\frac{1}{\sqrt{2}} \sin \theta \\ \frac{1}{2}(1-\cos \theta) & \frac{1}{\sqrt{2}} \sin \theta & \frac{1}{2}(1+\cos \theta)\end{array}\right).
\]
Dica: Mostre que $J_{y}^{3}=J_{y}$.

\textbf{Exercício 4:}

Sejam $\ket{j,m}$ os autoestados de $\hat{\vec{J}}^2$ and $\hat{J}_0 = \hat{J}_z$,
e seja
\[
\chi=\langle 2,2|\hat{x} \hat{y}| 0,0\rangle
\] 
Usando o teorema de Wigner-Eckart, calcule como função de $\chi$ os elementos de matriz:
\[
\langle 2, m|\hat{\mathcal{O}}| 0,0\rangle
\]
para cada um dos seguintes operadores $\mathcal{O}$:
a) $\hat{x}\hat{y}$ b) $\hat{x} \hat{z}$, c) $\hat{y} \hat{z}$, d) $\hat{x}^{2}$ e) $\hat{y}^{2}$ e f) $\hat{z}^{2}$.
Veja que em princípio você poderia calcular os elementos de matriz diretamente, mas
\textbf{o ponto do exercício é utilizar o teorema de Wigner-Eckart várias vezes}.


\textbf{Exercício 5:} 

O Hamiltoniano de um sistema de spin 1 é
\[
H=A S_{z}^{2}+B\left(S_{x}^{2}-S_{y}^{2}\right).
\]

a. Resolva este problema \textbf{exatamente} para encontrar os autovalores e os autoestados de energia normalizados.

b. Este Hamiltoniano é invariante por reversão temporal? Demonstre sua resposta.

c. Como os autoestados normalizados que você obteve se transformam por reversão temporal?
\end{document}