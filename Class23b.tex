%!TEX TS-program = xelatex
%!TEX encoding = UTF-8 Unicode

\documentclass[12pt]{article}
\usepackage{geometry}                % See geometry.pdf to learn the layout options. There are lots.
\geometry{a4paper,top=2cm}
\usepackage[parfill]{parskip}    % Activate to begin paragraphs with an empty line rather than an indent
\usepackage{graphicx}
\usepackage{amsmath}
\usepackage{amssymb}
\usepackage{mathtools}
\usepackage{physics}
\newcommand{\be}{\begin{equation}}
\newcommand{\ee}{\end{equation}}
\usepackage[thicklines]{cancel}
\usepackage{url}
\usepackage{booktabs}
\usepackage{csquotes}
\usepackage{qcircuit}
\usepackage{circledsteps}
\usepackage{nicefrac}
\usepackage{fontspec,xltxtra,xunicode}
\usepackage{xcolor}
\defaultfontfeatures{Mapping=tex-text}

\newcommand{\polv}{\ensuremath{\updownarrow}}
\newcommand{\polh}{\ensuremath{\leftrightarrow}}
\newcommand{\poldr}{\rotatebox[origin=c]{45}{\ensuremath{\leftrightarrow}}}
\newcommand{\poldl}{\rotatebox[origin=c]{-45}{\ensuremath{\leftrightarrow}}}
\newcommand{\bigzero}{\mbox{\normalfont\Large\bfseries 0}}

\title{Advanced Quantum Mechanics\\Class 23 (b)}
%\author{The Author}
\date{November 8, 2022}                                           % Activate to display a given date or no date

\begin{document}
\maketitle

%%% 16 OKAY
\setcounter{section}{6}
\setcounter{equation}{35}

\section{Approximation Methods}

\setcounter{subsection}{1}
\subsection{Time-dependent perturbation theory}

$\Rightarrow$ Fermi's Golden Rule

Statement of the problem: let $\hat{H} = \hat{H}(t)$ be a time-dependent Hamiltonian,
that can be split as
%%% 17 OKAY
\be
\hat{H}(t) = \hat{H}_0 + \lambda W(t)
\label{eq:g36}
\ee
$\hat{H}_0$ time-independent, with a known spectrum, given
by Eq.~(2). Suppose that at $t=0$ the system is
in the initial state
\be
\ket{\psi(t=0)} \equiv \ket{i}
\ee
which is one of the eigenstates of $\hat{H}_0$
\be
\hat{H}_0 \ket{i} = E_i \ket{i}
\ee
We want to know the probability $p_{i \rightarrow f}(t)$ of finding
the system in the eigenstate $|f\rangle$ of $\hat{H}_{0}$:
\be
\hat{H}_0 \ket{f} = E_f \ket{f}
\ee
The answer is
\be
p_{i \rightarrow f}(t) = |\bra{f}\ket{\psi(t)}|^2
\label{eq:g40}
\ee
where $\ket{\psi(t)}$ is the solution of the Schrödinger equation
for the Hamiltonian in Eq.~\eqref{eq:g36}.
We have seen a similar problem when
studying two-level systems; here that
problem is generalized to many-level problems.

%%% 18 OKAY
Let us decompose $|\psi(t)\rangle$ on the basis $\{\varphi_{0}^{(l)} \equiv|l\rangle\}$
of the eigenstates of $\hat{H}_{0}$:
\be
|\psi(t)\rangle=\sum_{l=0}^{\infty} C_{l} (t)|l\rangle, \quad C_{l}(t)=\langle l | \psi(t)\rangle
\ee
Applying $\hat{H}_{0}$ to this and projecting onto a state
$|\varphi_{0}^{(n)}\rangle \equiv |n\rangle$, one obtains:
\[
\begin{aligned}
\bra{n}\hat{H}_0\ket{\psi(t)} 
&= \sum_l \bra{n}\hat{H}_0\ket{l} C_l(t) = E_n C_n(t) \Rightarrow\\
\underbrace{E_n^{0}}_{\equiv E_n} \bra{n}\ket{\psi(t)}  
&= \quad\quad\quad\! (\hat{H}_0)_{nl} \, C_l(t) =  E_n C_n(t)
\end{aligned}
\]
therefore
\be
\sum_l (\hat{H}_0)_{nl} \, C_l(t)  =  E_n C_n(t)
\label{eq:g42}
\ee
The $C_l(t)$ are solutions of a coupled set of differential
equations obtained from the Schrödinger equation
for $\ket{\psi(t)}$, namely
\be
i\hbar \frac{d}{d t}|\psi(t)\rangle=\left[\hat{H}_{0}+\lambda \hat{W}(t)\right](\psi(t)\rangle
\ee
$\Rightarrow$ projecting onto $\ket{n}$:
%%% 19 OKAY
\be
\begin{aligned}
i\hbar \dot{C}_n(t) 
&= E_n C_n(t) + \lambda \bra{n} \hat{W}(t)
\left(\sum_l \op{l}\right)\ket{\psi}\\
\text{using Eq.~\eqref{eq:g42}}\\
&= E_n C_n(t) + \lambda \sum_l W_{nl}(t) C_l(t)
\end{aligned}
\ee
and one can eliminate the $E_n C_n(t)$ term by writing
\be
C_n(t) = \exp(-i/\hbar E_n(t)) \gamma_n(t)
\ee
hence
\be
i \hbar \dot{\gamma}_{n}(t)=\lambda \sum_{l} e^{i \omega_{n l} t} W_{n l}(t) \gamma_{l}(t)
\label{eq:g46}
\ee
where
\be
\omega_{n l}=\frac{E_{n}-E_{l}}{\hbar}
\label{eq:g47}
\ee
In general, it is not possible to find analytical
solutions to the infinite set of differential equations
in Eq.~\eqref{eq:g46} $\Rightarrow$ use of time-dependent perturbation
theory.

The following is \emph{limited to $\mathcal{O}(\lambda)$} calculation.
At time $t=0$, assume that the system is in state $\ket{i}$
\be
\ket{\psi(t=0)} = \ket{i} \therefore \ket{\psi(t=0)} = \sum_n C_n(0) \ket{n}
\ee
$\Rightarrow C_n(0) = \bra{i}\ket{n} = \delta_{in} \Rightarrow \gamma_n(0) = \delta_{ni}$.
%%% 20 OKAY
From Eq.~\eqref{eq:g46}, one sees that
\be
\gamma_n (t) = \delta_{ni} + \gamma_n^{(1)}(t) + \ldots
\ee
For $t$ small, one expects $|\gamma_{n}^{(1)}| \ll 1$ : system did not
have enough time for $\gamma_{n}^{(1)}$ to grow much. Therefore,
substitute $\gamma_n(t) \simeq \gamma_n(0) = \delta_{ni}$ on the right-hand-side:
\be
\begin{aligned}
i \hbar \dot{\gamma}_{n}^{(1)}(t)
&=\lambda \sum_{l} e^{i \omega_{n l} t} W_{nl}(t) \delta_{l i}\\
&=\lambda e^{i \omega_{n i} t} W_{ni}(t)
\end{aligned}
\label{eq:g50}
\ee

\subsubsection{Time-dependent perturbation: oscillating field}

Let us consider the \emph{special case} where
\be
\hat{W}(t) = \hat{A}e^{-i\omega t} + \hat{A}^\dagger e^{i\omega t}
\ee
where $\hat{A}$ is time-independent.
This is the relevant case for \emph{interaction of radiation with matter}:
an oscillating field $\hat{W}(t)$ interacts with a (many-level) atom, and excites/ionizes it.
Eq.~\eqref{eq:g50} leads to:
\be
\begin{aligned} 
i \hbar \dot{\gamma}_{n}^{(1)}(t) 
&=\lambda e^{i \omega_{n i} t}\left(A_{n i} e^{-i \omega t}+A_{i n}^{*} e^{i \omega t}\right) \\
&=\lambda\left[A_{n i} e^{-i\left(\omega-\omega_{n i}\right) t}+A_{i n}^{*} e^{i\left(\omega+\omega_{n i}\right) t}\right] 
\end{aligned}
\ee
where $A_{ni} = \bra{n}\hat{A}\ket{i}$.
Integrating both sides from $t=0$ to $t$,
one obtains:
%%% 21 OKAY
\be
\begin{aligned}
i \hbar \gamma_{n}^{(1)} (t) 
&=\lambda\left[
\frac{e^{-i\left(\omega-\omega_{n i}\right) t}-1}{-i\left(\omega-\omega_{n i}\right)} A_{n i}
+
\frac{e^{i\left(\omega-\omega_{n i}\right) t}-1}{i\left(\omega+\omega_{n i}\right)} A_{i n}^{*}\right]
\end{aligned}
\ee
We want to consider a transition into a well-defined
final energy eigenstate of $\hat{H}_{0}$, $\ket{f} \neq \ket{i}$. The
transition probability amplitude is, from
Eq.~\eqref{eq:g40}, given by
\be
\begin{aligned}
\bra{f}\ket{\psi(t)}
&= \sum_l C_l(t) \bra{f}\ket{l}\\
&= e^{-i/\hbar E_f t} \gamma_f^{1}(t)
\end{aligned}
\ee
where $C_l(t) = e^{-i/\hbar E_l t}$.
Up to an overall phase factor, the transition probability
amplitude can be written as
\be
\langle f | \psi(t)\rangle \sim \gamma_{f}^{(1)}(t) 
=\frac{\lambda}{\hbar}
\left[A_{f i} \frac{e^{-i\left(\omega-\omega_{0}\right) t}-1}{\omega-\omega_{0}}\right.-A_{l f}^{*} \left.\frac{e^{i\left(\omega+\omega_{0}\right) t}-1}{\omega+\omega_{0}}\right] 
\ee
where (from Eq.~\eqref{eq:g47})
\be
\omega_0 = \frac{E_f - E_i}{\hbar}
\ee

%%% 22 OKAY

The amplitude is more important in on-resonance
situations, namely:
\be
\omega \approx \pm \omega_{0} \Rightarrow E_{f} \simeq E_{i} \pm \hbar \omega
\ee
$\Rightarrow$ system absorbs ($+$) or emits ($-$) a photon
of energy $\hbar \omega$.
Consider the case of absorption:
the time-dependent transition probability is
\setcounter{equation}{56}
\be
\begin{aligned}
p_{i \rightarrow f}(t)
&=\left|\gamma^{(n}(t)\right|^{2}\\
&=\frac{\lambda^{2}}{\hbar^{2}} \frac{\left|A_{f i}\right|^{2}}{\left(\omega-\omega_{0}\right)^{2}}
\left(e^{-i\left(\omega-\omega_{0}\right) t}-1\right)\left(e^{i\left(\omega-\omega_{0}\right) t}-1\right)\\
&=\frac{\lambda^{2}}{\hbar^{2}} \frac{\left|A_{f i}\right|^{2}}{\left(\omega-\omega_{0}\right)^{2}}
\left[2-2\cos(\omega-\omega_0)(t)\right]\\
&=\frac{\lambda^{2}}{\hbar^{2}} \left|A_{f i}\right|^{2} t^2 \times
\underbrace{\frac{\sin^2\left[(\omega-\omega_0)(t/2)\right]}{\left[(\omega-\omega_{0})(t/2)\right]^{2}}}%
_{f(\omega-\omega_0,t)}
\end{aligned}
\ee
where $f(\omega-\omega_0,t) \xrightarrow{\omega\,\approx\,\omega_0} \frac{2\pi}{t} \delta(\omega-\omega_0)$.
Therefore
\be
p_{i \rightarrow f}(t) = \frac{\lambda^{2}}{\hbar^{2}}\left|A_{f i}\right|^{2} t^2 f(\omega-\omega_0,t)
\label{eq:g58}
\ee
where $p_{i \rightarrow f}(t)$ is related to the transition probability per unit time $\Gamma_{i\to f}$:
\[
\Gamma_{i\to f} = \frac{p_{i \rightarrow f}}{t} = \frac{\lambda^{2}}{\hbar^{2}}\left|A_{f i}\right|^{2} t \times f(\omega-\omega_0,t).
\]

%%% 23 OKAY

This result was derived using perturbation theory,
based on the assumption of $\lambda$ ``small'' and
short times $\to$ 
Eq.~\eqref{eq:g58} is valid only when $p_{i \rightarrow f}(t) \ll 1$.

In practice, not always an isolated final state
$\ket{f}$ is accessible; one needs then to sum over
many or all final states:
\be
\Gamma=\sum_{f} \Gamma_{i \rightarrow f} \rightarrow \int d E D(E) \Gamma(E)
\ee
where the ``sum to integral'' step can be done when levels are densely packed.
$D(E)$ is the density of states (see below).

Using the perturbative result of Eq.~\eqref{eq:g58} in this,
one obtains, for the on-resonance situation,
and already substituting $f(\omega-\omega_0,t) = \frac{2\pi}{t} \delta(\omega-\omega_0)$
\[
\begin{aligned}
\Gamma 
&=\int d E D(E) \frac{\lambda^{2}}{\hbar^{2}} \left|A_{f i}\right|^{2} t \times \frac{2\pi}{t}
\delta\left(\omega-E / \hbar+E_{i} / \hbar\right)\\
&= \frac{\lambda^{2}}{\hbar^{2}} \left|A_{f i}\right|^{2} 2\pi \hbar \, D(\underbrace{E_i + \hbar \omega}_{E_f})
\end{aligned}
\]
so finally
\be
\Gamma  = \frac{2\pi}{\hbar} \lambda^2 \left|A_{f i}\right|^{2} D(E_f)
\ee
This is Fermi's Golden Rule. 

%%% 24 OKAY

\subsubsection{Density of states}

(Here we consider noninteracting particles only.)

System in a box of side $L$, under periodic
boundary conditions (p.b.c).
Consider first \emph{one-dimension}:
\be
-\frac{\hbar^{2}}{2 m} \nabla^{2} \psi=E \psi \therefore\left(\frac{d^{2}}{d x^{2}}+k^{2}\right) \psi=0, k^{2}=\frac{2 m E}{\hbar^{2}}
\ee
$\rightarrow$ $\psi(x)=e^{i k x} \quad \psi(x)=\psi(x+L) \Rightarrow e^{i k L}=1$
%
\be
\cos k L=1 \rightarrow k L=2 n \pi \rightarrow k_{n}=\frac{2 n \pi}{L}, n=0, \pm 1 \pm 2, \ldots
\ee
%
\be
E \rightarrow E(n)=\frac{\hbar^{2}}{2 m}\left(\frac{2 \pi}{L}\right) n^{2}
\ee

There are situations (like in the previous discussion)
where one needs to sum over levels. When levels
are close-packed (or in the continuum), sum is replaced
by an integral $\Rightarrow$ this is achieved by taking $L \to \infty$.
\begin{itemize}
\item note that for two consecutive levels
\be
\Delta n=(n+1)-n=1 \Rightarrow \Delta n=\frac{L}{2 \pi} \Delta k_{n}
\ee
%%% 25 OKAY (SKIPPED STEPS)
\item therefore, one can write
\be
\sum_{n} f\left(k_{n}\right)=\sum_{n} f\left(k_{n}\right) \Delta n=\frac{L}{2 \pi} \sum_{n} \Delta k_{n} f\left(k_{n}\right)
\ee
%
\item for $L \to \infty$, $\Delta k_{n}=\frac{2 \pi}{L} \rightarrow dk$ and
$\sum_{n} \rightarrow \int_{-\infty}^{+\infty} d k$ 
\be
\sum_{n} f\left(k_{n}\right) \rightarrow \int_{-\infty}^{+\infty} \underbrace{d k \frac{L}{2 \pi}}%
_{D(k) dk} 
f(k)
\ee
where we identify the $D(k)$ the density of states.
\end{itemize}

For three-dimensions:
\[
\left(\nabla^{2}+k^{2}\right) \psi(x, y, z)=0 \Rightarrow\left(\frac{\partial^{2}}{\partial x^{2}}+\frac{\partial^{2}}{\partial y^{2}}+\frac{\partial^{2}}{\partial y^{2}}+k^{2}\right) \psi=0
\]

(skipped steps)

%%% 26 OKAY (SKIPPED STEPS)
\setcounter{equation}{67}
\be
\psi(\vec{r}) = e^{i(k_x x + k_y y + k_z z)} = e^{i \vec{k} \cdot \vec{r}}
\ee
Imposing the p.b.c:
\[
(k_x)_{n_x} = \frac{2 \pi}{L} n_{x}, n_{x}=0, \pm 1, \ldots
\]
and same for $(k_y)_{n_y}$ and $(k_z)_{n_z}$.
The energy, $E=\frac{\hbar^{2} k^{2}}{2 m}$, is then given by
\be
E \rightarrow E\left(n_{x}, n_{y}, n_{z}\right)=\frac{\hbar^{2}}{2 m}\left(\frac{2 \pi}{L}\right)^{2}\left(n_{x}^{2}+n_{y}^{2}+n_{z}^{2}\right)
\ee
%%% 27 OKAY
Sum over levels:
\be
\sum_{n_{x} n_{y} n_{z}} f\left(\left(k_{x}\right)_{n_{x}},\left(k_{y}\right)_{n_{y}},\left(k_{z}\right)_{n_{z}}\right)
\xrightarrow{L \to \infty}
\int d k_{x} d k_{y} d k_{z}\left(\frac{L}{2 \pi}\right)^{3} f\left(k_{x}, k_{y}, k_{z}\right)
\ee
and using that $L^3 = V$
\be
\begin{gathered}
\int \underbrace{d^{3} k\left(\frac{L}{2 \pi}\right)^{3}}%
_{D(\vec{k}) d^3k} f(\vec{k})=\int d^{3} k \frac{V}{(2 \pi)^{3}} f(\vec{k})\\
D(\vec{k}) = \frac{V}{8\pi^3}
\end{gathered}
\ee

\emph{Momentum} density of states:
\[
\vec{p} = \hbar \vec{k} \therefore 
\int d^{3} k \frac{V}{(2 \pi)^{3}} f(\vec{k}) = 
\int d^{3} p \underbrace{\frac{1}{\hbar^3}\frac{V}{(2 \pi)^{3}}}_{D(\vec{p})} f(\vec{p})
\]
therefore
\be
D(\vec{p}) = \frac{V}{(2 \pi \hbar)^{3}}=\frac{V}{h^{3}}
\label{eq:g72}
\ee

\emph{Energy} density of states: useful when $f(\vec{p}) = f(p)$
\[
\int d^{3} p \frac{V}{(2 \pi \hbar)^{3}} f(p)=\int d p \underbrace{4 \pi p^{2} \frac{V}{(2 \pi \hbar)^{3}}}%
_{D(p)} f(p)
\]
therefore
\be
D(p)=\frac{4 \pi V}{8 \pi^{3} \hbar^{3}} p^{2}=\frac{V}{2 \pi^{2} \hbar^{3}} p^{2}
\ee
%%% 28 OKAY
Using the fact that 
\[
E=\frac{\hbar^{2}}{2 m} k^{2}=\frac{p^{2}}{2 m} \Rightarrow
d E=\frac{p}{m} d p = \frac{\sqrt{2 m E}}{m} d p
\]
we have
\be
\int d p \frac{V}{2 \pi^{2} \hbar^{3}} p^{2} f\left(p^{2}\right)=\int d E 
\underbrace{\frac{d p}{d E} \frac{V}{2 \pi^{2} \hbar^{3}}}_{D(E)} p^{2} f(E)
\ee
with
\[
D(E)=\frac{V}{2 \pi^{2} \hbar^{3}} p^{2} \frac{d p}{d E}=\frac{V}{2 \pi^{2} \hbar^{3}} p^{\cancel{2}} \frac{m}{\cancel{p}}
\]
but
\[
\frac{V}{2 \pi^{2} \hbar^{3}} m p=\frac{V m}{2 \pi^{2} \hbar^{3}}(2 m E)^{1 / 2}
\]
therefore
\be
D(E)=\frac{V m}{2 \pi^{2} \hbar^{3}}(2 m E)^{1 / 2}
\ee
where $D(E)$ is the number of energy levels between $E$ and $E+dE$.

Note that 
\be
V = \int_V d^3r \to d^3r d^3p
\ee
which can be put back in Eq.~\eqref{eq:g72}. So the
number of levels in phase space $(\vec{r},\vec{p})$:
\be
d N=\frac{d^{3} r d^{3} p}{(2 \pi \hbar)^{3}}=\frac{d^{3} r d^{3} p}{h^{3}}
\ee
where $h^3$ is the volume of an elementary cell in phase space.
%%% 29
$\frac{1}{h^3}$: one level per cell
$\Rightarrow$
particle localized within $\Delta x \rightarrow$ momentum $\frac{h}{\Delta x}$,
a quantum particle must have a
volume in phase space as large as $(\Delta x)^{3}(\Delta p)^{3} \sim h^{3}$.

\end{document}
















