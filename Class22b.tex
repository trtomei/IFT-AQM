%!TEX TS-program = xelatex
%!TEX encoding = UTF-8 Unicode

\documentclass[12pt]{article}
\usepackage{geometry}                % See geometry.pdf to learn the layout options. There are lots.
\geometry{a4paper,top=2cm}
\usepackage[parfill]{parskip}    % Activate to begin paragraphs with an empty line rather than an indent
\usepackage{graphicx}
\usepackage{amsmath}
\usepackage{amssymb}
\usepackage{mathtools}
\usepackage{physics}
\newcommand{\be}{\begin{equation}}
\newcommand{\ee}{\end{equation}}
\usepackage[thicklines]{cancel}
\usepackage[colorlinks=true,citecolor=blue,linkcolor=blue,urlcolor=blue]{hyperref}
\usepackage{booktabs}
\usepackage{csquotes}
\usepackage{qcircuit}
\usepackage{circledsteps}
\usepackage{nicefrac}
\usepackage{fontspec,xltxtra,xunicode}
\usepackage{xcolor}
\usepackage{simplewick}
\defaultfontfeatures{Mapping=tex-text}

\newcommand{\polv}{\ensuremath{\updownarrow}}
\newcommand{\polh}{\ensuremath{\leftrightarrow}}
\newcommand{\poldr}{\rotatebox[origin=c]{45}{\ensuremath{\leftrightarrow}}}
\newcommand{\poldl}{\rotatebox[origin=c]{-45}{\ensuremath{\leftrightarrow}}}
\newcommand{\bigzero}{\mbox{\normalfont\Large\bfseries 0}}
\newcommand{\vecrp}{\ensuremath{\vec{r}^{\,\prime}}}
\newcommand{\vecnr}{\ensuremath{\vec{\nabla}_{\!r}}}

\title{Advanced Quantum Mechanics\\Class 22 (b)}
%\author{The Author}
\date{May 16, 2023}                                           % Activate to display a given date or no date

\setcounter{section}{4}
\setcounter{subsection}{12}
\setcounter{equation}{115}

\begin{document}
\maketitle

%%% 12 OKAY
\subsection{Time reversal (continued)}

\par\noindent\rule{\textwidth}{1pt}
\emph{Reminder from last class}

Sometimes one needs to relate matrix elements of
an operator between time-reversed states, \textit{e.g.} in
scattering processes, where one often needs to relate
an $i \rightarrow f$ process to its time-reversed motion $f \rightarrow i$.
Of course, this happens because we only have access to the final state process $f$.

Let $\hat{A}$ be an arbitrary linear operator in $H$ and
$|\varphi\rangle$ an arbitrary vector in $H$. We remember that
we have:
\be
\hat{\Theta} \hat{A} \ket{\varphi}=
\begin{cases}
\begin{aligned}
\hat{\Theta}\!\left(\sum_n\op{n}\hat{A}\ket{\varphi}\right) 
&= \sum_n \op{n_t}{n}\hat{A}\ket{\varphi}^*
= \sum_n \op{n_t}{\varphi}\hat{A}^{\dagger}\ket{n}\\
\text{also: }\hat{\Theta}\hat{A}\hat{\Theta}^{-1}\hat{\Theta}\ket{\varphi} 
&= \sum_n \op{n_t}{n}\hat{A}\hat{\Theta}^{-1}\ket{\varphi_t}^*
\end{aligned}
\end{cases}
\ee
Let $\ket{\psi}$ be another arbitrary state in $H$. We write 
the time-reversed operator $\hat{A}_t = \hat{\Theta}\hat{A}\hat{\Theta}^{-1}$
and the time-reversed states
$\hat{\Theta}\ket{\psi} = \ket{\psi_t}$.
Then:
\be
\begin{gathered}
\bra{\psi_t}\hat{A}_t\ket{\varphi_t} =
\bra{\psi_t}\hat{\Theta}\hat{A}\hat{\Theta}^{-1}\ket{\varphi_t}
 = \bra{\psi_t}\hat{\Theta}\left(
\sum_n\op{n}\hat{A}\underbrace{\hat{\Theta}^{-1}\ket{\varphi_t}}_{\ket{\varphi}}
\right)\\
= \bra{\psi_t} \hat{\Theta} \left( \sum_n \ket{n} \bra{n}\hat{A}\ket{\varphi} \right), 
%
\hat{\Theta} \text{ acts on $\ket{n}$ and conjugates everything},\\
%
=\sum_n \bra{\psi_t}\ket{n_t} \bra{\varphi}\hat{A}^{\dagger}\ket{n}
=\sum_n \bra{\hat{\Theta}\psi}\ket{\hat{\Theta}n} \bra{\varphi}\hat{A}^{\dagger}\ket{n}
=\sum_n \bra{n}\ket{\psi} \bra{\varphi}\hat{A}^{\dagger}\ket{n}\\
=\sum_n \bra{\varphi}\hat{A}^{\dagger} \ket{n}\!\!\bra{n}\ket{\psi}
= \bra{\varphi}\hat{A}^{\dagger}\ket{\psi}
\end{gathered}
\label{eq:g117}
\ee
When $\hat{A}^{\dagger} = \hat{A}$ and time-reversal invariant:
\be
\bra{\psi_t}\hat{\Theta}\hat{A}\hat{\Theta}^{-1}\ket{\varphi_t} = 
\bra{\psi_t}\hat{A}\ket{\varphi_t} =
\bra{\varphi}\hat{A}\ket{\psi}
\ee
so the matrix element transition $\psi \to \varphi$, mediated by $\hat{A}$,
is the same as that obtained by exchanging final and initial state, 
but between the time-reversed states. So:
\[
(\varphi_t \to \psi_t) \leftarrow \text{ time reversal }\rightarrow (\psi \to \varphi).
\]

\par\noindent\rule{\textwidth}{1pt}

%%% 13 OKAY

\emph{Spinless particles}

Representation of the time reversal for spinless particles:
$\vec{S}=0 \Rightarrow \vec{J}=\vec{L}$.
We assume a Hamiltonian $\hat{H}$ invariant under time-reversal.

\emph{Coordinate representation}

\[
\begin{aligned}
\varphi(\vec{r}, t)=    \langle\vec{r} | \varphi(t)\rangle, 
|\varphi(t)\rangle    =e^{-i / \hbar \hat{H} t} | \varphi\rangle,
&\text{ where } |\varphi\rangle = |\varphi(0)\rangle\\
\varphi_{t}(\vec{r}, t)=\langle\vec{r} | \varphi_{t}(t)\rangle, 
|\varphi_{t}(t)\rangle=e^{-i / \hbar \hat{H} t} |\varphi_{t}\rangle,
&\text{ where } |\varphi_t\rangle = |\hat{\Theta}\varphi(0)\rangle\\ 
\end{aligned}
\]

Since $\vec{r}$ is time-reversal invariant, Eq.~\eqref{eq:g117} %%% FIXME
implies
\be
\begin{aligned}
\varphi_{t}(\vec{r}, t)
&=\left\langle\vec{r} \,| \varphi_{t}(t)\right\rangle=\left\langle\vec{r}_{t} | \varphi_{t}(t)\right\rangle\\
&=\left\langle\vec{r}_{t}\left|e^{-i / \hbar \hat{H} t}\right| \varphi_{t}\right\rangle\\
&=\left\langle\vec{r}_{t}\left|\left(\hat{\Theta} e^{i / \hbar \hat{H} t} \hat{\Theta}^{-1}\right)\right| \varphi_{t}\right\rangle\\
&=\left\langle\varphi\left|e^{-i / \hbar H t}\right| \vec{r}\right\rangle=\left\langle e^{i / \hbar \hat{H} t} \varphi | \vec{r}\right\rangle\\
&=\left\langle e^{-i / \hbar \hat{H}(-t)} \varphi | \vec{r}\right\rangle=\langle\varphi(-t) | \vec{r}\rangle\\
&=\varphi^{*}(\vec{r},-t)
\end{aligned}
\ee
For spin 0, time reversal is accomplished by
complex conjugation and $t \to -t$:
$\hat{\Theta} = \hat{\Theta}_0 + t\to-t$. 
%%% 14 OKAY

Spherical symmetry: $\varphi(\vec{r}) = R(r) Y^m_l(\theta,\phi)$,
where the spherical harmonic factor is, in general, complex.
Under $\hat{\Theta}$:
\be
Y_{l}^{m}(\theta, \phi) \rightarrow Y_{l}^{m *}(\theta, \phi)=(-1)^{m} Y_{l}^{- m}(\theta, \phi)
\ee

\emph{Momentum representation}

Momentum eigenstates: $\ket{\vec{p}}$
\be
\begin{aligned}
\hat{\Theta}\ket{\vec{p}} = \ket{\hat{\Theta}\vec{p}} 
&= \ket{-\vec{p}} = \ket{\vec{p}_t}\\
\hat{\Theta}\ket{-\vec{p}} = \ket{\hat{\Theta}(-\vec{p})} 
&= \ket*{-\underbrace{\hat{\Theta}\vec{p}}%
_{\vec{p}_t}
} 
= \ket{\vec{p}} \equiv \ket{-\vec{p}_t}
\end{aligned}
\ee
Therefore the momentum space wave function
\be
\widetilde{\varphi}(\vec{p}, t)=\langle\vec{p} | \varphi(t)\rangle=\left\langle\vec{p}\left|e^{-i / \hbar \hat{H} t}\right| \varphi\right\rangle
\ee
transforms under time reversal as:
\be
\begin{aligned}
\widetilde\varphi_{t}(\vec{p}, t)
&=\left\langle\vec{p} \,| \varphi_{t}(t)\right\rangle=\left\langle \vec{p}\left|e^{-i / \hbar \hat{H} t}\right| \varphi_{t}\right\rangle\quad\rightarrow(\varphi_{t} = \hat{\Theta} \varphi(0))\\
&=\left\langle-\vec{p}_{t}\left|e^{-i / \hbar\hat{H} t}\right| \varphi_{t}\right\rangle\\
&=\left\langle-\vec{p}_{t}\right|\left(\hat{\Theta} e^{i / \hbar \hat{H} t} \hat{\Theta}^{-1}\right)
\left|\varphi_{t}\right\rangle
\quad\text{ , using Eq.~\eqref{eq:g117}}\\
&=\left\langle\varphi\left|e^{-i/ \hbar \hat{H} t}\right|-\vec{p}\right\rangle
=\widetilde{\varphi}^{*}(-\vec{p},-t)
\end{aligned}
\ee

%%% 15 OKAY

\emph{Spin 1-2}

Since $\hat{\Theta}\vec{J}\hat{\Theta}^{-1} = -\vec{J}$, one has for $j=1/2$:
\be
\hat{\Theta} \vec{\sigma} \hat{\Theta}^{-1} = -\vec{\sigma}
\label{eq:g124}
\ee
Remembering that  
$\hat{\Theta} = \hat{U}_t\hat{\Theta}_0$, Eq.~\eqref{eq:g124} implies
\be
\hat{U}_{t} \hat{\Theta}_{0} \vec{\sigma} \hat{\Theta}_{0} \hat{U}_{t}^{\dagger}=
\hat{U}_{t} \,\vec{\sigma}^{*}\, \hat{U}_{t}^{-1}=-\vec{\sigma}
\ee
which means the three conditions
\[
\begin{aligned}
\hat{U}_{t} \sigma_{x} \hat{U}_{t}^{-1} &=-\sigma_{x} \\ 
\hat{U}_{t} \sigma_{y} U_{t}^{-1}       &=+\sigma_{y} \\ 
\hat{U}_{t} \sigma_{z} \hat{U}_{t}^{-1} &=-\sigma_{z}
\end{aligned}
\]
Satisfied by choosing, for instance
\be
\hat{U}_{t}=e^{i \delta} e^{i \frac{\pi}{2} \sigma_{y}}\,\sim
\text{rotation of $-\pi/2$ around $\hat{y}$}
\ee
with $\delta$ an arbitrary phase, but this choice \emph{is not} unique.
Notice: $e^{i \frac{\pi}{2} \sigma_{y}} = i\sigma_y$.

Therefore, when acting on spin 1-2 states and operators:
\be
\hat{\Theta} = e^{i \delta} %e^{i \frac{\pi}{2} \sigma_{y}}
i\sigma_{y}
\hat{\Theta}_0
\ee
With this choice, action of $\hat{\Theta}$ on $\ket{1/2,\pm1/2}$ in the
conventional basis gives:
\be
\begin{gathered}
\hat{\Theta} \ket{1/2,1/2} = \hat{\Theta}\begin{pmatrix}1\\0\end{pmatrix}
= e^{i \delta} i \sigma_{y} \begin{pmatrix}1\\0\end{pmatrix} = -e^{i \delta} \begin{pmatrix}0\\1\end{pmatrix} = -e^{i \delta} \ket{1/2,-1/2}\\
\hat{\Theta} \ket{1/2,-1/2} = + e^{i \delta} \ket{1/2,1/2}
\end{gathered}
\ee
$\Rightarrow$ notice a possible phase convention here; so
%%% 16 OKAY
\be
\hat{\Theta} \ket{1/2,m} = e^{i \delta} (-1)^{1/2+m} \ket{1/2,-m}
\ee

Consider now the action on $\ket{\vec{p}} \otimes \ket{m} = \ket{\vec{p}m}$ and
$\ket{\vec{r}} \otimes \ket{m} = \ket{\vec{r}m}$:
\be
\begin{aligned}
\hat{\Theta} \ket{\vec{p}m} &= e^{i \delta} (-1)^{1/2+m} \ket{-\vec{p}-m}\\
\hat{\Theta} \ket{\vec{r}m} &= e^{i \delta} (-1)^{1/2+m} \ket{ \vec{r}-m}
\end{aligned}
\ee
$\Rightarrow$ square of $\hat{\Theta}$ in coordinate/momentum representation.

\emph{Spin 0:} (complex conjugation)
\be
\hat{\Theta}^2 = \hat{\Theta}(\hat{\Theta}_0+t\to-t) = 1
\ee

\emph{Spin 1/2:}
\be
\begin{gathered}
\hat{\Theta}^2 =  \hat{\Theta}(e^{i \delta} i \sigma_y \hat{\Theta}_0)
= e^{i \delta} i \sigma_y \hat{\Theta}_0 (e^{i \delta} i \sigma_y \hat{\Theta}_0)\\
= e^{i \delta} (i \sigma_y) e^{-i \delta} (i \sigma_y) = -\sigma_y^2 = -1
\end{gathered}
\ee
Note that $e^{i \delta}$ drops out in $\hat{\Theta}^2$ $\to$ actually $\hat{\Theta}^2$ has no dependence
on the representation.

Consider now a system of $N$ particles:
\begin{itemize}
\item if the particles have spin 0 $\Rightarrow$ $\hat{\Theta}^2 = 1$ for any $N$ $\to$ 
nothing interesting happens.
%%% 17 OKAY
\item if the particles have spin 1/2: (neglecting the phase)
\[
\begin{gathered}
\hat{\Theta} =
(i \sigma_y \hat{\Theta}_0))_1 \otimes
(i \sigma_y \hat{\Theta}_0))_2 \otimes
\ldots \otimes
(i \sigma_y \hat{\Theta}_0))_N\\
\hat{\Theta}^2 =
[(i \sigma_y \hat{\Theta}_0))_1]^2
\ldots
[(i \sigma_y \hat{\Theta}_0))_N]^2
\end{gathered}
\]
and finally
\be
\boxed{\hat{\Theta}^2 = (-1)^N}
\ee
\end{itemize}
Consequence of this result:
\begin{itemize}
\item let $\{\ket{E,N}\}$ be the stationary states of a
system of $N$ spin 1-2 particles
%
\item assume
\be
[\hat{\Theta},\hat{H}] = 0
\ee
$\Rightarrow$ $\hat{\Theta}\ket{E,N}$ is also an eigenstate of $\hat{H}$
%
\item \emph{question:} is $\hat{\Theta}\ket{E,N}$ linearly independent of
the state $\ket{E,N}$? $\Rightarrow$
If it is linearly independent, then
there is a degeneracy in the spectrum
of the system.
%%% 18 OKAY
\item so, if $\hat{\Theta}\ket{E,N}$ \emph{is not} degenerate
\be
\hat{\Theta}\ket{E,N} = \lambda\ket{E,N}
\ee
therefore
\be
\begin{aligned}
\underbrace{\hat{\Theta}^2}_{(-1)^N}\ket{E,N} = &\lambda^*\lambda \ket{E,N} &= (-1)^N\ket{E,N}\\
\Rightarrow & \underbrace{|\lambda|^2}_{\text{positive}} &= \underbrace{(-1)^N}_{\text{negative if $N$ odd}}
\end{aligned}
\ee
\item therefore, if $N$ is odd $\Rightarrow$ $\hat{\Theta}\ket{E,N}$ must be degenerate.
\end{itemize}

\emph{Kramer's theorem:} If the Hamiltonian of a system composed
by \emph{an odd number} of spin-1/2 particles
is time-reversal invariant, then all of its
stationary states are degenerate.
\emph{Exercise:} show that the degree of degeneracy is even.

When a system with an odd number of spin-1/2
particles is exposed to magnetic field (breaks
time-reversal invariance), the energy levels will split
into multiplets with an even number of members.

\end{document}

\hat{\Theta}



















