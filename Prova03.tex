%!TEX TS-program = xelatex
%!TEX encoding = UTF-8 Unicode

\documentclass[12pt]{article}
\usepackage{geometry}                % See geometry.pdf to learn the layout options. There are lots.
\geometry{a4paper,top=2cm}
\usepackage[parfill]{parskip}    % Activate to begin paragraphs with an empty line rather than an indent
\usepackage{graphicx}
\usepackage{amsmath}
\usepackage{amssymb}
\usepackage{physics}
\usepackage{polyglossia}
\setdefaultlanguage{portuguese}

\usepackage{fontspec,xltxtra,xunicode}
\defaultfontfeatures{Mapping=tex-text}

\title{Mecânica Quântica Avançada\\%
Prova 03\\%
01 de dezembro -- 13 de dezembro}
%\author{The Author}
\date{}

\begin{document}
\maketitle
\vspace*{-4em}

\textbf{Exercício 1:} 

Considere o problema de duas partículas idênticas descritas pelo Hamiltoniano
\[
\hat{H}=\hat{H}_{0}(1)+\hat{H}_{0}(2)+\hat{H}_{I}(1,2)
\]
onde os números em parênteses identificam as partículas.
Sejam $\varphi_1(\vec{r})$ e  $\varphi_2(\vec{r})$ as partes espaciais dos autoestados de $\hat{H}_{0}$
(supostos independentes de spin):
\[
H_{0}(\vec{r}) \varphi_{1}(\vec{r})=E_{1}^{(0)} \varphi_{1}(\vec{r}), \quad H_{0}(r) \varphi_{2}(\vec{r})=E_{2}^{(0)} \varphi_{2}(\vec{r})
\]
onde $E_{1}^{(0)}$ e $E_{2}^{(0)}$ são as energias dos estados fundamental e primeiro estado excitado de
$\hat{H}_{0}$, respectivamente.

a) Suponha primeiro que $\hat{H}_{I}(1,2) = 0$. 
Encontre a energia do estado fundamental do sistema supondo que as partículas são bósons.
Repita o problema para férmions de spin-1/2, para os estados de singleto e tripleto.

b) Suponha agora que  $\hat{H}_{I}(1,2) =\hat{H}_{I}\left(\left|\vec{r}_{1}-\vec{r}_{2}\right|\right) \neq 0$, mas independente de spin.
Encontre, em primeira ordem em $\hat{H}_{I}(1,2)$, a energia do estado fundamental, supondo que as partículas são bósons.
Repita o problema para férmions de spin-1/2, para os estados de singleto e tripleto.

c) Repita o item b), para o caso de férmions de spin-1/2, supondo
\[
\hat{H}_{0}=\frac{\vec{p}^{\,2}}{2 m}+\frac{1}{2} m \omega^{\,2} \vec{r}^{\,2}
\]
e $\hat{H}_{I}(1,2)=g / \hbar^{2} \hat{\vec{S}}_{1} \cdot \hat{\vec{S}}_{2}$,
onde $\hat{\vec{S}}$ é o operador de spin e $g$ é uma constante.

\textbf{Exercício 2:} 

Considere os operadores do espaço de Fock:
\[
\hat{\rho}(\vec{r})=\int d^{3} r^{\prime} \hat{\psi}^{\dagger}\left(\vec{r}^{\,\prime}\right) \delta\left(\vec{r}^{\,\prime}-\vec{r}\right) \hat{\psi}\left(\vec{r}^{\,\prime}\right)
\]
\[
\hat{\vec{L}}(\vec{r})=\frac{\hbar}{2 m i} \int d^{3} r^{\prime} \hat{\psi}^{\dagger}\left(\vec{r}^{\,\prime}\right)\left[\vec{\nabla}_{r^{\prime}} \delta\left(\vec{r}^{\,\prime}-\vec{r}\right)+\delta\left(\vec{r}^{\,\prime}-\vec{r}\right) \vec{\nabla}_{r^{\prime}}\right] \hat{\psi}\left(\vec{r}^{\,\prime}\right)
\]
Calcule os valores esperados desses operadores no estado de uma partícula (bóson ou férmion).
\[
|\kappa\rangle=\hat{a}_{\kappa}^{\dagger}|0\rangle, \quad \hat{a}_{\kappa}^{\dagger}=\int d^{3} r \varphi_{\kappa}(\vec{r}) \hat{\psi}^{\dagger}(\vec{r})
\]

\textbf{Exercício 3:} 

Considere a representação no espaço de Fock do operator momento angular total
(lembre-se que nós usamos ``chapéus'' para operadores no espaço de Fock, operadores de primeira quantização perdem seus chapéus)
\[
\hat{\vec{J}}=\int d^{3} r \hat{\psi}_{\kappa j m}^{\dagger}(\vec{r})\left\langle\kappa j m|\hat{\vec{J}}| \kappa^{\prime} j m^{\prime}\right\rangle \hat{\psi}_{\kappa^{\prime} j m^{\prime}}(\vec{r})
\]
onde estamos usando a convenção de que a soma sobre índices repetidos é implícita. Mostre que
\[
\left[\hat{J}_{i}, \hat{J}_{j}\right]=i \hbar \epsilon_{i j k} \hat{J}_{k}, \quad i, j, k=1,2,3
\]
é válido para bósons e férmions.

\textbf{Exercício 4:}

Mostre que o Hamiltoniano
\begin{gather}
\hat{H}=\sum_{m} \int d^{3} r \hat{\psi}_{m}^{\dagger}(\vec{r})\left(-\frac{h^{\,2}}{2 M} \nabla^{\,2}\right) \hat{\psi}_{m}(\vec{r})\\
+\frac{1}{2} \sum_{m_{1}, m_{1}^{\prime}} \sum_{m_{2}, m_{2}^{\prime}} \int d^{3} r d^{3} r^{\prime} \hat{\psi}_{m_{1}^{\prime}}^{\dagger}(\vec{r}) \hat{\psi}_{m_{2}^{\prime}}^{\dagger}\left(\vec{r}^{\,\prime}\right)\left[V_{2}\left(\left|\vec{r}-\vec{r}^{\,\prime}\right|, s\right)\right]_{m_{1}^{\prime} m_{2}^{\prime} ; m_{1} m_{2}} \hat{\psi}_{m_{2}}\left(\vec{r}^{\,\prime}\right) \hat{\psi}_{m_{1}}(\vec{r})
\end{gather}
conserva o número de partículas.

\textbf{Exercício 5:}

Considere o hamiltoniano:
%
\[
\hat{H} = \sum _ { \alpha } T _ { \alpha } \hat{a} _ { \alpha } ^ { \dagger } \hat{a} _ { \alpha } + \sum _ { \alpha , \beta , \gamma } V _ { \alpha \beta \gamma } \left( \hat{a} _ { \alpha } ^ { \dagger } \hat{a} _ { \beta } \hat{a} _ { \gamma } + \hat{a} _ { \alpha } ^ { \dagger } \hat{a} _ { \beta } ^ { \dagger } \hat{a} _ { \gamma } \right)
\]
onde $T^*_\alpha = T_\alpha$, $V^*_{\alpha\beta\gamma} = V_{\alpha\beta\gamma}$ é simétrico em todos os índices, e os operadores $a_\alpha$ e $a_\alpha^\dagger$ satisfazem regras de comutação bosônicas.

a) Este hamiltoniano é hermiteano? Prove sua resposta.

b) Este hamiltoniano conserva o número de partículas $N = \sum_\alpha a_\alpha^\dagger  a_\alpha$?
Prove sua resposta.

\end{document}