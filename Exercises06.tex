%!TEX TS-program = xelatex
%!TEX encoding = UTF-8 Unicode

\documentclass[12pt]{article}
\usepackage{geometry}                % See geometry.pdf to learn the layout options. There are lots.
\geometry{a4paper,top=1.8cm,bottom=2cm,right=2.7cm}
\usepackage[parfill]{parskip}    % Activate to begin paragraphs with an empty line rather than an indent
\usepackage{graphicx}
\usepackage{amsmath}
\usepackage{amssymb}
\usepackage{mathtools}
\usepackage{physics}
\newcommand{\be}{\begin{equation}}
\newcommand{\ee}{\end{equation}}
\usepackage[thicklines]{cancel}
\usepackage{url}
\usepackage{booktabs}

\usepackage{fontspec,xltxtra,xunicode}
\defaultfontfeatures{Mapping=tex-text}

\newcommand{\polv}{\ensuremath{\updownarrow}}
\newcommand{\polh}{\ensuremath{\leftrightarrow}}
\newcommand{\poldr}{\rotatebox[origin=c]{45}{\ensuremath{\leftrightarrow}}}
\newcommand{\poldl}{\rotatebox[origin=c]{-45}{\ensuremath{\leftrightarrow}}}

\title{Advanced Quantum Mechanics\\Exercises Set 6\vspace{-1em}}
%\author{The Author}
\date{Due date: June 29\textsuperscript{th}, 2023}

\begin{document}
\maketitle

\textbf{Textbook exercises:}

\textbf{Exercises 1, 2:} 14.6.2 and 14.6.3 of M.~Le~Bellac.

\textbf{Exercises 3, 4, 5, 6:} Problems 5.22, 5.23, 5.24, 5.26 of Sakurai \& Napolitano's \emph{Modern Quantum Mechanics (Second Edition)} -- problem numbers are the same as in Sakurai's \emph{Revised Edition}.

\textbf{Exercise 7:} The Hamiltonian for two interacting spins
(both $s=1 / 2$), $\sigma^{(1)}$ and $\sigma^{(2)}$, in a magnetic field $B$ directed along the $z$-axis is
\[
\hat{H}=B\left(a_{1} \sigma_{z}^{(1)}+a_{2} \sigma_{z}^{(2)}\right)+K \sigma^{(1)} \cdot \sigma^{(2)} \text {, }
\]
where $a_{1}$ and $a_{2}$ are the negatives of the magnetic moments (assumed to be unequal to avoid
degeneracy), and $K$ is the interaction strength.

\textbf{a)} Use second order perturbation theory to calculate the energy eigenvalues, assuming that
$B$ is small.

\textbf{b)} Use second order perturbation theory to calculate the energy eigenvalues, under the opposite assumption that $K$ is small.

\textbf{c)} Find the exact energy eigenvalues for this Hamiltonian, and verify the correctness of your
answers in parts a) and b).

\textbf{Exercise 8:} The Hamiltonian matrix
for a two-level system can be written as
\[
H=\begin{pmatrix}E_{1}^{0} & \lambda W \\ \lambda W & E_{2}^{0}\end{pmatrix},
\quad \lambda W\text{ real}
\]
and the energy eigenvectors for the unperturbed ($\lambda=0$) system are given by
\[
\phi_{1}^{0}=\begin{pmatrix}1 \\ 0\end{pmatrix}, \quad \phi_{2}^{0}=\begin{pmatrix}0 \\ 1\end{pmatrix}.
\]

\textbf{a)} Solve this problem exactly to find the energy eigenfunctions $\psi_{1}$ and $\psi_{2}$ and the energy
eigenvalues $E_{1}$ and $E_{2}$.

\textbf{b)} Assuming that $\lambda W \ll\left|E_{1}^{0}-E_{2}^{0}\right|$, solve the same problem using time-independent perturbation theory up to $\mathcal{O}(\lambda)$ for the eigenvectors and up to $\mathcal{O}\left(\lambda^{2}\right)$ for the eigenvalues. Compare
with the exact solutions obtained in a).

\textbf{c)} Suppose the unperturbed energies are ``almost degenerate''; that is, $\left|E_{1}^{0}-E_{2}^{0}\right| \ll \lambda W$. Show
that the exact results obtained in a) closely resemble what you would expect by applying
degenerate perturbation theory with the value of $E_{1}^{0}$ set exactly equal to $E_{2}^{0}$.

%%% Exercises of Many-Particle Systems

\textbf{Exercise 9:} 

Consider the Fock space operators:
\[
\hat{\rho}(\vec{r})=\int d^{3} r^{\prime} \hat{\psi}^{\dagger}\left(\vec{r}^{\,\prime}\right) \delta\left(\vec{r}^{\,\prime}-\vec{r}\right) \hat{\psi}\left(\vec{r}^{\,\prime}\right)
\]
\[
\hat{\vec{L}}(\vec{r})=\frac{\hbar}{2 m i} \int d^{3} r^{\prime} \hat{\psi}^{\dagger}\left(\vec{r}^{\,\prime}\right)\left[\vec{\nabla}_{r^{\prime}} \delta\left(\vec{r}^{\,\prime}-\vec{r}\right)+\delta\left(\vec{r}^{\,\prime}-\vec{r}\right) \vec{\nabla}_{r^{\prime}}\right] \hat{\psi}\left(\vec{r}^{\,\prime}\right)
\]
Calculate the expected values of those operators in the one-state particle (boson or fermion).
\[
|\kappa\rangle=\hat{a}_{\kappa}^{\dagger}|0\rangle, \quad \hat{a}_{\kappa}^{\dagger}=\int d^{3} r \varphi_{\kappa}(\vec{r}) \hat{\psi}^{\dagger}(\vec{r})
\]

\textbf{Exercise 10:} 

Consider the Fock space representation of the total angular momentum
(remember that we use ``hats'' for Fock space operators, while first-quantization operators lose their hats)
\[
\hat{\vec{J}}=\int d^{3} r \hat{\psi}_{\kappa j m}^{\dagger}(\vec{r})\left\langle\kappa j m|\hat{\vec{J}}| \kappa^{\prime} j m^{\prime}\right\rangle \hat{\psi}_{\kappa^{\prime} j m^{\prime}}(\vec{r})
\]
where we are using the convention that there is an implicit sum over repeated indices. Show that
\[
\left[\hat{J}_{i}, \hat{J}_{j}\right]=i \hbar \epsilon_{i j k} \hat{J}_{k}, \quad i, j, k=1,2,3
\]
is valid for bosons and fermions.


\end{document}
