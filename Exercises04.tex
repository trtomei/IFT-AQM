%!TEX TS-program = xelatex
%!TEX encoding = UTF-8 Unicode

\documentclass[12pt]{article}
\usepackage{geometry}                % See geometry.pdf to learn the layout options. There are lots.
\geometry{a4paper,top=1.8cm,bottom=2cm,right=2.7cm}
\usepackage[parfill]{parskip}    % Activate to begin paragraphs with an empty line rather than an indent
\usepackage{graphicx}
\usepackage{amsmath}
\usepackage{amssymb}
\usepackage{mathtools}
\usepackage{physics}
\newcommand{\be}{\begin{equation}}
\newcommand{\ee}{\end{equation}}
\usepackage[thicklines]{cancel}
\usepackage{url}
\usepackage{booktabs}

\usepackage{fontspec,xltxtra,xunicode}
\defaultfontfeatures{Mapping=tex-text}

\newcommand{\polv}{\ensuremath{\updownarrow}}
\newcommand{\polh}{\ensuremath{\leftrightarrow}}
\newcommand{\poldr}{\rotatebox[origin=c]{45}{\ensuremath{\leftrightarrow}}}
\newcommand{\poldl}{\rotatebox[origin=c]{-45}{\ensuremath{\leftrightarrow}}}

\title{Advanced Quantum Mechanics\\Exercises Set 4\vspace{-0.5em}}
%\author{The Author}
\date{Due date: October 27\textsuperscript{th}, 2022}

\begin{document}
\maketitle


\textbf{Exercises 1 and 2:} exercises from Le~Bellac 8.5.2 and 8.5.6.

\textbf{Exercise 3.}
 So far we have considered Galilean boosts by a speed \(v\) (in one spatial dimension)
at \(t=0\) and obtained, by demanding that the expectation values of the position \(\hat{X}\), speed
\(d \hat{X} / d t\) and momentum \(\hat{P}\) transform like their classical counterparts, that the unitary operator
\(\hat{U}(v)\) that implements such boosts in a Hilbert space is given by (employing the active point
of view of the boost)

\[\hat{U}(v)=e^{i m v \hat{X} / \hbar}\]

where \(m\) is the mass of the particle. Show that this generalizes for an arbitrary time \(t\) to

\[\hat{U}(v)=e^{i m v \hat{X} / \hbar} e^{-i t v \hat{P} / \hbar} e^{-i m v^{2} t / 2 \hbar}=e^{-i t v \hat{P} h} e^{i m v \hat{X} / \hbar} e^{i m v^{2} t / 2 \hbar}\]

Show also that the last result generalizes in three dimensions to

\[\hat{U}(v)=e^{i m \vec{v} \cdot \hat{\vec{X}} / h} e^{-i t \vec{v} \cdot \hat{\vec{P}} / h} e^{-i m \vec{v}^{2} t / 2 h}=e^{-i t \vec{v} \cdot \hat{\vec{P}} / h} e^{i m \vec{v} \cdot \hat{\vec{X}} / h} e^{i m \vec{v}^{2} t / 2 h}\]

\textbf{Exercise 4.} The Galilean boosts, \textit{a.k.a.} pure Galilean transformations, form a subgroup of a
larger, 10-dimensional group named Galilei (or Galileo) group of space-time transformations:

\[
\begin{aligned}
\vec{x} \rightarrow \vec{x}^{\,\prime}&=R \vec{x}+\vec{a}+\vec{v} t \\ 
t \rightarrow t^{\prime}&=t+s
\end{aligned}
\]

where in the addition to the displacement \(\vec{a}\) and boost velocity \(\vec{v}\) studied so far, one also has
a spatial rotation \(R\) and time displacement \(s\). Let \(g=(R, \vec{a}, \vec{v}, s)\) denote such a transformation.

Show that the composition law for \(g_{3}=g_{2} g_{1}\), with \(g_{3}=\left(R_{3}, a_{3}, v_{3}, s_{3}\right)\) is:
\[
\begin{aligned} 
R_{3} &=R_{2} R_{1} \\ 
\vec{a}_{3} &=\vec{a}_{2}+R_{2} \vec{a}_{1}+\vec{v}_{2} s_{1} \\ 
\vec{v}_{3} &=\vec{v}_{2}+R_{2} \vec{v}_{1} \\ 
s_{3} &=s_{2}+s_{1} 
\end{aligned}
\]

\textbf{Exercises 5--7:} exercises from Le~Bellac 10.7.1, 10.7.2, and 10.7.4.

\end{document}