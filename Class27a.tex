%!TEX TS-program = xelatex
%!TEX encoding = UTF-8 Unicode

\documentclass[12pt]{article}
\usepackage{geometry}                % See geometry.pdf to learn the layout options. There are lots.
\geometry{a4paper,top=2cm}
\usepackage[parfill]{parskip}    % Activate to begin paragraphs with an empty line rather than an indent
\usepackage{graphicx}
\usepackage{amsmath}
\usepackage{amssymb}
\usepackage{mathtools}
\usepackage{physics}
\newcommand{\be}{\begin{equation}}
\newcommand{\ee}{\end{equation}}
\usepackage[thicklines]{cancel}
\usepackage[colorlinks=true,citecolor=blue,linkcolor=blue,urlcolor=blue]{hyperref}
\usepackage{booktabs}
\usepackage{csquotes}
\usepackage{qcircuit}
\usepackage{circledsteps}
\usepackage{nicefrac}
\usepackage{fontspec,xltxtra,xunicode}
\usepackage{xcolor}
\defaultfontfeatures{Mapping=tex-text}

\newcommand{\polv}{\ensuremath{\updownarrow}}
\newcommand{\polh}{\ensuremath{\leftrightarrow}}
\newcommand{\poldr}{\rotatebox[origin=c]{45}{\ensuremath{\leftrightarrow}}}
\newcommand{\poldl}{\rotatebox[origin=c]{-45}{\ensuremath{\leftrightarrow}}}
\newcommand{\bigzero}{\mbox{\normalfont\Large\bfseries 0}}
\newcommand{\vecrp}{\ensuremath{\vec{r}^{\,\prime}}}
\newcommand{\vecnr}{\ensuremath{\vec{\nabla}_{\!r}}}

\title{Advanced Quantum Mechanics\\Class 27 (a)}
%\author{The Author}
\date{December 01, 2022}                                           % Activate to display a given date or no date

\begin{document}
\maketitle

\setcounter{section}{8}

%%% 01 OKAY

\section{Many-particle systems}
\setcounter{subsection}{2}
Reminder:
\begin{gather*}
\ket{\vec{r}_1\ldots\vec{r}_N} = \hat{\psi}^{\dagger}(\vec{r}_1)\ldots\hat{\psi}^{\dagger}(\vec{r}_N) \ket{0}\label{eq:g50}\\
\hat{\psi}(\vec{r})\ket{0} = 0\label{eq:g51}\\
\left[
\hat{\psi}(\vec{r}),\hat{\psi}(\vec{r}^{\,\prime})
\right]_\mp 
= 0 = 
\left[
\hat{\psi}^{\dagger}(\vec{r}),\hat{\psi}^{\dagger}(\vec{r}^{\,\prime})
\right]_\mp\label{eq:g52}\\
\left[
\hat{\psi}(\vec{r}),\hat{\psi}^{\dagger}(\vec{r}^{\,\prime})
\right]_\mp = \delta(\vec{r} - \vec{r}^{\,\prime})\label{eq:g53}
\end{gather*}
\rule{\textwidth}{1pt}

\subsection{Fock space -- observables}

We will express observables in terms of $\hat{\psi}$ and $\hat{\psi}^\dagger$.
Let us consider first the Hermitian operator
\setcounter{equation}{56}
\be
\hat{\rho}(\vec{r})=\hat{\psi}^{\dagger}(\vec{r}) \hat{\psi}(\vec{r})
\ee
which is easier to work with -- same for bosons and fermions -- and examine its action on a many particle state:
\be
\begin{aligned}
\hat{\rho}(\vec{r})\ket{\vec{r}_{1} \ldots \vec{r}_{n}}
&=\hat{\rho}(\vec{r}) \hat{\psi}^{\dagger}\left(\vec{r}_{1}\right) \ldots \hat{\psi}^\dagger\left(\vec{r}_{1}\right)|0\rangle\to\\
&\to\text{move $\hat{\psi}$ until it hits }\ket{0}\to\\
&\left([\hat{\rho}(\vec{r}), \hat{\psi}^{\dagger}(\vec{r}_{1})]+
\hat{\psi}^{\dagger}(\vec{r}_{1}) \hat{\rho}(\vec{r})\right) 
\hat{\psi}^{\dagger}(\vec{r}_{2}) \ldots \hat{\psi}^{\dagger}(\vec{r}_{N}) \ket{0}\\
&=[\hat{\rho}(\vec{r}), \hat{\psi}^{\dagger}(\vec{r}_{1})]
\hat{\psi}^{\dagger}(\vec{r}_{2}) \ldots \hat{\psi}^{\dagger}(\vec{r}_{N}) \ket{0}\\
&+\hat{\psi}^{\dagger}(\vec{r}_{1})
[\hat{\rho}(\vec{r}), \hat{\psi}^{\dagger}(\vec{r}_{2})]
\hat{\psi}^{\dagger}(\vec{r}_{3}) \ldots \hat{\psi}^{\dagger}(\vec{r}_{N}) \ket{0}\\
&+\hat{\psi}^{\dagger}(\vec{r}_{1})\hat{\psi}^{\dagger}(\vec{r}_{2})
[\hat{\rho}(\vec{r}), \hat{\psi}^{\dagger}(\vec{r}_{3})]
\hat{\psi}^{\dagger}(\vec{r}_{4}) \ldots \hat{\psi}^{\dagger}(\vec{r}_{N}) \ket{0}\\
%%% 02 OKAY
\ldots\\
&+\hat{\psi}^{\dagger}(\vec{r}_{1})\ldots\hat{\psi}^{\dagger}(\vec{r}_{N-1})
[\hat{\rho}(\vec{r}), \hat{\psi}^{\dagger}(\vec{r}_{N})]\ket{0}\\
&+\hat{\psi}^{\dagger}(\vec{r}_{1})\ldots\hat{\psi}^{\dagger}(\vec{r}_{N})\underbrace{\hat{\rho}\ket{0}}_{=0}
\end{aligned}
\label{eq:g58}
\ee
We need the commutator $[\hat{\rho}(\vec{r}), \hat{\psi}^{\dagger}(\vec{r}^{\,\prime})]$ $\to$ Exercise:
\be
[\hat{\rho}(\vec{r}), \hat{\psi}^{\dagger}(\vec{r}^{\,\prime})] = \delta(\vec{r}-\vec{r}^{\,\prime})\hat{\psi}^\dagger(\vec{r})
\ee
independently if we're referring to bosons or fermions.
Using this into Eq.~\eqref{eq:g58}, one obtains:
\be
\begin{gathered}
\hat{\rho}(\vec{r})\ket{\vec{r}_{1} \ldots \vec{r}_{N}}
= \delta(\vec{r}-\vec{r}_1)
\hat{\psi}^{\dagger}(\vec{r})\hat{\psi}^{\dagger}(\vec{r}_{2}) \ldots \hat{\psi}^{\dagger}(\vec{r}_{N}) \ket{0}\\
+ \delta(\vec{r}-\vec{r}_2)
\hat{\psi}^{\dagger}(\vec{r}_1)\hat{\psi}^{\dagger}(\vec{r}) \hat{\psi}^{\dagger}(\vec{r}_{3}) \ldots \hat{\psi}^{\dagger}(\vec{r}_{N}) \ket{0}\\
+\ldots+\delta(\vec{r}-\vec{r}_N)
\hat{\psi}^{\dagger}(\vec{r}_1)\ldots \hat{\psi}^{\dagger}(\vec{r}_{N-1}) \hat{\psi}^{\dagger}(\vec{r})\ket{0}\\
= \sum_{a=1}^N \delta(\vec{r}-\vec{r}_a) \ket{\vec{r}_1\ldots\vec{r}_N}
\end{gathered}
\ee
therefore
\be
\boxed{
\rho(\vec{r}) = \sum_{a=1}^N \delta(\vec{r}-\vec{r}_a)
}
\ee
%%% 03 OKAY
When one integrates $\rho(\vec{r})$ over the volume:
\be
\int d^3r \rho(\vec{r}) = \sum_{a=1}^N \int d^3r \delta(\vec{r}-\vec{r}_a) = N \to \text{ \# of particles}
\label{eq:g62}
\ee
That is, $\hat{\rho}(\vec{r})$ is the density operator. It leads
to the definition of the \emph{number} operator
\be
\hat{N} = \int d^3r \hat{\rho}(\vec{r}) = \int d^3r \hat{\psi}^\dagger(\vec{r}) \hat{\psi} (\vec{r})
\ee
$\Rightarrow$ number of particles
\be
\boxed{
\hat{N}\ket{\vec{r}_1\ldots\vec{r}_N} = N\ket{\vec{r}_1\ldots\vec{r}_N}
}
\ee
$\ket{\vec{r}_1\ldots\vec{r}_N}$ is an eigenstate of the number operator.
Note that
\be
\begin{aligned}
[\hat{N},\hat{\psi}^\dagger(\vec{r})]
&=\int d^3r^\prime[\hat{\rho}(\vec{r}^{\,\prime}),\hat{\psi}^\dagger(\vec{r})]\\
&=\int d^3r^\prime\delta(\vec{r}^{\,\prime}-\vec{r}) \hat{\psi}^\dagger (\vec{r}^{\,\prime}) =\hat{\psi}^\dagger(\vec{r})
\end{aligned}
\ee
which implies
%%% 04 OKAY
\be
\hat{N}\hat{\psi}^\dagger(\vec{r}) = \hat{\psi}^\dagger(\vec{r})\hat{N} + \hat{\psi}^\dagger(\vec{r}) = 
\hat{\psi}^\dagger(\vec{r})(\hat{N}+1)
\ee
Also (exercise):
\be
\hat{N}\hat{\psi}(\vec{r}) = \hat{\psi}(\vec{r})(\hat{N}-1)
\ee

\subsubsection{Momentum operator}

We will verify that the operator
\be
\hat{\vec{P}}=\int d^{3}\vec{r} \hat{\psi}^{\dagger}(\vec{r})\left(-i \hbar \vecnr\right) \hat{\psi}(\vec{r})
\ee
satisfies the common properties one expects from the
momentum operator. First, we show that $\hat{\vec{P}}$ is Hermitean: $\hat{\vec{P}}^\dagger = \hat{\vec{P}}$
%
\be
\begin{aligned}
\hat{\vec{P}}^{\dagger}
&=\left[\int d^{3}\vec{r} \hat{\psi}^{\dagger}(\vec{r})\left(-i \hbar \vecnr\right) \hat{\psi}(\vec{r})\right]^{\dagger}
%%% 05 OKAY
=\int d^{3} \vec{r} \left[\hat{\psi}^{\dagger}(\vec{r})\left(-i \hbar \vecnr\right) \hat{\psi}(\vec{r})\right]^{\dagger}\\
&=i\hbar \int d^{3}\vec{r} \left[\hat{\psi}^{\dagger}(\vec{r})\vecnr \hat{\psi}(\vec{r})\right]^{\dagger}
 =i\hbar \int d^{3}\vec{r} \left[\vecnr\hat{\psi}(\vec{r})\right]^\dagger \left[\hat{\psi}^\dagger(\vec{r})\right]^\dagger\\
&=i\hbar \int d^{3}\vec{r} \underbrace{\left(\vecnr\hat{\psi}^\dagger(\vec{r}\right) \hat{\psi}(\vec{r})}%
_{\vecnr(\hat\psi^\dagger\hat\psi) - \psi^\dagger\vecnr\hat\psi}\\
&=i\hbar \int d^{3}\vec{r}
\Big\{
\underbrace{\vecnr\left[\hat{\psi}^{\dagger}(\vec{r})\hat{\psi}(\vec{r})\right]}%
_{\mathrlap{\text{zero for localized systems}}\quad}%
- \hat{\psi}^{\dagger}(\vec{r})\vecnr\hat{\psi}(\vec{r})
\Big\}\\
&=\int d^{3}\vec{r} \hat{\psi}^{\dagger}(\vec{r})\left(-i \hbar \vecnr\right) \hat{\psi}(\vec{r}) = \hat{\vec{P}}
\end{aligned}
\ee
In addition, we have to calculate the following commutator:
\[
\begin{gathered}
\left[\hat{\psi}(\vec{r}),\hat{P}\right] = 
\int d^3r^\prime
\left[
\hat{\psi}(\vec{r}),\hat{\psi}^\dagger(\vecrp)(-i\hbar\vec{\nabla}_{r^\prime}\hat{\psi}(\vecrp))
\right]\\
=-i\hbar \int d^3r^\prime
\bigg[
\underbrace{\hat{\psi}(\vec{r}) \hat{\psi}^\dagger(\vecrp)}_{\Circled{1}} \vec{\nabla}_{r^\prime} \hat{\psi}(\vecrp) -
\hat{\psi}^\dagger (\vecrp)\vec{\nabla}_{r^\prime} \hat{\psi}(\vecrp) 
\bigg]
\end{gathered}
\]
but 
\[
\Circled{1}=\hat{\psi}(\vec{r}) \hat{\psi}^\dagger(\vecrp) =
\left[
\hat{\psi}(\vec{r}), \hat{\psi}^\dagger(\vecrp)
\right]_\mp \pm  \hat{\psi}^\dagger(\vecrp)\hat{\psi}(\vec{r})
\]
so
%%% 06
\[
\begin{gathered}
\left[\hat{\psi}(\vec{r}),\hat{P}\right] = -i\hbar \int d^3r^\prime \times
\bigg\{
    \delta(\vec{r}-\vecrp) \left(\vec{\nabla_{r^\prime}}\hat{\psi}(\vecrp)\right)\\ 
	\pm \underbrace{\hat{\psi}^{\dagger}(\vecrp) \hat{\psi}(\vec{r}) \left(\vec{\nabla_{r^\prime}}\hat{\psi}(\vecrp)\right)}%
	_{\Circled{2}}
	- \underbrace{\hat{\psi}^{\dagger}(\vecrp) \left(\vec{\nabla_{r^\prime}}\hat{\psi}(\vecrp)\right) \hat{\psi}(\vec{r})}%
	_{\Circled{3}}
\bigg\}\\
\end{gathered}
\]
but $\Circled{2} = \pm \hat{\psi}^{\dagger}(\vecrp) \left(\vec{\nabla_{r^\prime}}\hat{\psi}(\vecrp)\right) \hat{\psi}(\vec{r}) = \pm\Circled{3}$; so, the two $\pm$ signs combine, and we are left only with the subtraction in the middle,
\textit{i.e.} $\pm \Circled{2} - \Circled{3} = 0$. So
\[
\begin{gathered}
\left[\hat{\psi}(\vec{r}),\hat{P}\right]
= -i\hbar \int d^3r^\prime \delta(\vec{r}-\vecrp) \vec{\nabla}_{r^\prime} \hat{\psi}(\vecrp) = 
-i\hbar \vecnr \hat{\psi}(\vec{r})
\end{gathered}
\]
therefore %
\setcounter{equation}{70}
\be
\left[\hat{\psi}(\vec{r}),\hat{P}\right] = -i\hbar \vecnr \hat{\psi}(\vec{r})
\ee
and that also implies
\be
\left[\hat{\psi}^\dagger(\vec{r}),\hat{P}\right] = -i\hbar \vecnr \hat{\psi}^\dagger(\vec{r})
\ee
Since $\hat{\psi}\ket{0} = 0$, one has that the zero-particle
state, the ``vacuum'', has zero momentum:
\be
\boxed{\vec{P}\ket{0}=0}
\ee

\emph{Exercise:}
\be
\begin{gathered}
\bra{\vec{r}_1\ldots\vec{r}_N} \hat{\vec{P}}\ket{\Psi} =
\int d^{3} r_{1}^{\prime} \cdots d^{3} r_{N}^{\prime}
\bra{\vec{r}_1\ldots\vec{r}_N} \hat{\vec{P}} \ket{\vec{r}_1^{\,\prime}\ldots\vec{r}_N^{\,\prime}}
\bra{\vec{r}_1^{\,\prime}\ldots\vec{r}_N^{\,\prime}}\ket{\Psi}\\
\int d^{3} r_{1}^{\prime} \cdots d^{3} r_{N}^{\prime}
\bra{0}\hat{\psi}(\vec{r}_1)\ldots\hat{\psi}(\vec{r}_N)
\hat{\vec{P}}
\hat{\psi}^\dagger(\vec{r}_1^{\,\prime})\ldots\hat{\psi}^\dagger(\vec{r}_N^{\,\prime})\ket{0}\\
\ldots\\
=\sum_{a=1}^{N}\left(-i \hbar \vec{\nabla}_{r_{a}}\right) \Psi\left(\vec{r}_{1}, \ldots, \vec{r}_{N}\right)
\end{gathered}
\ee
that is, the first-quantized the total momentum of an $N$-particle system.

%%% 07 OKAY
\subsection{Fock space -- one-body operators}

The Fock-space representation of an \emph{one-body} operator
$\hat{O}_1(\hat{\vec{r}})$, whose elements in first quantization
are given by
\be
\bra{\vecrp}\hat{O}_1(\hat{\vec{r}})\ket{\vec{r}} = 
O_1(\vec{r}) \delta(\vecrp-\vec{r})
\ee
It is a diagonal operator, for we are taking the matrix elements, 
in between $\bra{\vecrp}$ and $\ket{\vec{r}}$,
of operators that are written as functions of the first-quantized operator $\hat{\vec{r}}$ itself.
After we take the matrix elements, these are numbers,
so they lose their operator status ($\hat{O} \to O$, lose their ``hats'').
In Fock space the second-quantized operator is then represented by
\be
\begin{aligned}
\hat{O}_1
&=\int d^3r^\prime d^3r\, \psi^\dagger(\vecrp)
\bra{\vecrp}\hat{O}_1(\hat{\vec{r}})\ket{\vec{r}}\psi{\vec{r}}\\
&=\int d^3r\, \psi^\dagger(\vec{r}) O_1(\vec{r}) \psi(\vec{r})
\end{aligned}
\ee
Note that  the momentum operator, introduced
above, is consistent with this, since
\[
\begin{aligned}
\hat{\vec{P}} 
&= \int d^3r^\prime d^3r\, 
\hat{\psi}^\dagger(\vecrp) \underbrace{\bra{\vecrp}\vec{P}\ket{\vec{r}}}%
_{-i\hbar\vec{\nabla}_{\vec{r}^{\,\prime}}\delta(\vecrp-\vec{r})}
\hat{\psi}(\vec{r})\\
&= \int d^3r^\prime d^3r\,
\hat{\psi}^\dagger(\vecrp)
\left[-i\hbar\vec{\nabla}_{\vec{r}^{\,\prime}}\delta(\vecrp-\vec{r})\right]
\hat{\psi}(\vec{r})\\
&= \int d^3r\,
\hat{\psi}^\dagger(\vec{r})\left(-i\hbar\vecnr\right)\hat{\psi}(\vec{r})
\end{aligned}
\]
which is the same expression as given in Eq.~\eqref{eq:g62}.

%%% 08 OKAY

The rule is: 
\begin{itemize}
\item sandwich the matrix element in first
quantization between $\hat{\psi}^{\dagger}$and $\hat{\psi}$.
\end{itemize}
For the important case of a single-particle
Hamiltonian, this leads to:
\be
\hat{H}_1(\vec{r}) = \int d^3r\,
\hat{\psi}^\dagger(\vec{r})
\left[
\underbrace{-\frac{\hbar^2}{2m}\vecnr^2}_{\text{kinetic energy}} 
+ 
\underbrace{V_1(\vec{r})}_{\text{external potential}} 
\right]
\hat{\psi}(\vec{r})
\ee
leading to the following properties:
\begin{enumerate}
\item $\hat{H}_1(\vec{r})$ is Hermitian
%
\item $\hat{H}_1(\vec{r})\ket{0}=0$
%
\item
\be
\bra{\vec{r}_1\ldots\vec{r}_N}\hat{H}_1\ket{\Psi} =
\sum_{a=1}^n
\left[
-\frac{\hbar^2}{2m} \vec{\nabla}_{r_a} + V_1(\vec{r}_a)
\right]
\Psi(\vec{r}_1\ldots\vec{r}_N)\,\to\,\text{Exercise}
\ee
\end{enumerate}
The eigenvalue equation
\[
\hat{H}_1\ket{\Psi} = E\ket{\Psi}
\]
leads to the traditional expression:
\be
\sum_{a=1}^n
\left[
-\frac{\hbar^2}{2m} \vec{\nabla}_{r_a} + V_1(\vec{r}_a)
\right] \Psi(\vec{r}_1\ldots\vec{r}_N)
= E \Psi(\vec{r}_1\ldots\vec{r}_N)
\ee
Notice that here all the particles have are their kinetic energies 
and interaction with an external potential $V_1(\vec{r})$.
In other words, we are disregarding mutual interactions between the particles;
that is the subject of the next subsection.

\subsection{Fock space -- two-body operators}

Mutual interaction: $\hat{V}_2 = \hat{V}_2(\hat{\vec{r}}^{\,\prime},\hat{\vec{r}})$
$\to$ Hermitian, symmetric in $\vecrp \leftrightarrow \vec{r}$.

%%% 09 OKAY

Matrix elements in first quantization is
\be
\bra{\vec{r}_1^{\,\prime} \vec{r}_1^{\,\prime}} \hat{V}_2(\vecrp,\vec{r})\ket{\vec{r}_1 \vec{r}_2}
\delta(\vec{r}_1^{\,\prime} - \vec{r}_1)\delta(\vec{r}_2^{\,\prime} - \vec{r}_2) V_2(\vec{r}_1, \vec{r}_2)
\ee 
where $V_2(\vec{r}_1, \vec{r}_2)$ is real. 
The above sandwich rule leads to
\be
\begin{aligned}
\hat{V}_2 
&= 1/2 \int d^3r_1^\prime\ldots d^3r_2\,
\hat{\psi}^\dagger(\vec{r}_1^{\,\prime})\hat{\psi}^\dagger(\vec{r}_2^{\,\prime}) 
\bra{\vec{r}_1^{\,\prime} \vec{r}_2^{\,\prime}} \hat{V}_2\ket{\vec{r}_1 \vec{r}_2}
\hat{\psi}(\vec{r}_2)\hat{\psi}(\vec{r}_1)\\
&= 1/2 \int d^3r^\prime d^3r^\prime\,
\hat{\psi}^\dagger(\vec{r})\hat{\psi}^\dagger(\vecrp) 
V_2(\vec{r},\vecrp)
\hat{\psi}(\vecrp)\hat{\psi}(\vec{r})
\end{aligned}
\ee
%
This operator is Hermitean
\be
\begin{aligned}
&\left(\int d^3r d^3r^\prime\,
\hat{\psi}^\dagger(\vec{r})\hat{\psi}^\dagger(\vecrp) 
V_2(\vec{r},\vecrp)
\hat{\psi}(\vecrp)\hat{\psi}(\vec{r})\right)^\dagger\\
&= \int d^3r d^3r^\prime\,
\left(\hat{\psi}(\vecrp)\hat{\psi}(\vec{r})\right)^\dagger
V_2(\vec{r},\vecrp)
\left(\hat{\psi}^\dagger(\vec{r})\hat{\psi}^\dagger(\vecrp)\right)^\dagger\\
&=\int d^3r d^3r^\prime\,
\hat{\psi}^\dagger(\vec{r})\hat{\psi}^\dagger(\vecrp) 
V_2(\vec{r},\vecrp)
\hat{\psi}(\vecrp)\hat{\psi}(\vec{r})
\end{aligned}
\ee
\begin{quote}
\emph{OBS:} Note the order of $\vec{r}$ and $\vecrp$ in $\hat{\psi}^\dagger(\vec{r})\hat{\psi}^\dagger(\vecrp)$
and $\hat{\psi}(\vecrp)\hat{\psi}(\vec{r})$ $\to$ this order of the
operators is absolutely essential.
\end{quote}
A Hamiltonian containing the interaction with
%%% 10
an external potential and the mutual interaction
can be written as
\be
\begin{gathered}
\hat{H}_2 = \hat{H}_1 + \hat{V}_2 \\
= 
\int d^3r\,
\hat{\psi}^\dagger(\vec{r})
\left[
-\frac{\hbar^2}{2m}\vecnr^2 + V_1(\vec{r}) 
\right]\hat{\psi}(\vec{r})
\\
+1/2 \int d^3r d^3r^\prime\,
\hat{\psi}^\dagger(\vec{r})\hat{\psi}^\dagger(\vecrp) 
V_2(\vec{r},\vecrp)
\hat{\psi}(\vecrp)\hat{\psi}(\vec{r})
\end{gathered}
\ee

Let us check the eigenvalue equation
\[
\hat{H}_2 \ket{\Psi} = (\hat{H}_1 + \hat{V}_2)\ket{\Psi}
\]
that is, the first-quantized version is:
\[
\begin{gathered}
\sum_{a=1}^{N} \left(
-\frac{\hbar^{2}}{2 m} \vec{\nabla}_{a}^{2}+V_{1}\left(\vec{r}_{a}\right)\right) 
\Psi\left(\vec{r}_{1}, \ldots, \vec{r}_{N}
\right)\\
+1 / 2 \sum_{a, b} V_{2}\left(\vec{r}_{a}, \vec{r}_{b}\right) \Psi\left(\vec{r}_{1}, \ldots, \vec{r}_{N}\right) 
= E \Psi\left(\vec{r}_{1}, \ldots, \vec{r}_{N}\right)
\end{gathered}
\]

\begin{quote}
Note that one has to do large calculation to get to the first-quantized version shown above. 
This is left as an exercise.
\end{quote}








\end{document}
