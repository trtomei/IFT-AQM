%!TEX TS-program = xelatex
%!TEX encoding = UTF-8 Unicode

\documentclass[12pt]{article}
\usepackage{geometry}                % See geometry.pdf to learn the layout options. There are lots.
\geometry{a4paper,top=2cm}
\usepackage[parfill]{parskip}    % Activate to begin paragraphs with an empty line rather than an indent
\usepackage{graphicx}
\usepackage{amsmath}
\usepackage{amssymb}
\usepackage{mathtools}
\usepackage{physics}
\newcommand{\be}{\begin{equation}}
\newcommand{\ee}{\end{equation}}
\usepackage[thicklines]{cancel}
\usepackage{url}
\usepackage{booktabs}
\usepackage{csquotes}
\usepackage{qcircuit}
\usepackage{circledsteps}
\usepackage{nicefrac}
\usepackage{fontspec,xltxtra,xunicode}
\usepackage{xcolor}
\defaultfontfeatures{Mapping=tex-text}

\newcommand{\polv}{\ensuremath{\updownarrow}}
\newcommand{\polh}{\ensuremath{\leftrightarrow}}
\newcommand{\poldr}{\rotatebox[origin=c]{45}{\ensuremath{\leftrightarrow}}}
\newcommand{\poldl}{\rotatebox[origin=c]{-45}{\ensuremath{\leftrightarrow}}}
\newcommand{\bigzero}{\mbox{\normalfont\Large\bfseries 0}}

\title{Advanced Quantum Mechanics\\Class 23 (a)}
%\author{The Author}
\date{November 1, 2022}                                           % Activate to display a given date or no date

\setcounter{section}{6}

\begin{document}
\maketitle

%%% 01 OKAY
\section{Approximation Methods}

\subsection{Time-independent perturbation theory}

\subsubsection{Non-degenerate case}

Suppose one can break up the Hamiltonian of the
problem of interest into two pieces:
\be
\hat{H}=\hat{H}_{0}+\lambda \hat{W}
\ee
where $\lambda$ is a parameter.

%%% 02 OKAY

Let $\{\ket*{\varphi_{n}^{(0)}}\}$ be a complete set of eigenstates of $\hat{H}_{0}$,
with eigenvalues $E_{n}^{(0)}$ :
\be
\left(H_{0}-E_{n}^{(0)}\right)\left|\varphi_{n}^{(0)}\right\rangle= 0
\ee
The aim is to find the eigenstates $\left\{\left|\varphi_{N}\right\rangle\right\}$ and eigenvalues
$E_{N}$ of the complete Hamiltonian:
\be
\left(\hat{H}-E_{N}\right)\left|\varphi_{N}\right\rangle=0
\label{eq:g3}
\ee
where $n,m,\ldots$ unperturbed; $N,M,\ldots$: perturbed

The exact values of $E_{N}$ can be found by solving
the secular equation in the unperturbed basis:
\be
\left|\varphi_{N}\right\rangle=\sum c_{n}\left|\varphi_{n}^{(0)}\right\rangle
\ee

%%% FIXME
(Missing steps)
\setcounter{equation}{6}

\be
\operatorname{det}\left[\delta_{mn}\left(E_{N}-E_{n}^{(0)}\right)-\lambda\left\langle\varphi_{m}^{(0)}|\hat{W}|\varphi_{n}^{(m)}\right\rangle\right]=0
\ee
In general, this turns out to be impractical, matrices are too big.

%%% 03 OKAY

A more practical method is needed:
Consider first the situation that the spectrum of
$\hat{H_{0}}$ is \emph{nondegenerate}.
In such a case, in general, perturbed and unperturbed
quantities are in one-to-one correspondence:
\be
\left|\varphi_{N}\right\rangle \leftrightarrow\left|\varphi_{n}^{(0)}\right\rangle, E_{N} \leftrightarrow E_{n}^{(0)} \quad \text{ as }\lambda \rightarrow 0
\ee
and one can expand the perturbed states $\left|\varphi_{n}\right\rangle$
in terms of the unperturbed $|\varphi_{n}^{(0)}\rangle$:
\be
\left|\varphi_{N}\right\rangle=c_{n}\left|\varphi_{n}^{(0)}\right\rangle
+
\sum_{m \neq n} d_{m}\left|\varphi_{m}^{(0)}\right\rangle
\label{eq:g9}
\ee
therefore
\be
\left|c_{n}\right|^{2}+\sum_{m \neq n}\left|d_{m}\right|^{2}=1
\label{eq:g10}
\ee
Perturbation theory aims at calculating $E_{N}$ and the
$c_n$ and $d_n$ as a power series in $\lambda$ by requiring
that the expansion for $\left|\varphi_{N}\right\rangle$ satisfies the
eigenvalue equation.
Inserting the expansion into $\langle\varphi_{k}^{(0)}|(\hat{H}-E_{N})|\varphi_{N}\rangle=0$,
%%% 04 OKAY
with $k\neq n$, one obtains

%%% FIXME
(Missing steps)
\setcounter{equation}{12}

\be
d_{k} =\lambda c_{n} \frac{\left\langle\varphi_{k}^{(0)}|\hat{W}| \varphi_{n}^{(0)}\right\rangle}{E_{N}-E_{k}^{(0)}}+\mathcal{O}\left(\lambda^{2}\right)
\label{eq:g13}
\ee
This then means that Eq.~\eqref{eq:g10} implies
\be
\left|c_{n}\right|^{2}=1-\sum_{m \neq n}\left|d_{m}\right|^{2}=1-\mathcal{O}\left(\lambda^{2}\right)
\ee
which gives (up to a phase)
\be
c_{n}=1-\mathcal{O}\left(\lambda^{2}\right)
\ee
so $c_n$ in the development and, finally, in Eq.~\eqref{eq:g13} can be replaced by 1:
%%% 05 OKAY
\be
d_{k}=\lambda \frac{\left\langle\varphi_{k}^{(0)}|\hat{W}| \varphi_{n}^{(0)}\right\rangle}{E_{N}-E_{k}^{(0)}}
\ee

Since $E_{N}$ will be obtained as a power series in $\lambda$,
to leading order in $\lambda$, $E_{N}$ can be replaced by $E_{n}^{(0)}$ :
\be
d_{k}=\lambda \frac{\left\langle\varphi_{k}^{(0)}|\hat{W}| \varphi_{n}^{(0)}\right\rangle}{E_{n}^{(0)}-E_{k}^{(0)}}
\ee
Using this result into the expansion in Eq.~\eqref{eq:g9}, one
obtains the order $\lambda$ correction to the state vector:
\be
\boxed{%
\left|\varphi_{N}\right\rangle=\left|\varphi_{n}^{(0)}\right\rangle+\lambda \sum_{m \neq n} \frac{\left\langle\varphi_{m}^{(0)}|\hat{W}| \varphi_{n}^{(0)}\right\rangle}{E_{n}^{(0)}-E_{m}^{(0)}}\left|\varphi_{n}^{(0)}\right\rangle+O\left(\lambda^{2}\right)
}
\label{eq:g18}
\ee
Validity of such a formula requires:
\be
\left|\lambda \frac{\left\langle\varphi_{m}^{(0)}|\hat{W}| \varphi_{n}^{(0)}\right\rangle}{E_{n}^{(0)}-E_{m}^{(0)}}\right| \ll 1\hspace{5em}%
\begin{aligned}
&\text{\rule[0.5ex]{5em}{1pt}}\quad E_n^{(0)}\\
&\rotatebox[origin=c]{90}{\ensuremath{\xleftrightarrow{\hspace{3em}}}}\\
&\text{\rule[0.5ex]{5em}{1pt}}\quad E_m^{(0)}
\end{aligned}
\ee
This gets \emph{better and better} as $|E_{n}^{(0)}-E_{m}^{(0)}|$ increases
while the off-diagonal matrix elements of $\hat{W}$
do not grow. On the contrary, the expansion fails
when $|E_{n}^{(0)}-E_{m}^{(0)}|$ $\rightarrow 0$ (near degeneracy).

%%% 06 OKAY
The perturbed energies are obtained from
\[
\left\langle\varphi_{n}^{(0)}\left|\left(\hat{H}-E_{N}\right)\right| \varphi_{N}\right\rangle=0
\]

%%% FIXME
(Missing steps)
\setcounter{equation}{21}

\be
\boxed{
\begin{aligned} 
E_{N} &=E_{n}^{(0)}+\lambda\left\langle\varphi_{n}^{(0)}|\hat{W}| \varphi_{n}^{(0)}\right\rangle \\ 
&+\lambda^{2} \sum_{m \neq n} \frac{\left|\left\langle\varphi_{m}^{(0)}|\hat{W}| \varphi_{n}^{(0)}\right\rangle\right|^{2}}{E_{n}^{(0)}-E_{n}^{(0)}}+
\mathcal{O}\left(\lambda^{3}\right) 
\end{aligned}
\label{eq:g22}
}
\ee
This is a typical characteristic of perturbation theory: using the same matrix elements
-- $\bra*{\varphi_{n}^{(0)}}\hat{W}\ket*{\varphi_{n}^{(0)}}$ in this case --,
one can determine the eigenvalues to one order better than the eigenstates:
$\mathcal{O}(\lambda^2)$ in the former, $\mathcal{O}(\lambda)$ in the latter.

%%% PULA O EXEMPLO

%%% 08 (Começa a teoria de perturbação degenerada!) OKAY MENOS FIGURA

\subsubsection{Degenerate case}

From Eqs.~\eqref{eq:g18} and \eqref{eq:g22} one sees that one will
encounter problems when perturbing levels
that have nearby neighbors, \emph{even when} $\lambda$ is small
$\Rightarrow$ \textit{i.e.} when the unperturbed spectrum
is degenerate or nearly degenerate.
In such a situation, \emph{degenerate} perturbation
theory must be used.

Suppose one has a $\hat{H}_0$ with a spectrum as in
\[
\begin{aligned}
&\,\text{\rule[0.5ex]{10em}{1pt}}\quad m^\prime\\
\\
%
\left.%
\begin{gathered}
\overline{H}^{(n)}\text{-- nearly}\\
\text{degenerate multiplet}
\end{gathered}
\right\{
%
&\,\text{\rule[-10pt]{10em}{24pt}}\quad \bar{n}\\
\\
&\,\text{\rule[0.5ex]{10em}{1pt}}\quad m
\end{aligned}
\]

Unperturbed spectrum $\to$ 
use of a bar ($\overline{\phantom{x}}$)
to indicate a level in $\overline{H}^{(n)}$.

That is, $\hat{H}_0$ has a spectrum that contains a
degenerate or nearly degenerate subspace $\overline{H}^{(n)}$ spanned
by the unperturbed states $\{\ket{\overline{n}} \to \ket{\overline{n},r} \}$
$\Rightarrow$ aim is to find the \emph{perturbed} spectrum \emph{in $\overline{H}^{(n)}$.}

%%% 09

It does not matter whether or not the levels outside
$\overline{H}^{(n)}$ are degenerate: all that matters is that the
unperturbed levels $m$ and $m^\prime$ are distant from
$\overline{H}^{(n)}$, that is: %
\setcounter{equation}{24} %
\be
|\lambda\langle\bar{n}|\hat{W}| m\rangle| \ll \left|E_{\bar{n}}^{(0)}-E_{m}^{(0)}\right|
\ee
Within $\overline{H}^{(n)}$ no constraint on $\langle n|\hat{W}| m\rangle$
relative to energy splittings, which can actually
vanish (degeneracy).

The expansion in Eq.~\eqref{eq:g9} is changed to:
\be
\left|\varphi_{N}\right\rangle=\sum_{\bar{n}} c_{\bar{n}}\left|\varphi_{\bar{n}}^{(0)}\right\rangle+\sum_{m \neq \bar{n}} d_{m}\left|\varphi_{m}^{(0)}\right\rangle
\label{eq:g26}
\ee
in which one expects $c_{n} \sim \mathcal{O}\left(\lambda^{0}\right)$ and presumably
$d_m \sim \mathcal{O}(\lambda)$. Plugging the expression \eqref{eq:g26} into the
eigenvalue equation \eqref{eq:g3}, one obtains:
\be
\begin{gathered}
\sum_{\bar{n}} c_{\tilde{n}}\left(E_{\bar{n}}^{(0)}-E_{N}+\lambda \hat{W}\right)\left|\varphi_{\bar{n}}^{(0)}\right\rangle\\
+\sum_{m\neq\bar{n}} d_{m}\left(E_{m}-E_{N}+\lambda \hat{W}\right)\left|\varphi_{m}^{(0)}\right\rangle = 0\end{gathered}
\label{eq:g27}
\ee

%%% 10

First, let us project this equation onto some state $k$
outside of $\overline{H}^{(n)}$, $\ket*{\varphi_k^{(0)}}$:
\be
\begin{gathered}
\lambda \sum_{\bar{n}} c_{\bar{n}}\left\langle\varphi_{k}^{(0)}|\hat{W}| \varphi_{\bar{n}}^{(0)}\right\rangle+d_{k}\left(E_{k}^{(0)}-E_{N}\right)\\
+\lambda \sum_{m \neq \bar{n}} d_{m}\left\langle\varphi_{k}^{(0)}\right| \hat{W}\left|\varphi_{m}^{(0)}\right\rangle=0
\end{gathered}
\ee
hence, remembering that $d_m \sim \mathcal{O}(\lambda)$
\be
d_{k}=\frac{\lambda}{E_{k}^{(0)}-E_{N}} \sum_{\bar{n}} c_{\bar{n}}\left\langle\varphi_{k}^{(0)}|\hat{W}| \varphi_{\bar{n}}^{(0)}\right\rangle
\ee
$E_{N}$ can be replaced by \emph{any} $E_{\bar{n}}^{(0)}$ of $\bar{H}^{(n)}$, since
their mutual separation is negligible (zero in
case of degeneracy) in comparison with $E_{k}^{(0)}-E$.
We can replace it by $\overline{E}$ some mean energy in $\overline{H}^{n}$:
\be
d_{k}=\frac{\lambda}{\overline{E}-E_{k}^{(0)}} \sum_{\bar{n}} c_{\bar{n}}\left\langle\varphi_{k}^{(0)}\right| \hat{W}\left|\varphi_{\bar{n}}^{(0)}\right\rangle
\ee

Now, let us go back to Eq.~\eqref{eq:g27} and project it
on a state $\bar{n}^\prime$ in $\overline{H}^{(n)}$:

%%% FIXME
(Missing steps)
\setcounter{equation}{31}

%%% 11

\be
\begin{gathered}
c_{\bar{n}}\left(E_{\bar{n}}^{(0)}-E_{N}\right)+
\sum_{\bar{m}} c_{\bar{m}}\left(\lambda\left\langle\varphi_{\bar{n}}^{(0)}|\hat{W}| \varphi_{\bar{m}}^{(0)}\right\rangle\right.\\
\left.+\lambda^{2} \sum_{m \neq \bar{m}} \frac{\left\langle\varphi_{\bar{n}}^{(0)}|\hat{W}| \varphi_{m}^{(0)}\right\rangle\left\langle\varphi_{m}^{(0)}|\hat{W}| \varphi_{\bar{m}}^{(0)}\right\rangle}{\overline{E}-E_{m}^{(0)}}\right)=0
\end{gathered}
\label{eq:g32}
\ee
$\Rightarrow$ this is of the form
\be
c_{\bar{n}}\left(E_{\bar{n}}^{(0)}-E_{N}\right)+
\sum_{\bar{m}} c_{\bar{m}}\left\langle\varphi_{\bar{n}}^{(0)}\left|\hat{H}_{\text {eff}}\right| \varphi_{\bar{m}}^{(0)}\right\rangle = 0
\label{eq:g33}
\ee
with $\langle\varphi_{\bar{n}}^{(0)}|\hat{H}_{\text{eff}}|\varphi_{\bar{m}}^{(0)}\rangle$ given by the expression
within the parenthesis in Eq.~\eqref{eq:g32}. In operator
form, $\hat{H}_{\text{eff}}$ can be written as:
\be
\hat{H}_{\text{eff}} = \lambda \hat{\overline{P}} \hat{W} \hat{\overline{P}} +
\lambda^{2}  \hat{\overline{P}} \hat{W} \frac{\hat{\overline{Q}}}{\overline{E} - \hat{H}_0} \hat{W}
\hat{\overline{P}}
\ee
with 
\be
\hat{\overline{P}} = \sum_m \op*{\varphi_{\overline{n}}^{(0)}},\quad \hat{\overline{Q}} = 1 - \hat{\overline{P}}
\ee

Eq.~\eqref{eq:g33} is nothing but the eigenvalue problem in
the subspace $\overline{H}^{(n)}$: 
\emph{one cannot escape having to solve an eigenvalue problem,
but now considering an effective Hamiltonian $H_{\text{eff}}$}.
From this, one obtains the sought
perturbed eigenvalues $E_a$.

\emph{Simplest situation:} $W$ only has matrix elements
within $\overline{H}^{(n)}$: the $\mathcal{O}(\lambda^2)$ term in (34) is zero, and the
calculation amounts to diagonalizing the
perturbation with $\overline{H}^{(n)}$.

Example:

\[
\hat{H}
=\begin{pmatrix}0 & 0 & \lambda w \\ 0 & 0 & \lambda w \\ \lambda w & \lambda w & E_{0}\end{pmatrix}
=\begin{pmatrix}0 & 0 & 0 \\ 0 & 0 & 0 \\ 0 & 0 & E_{0}\end{pmatrix}
+\begin{pmatrix}0 & 0 & \lambda w \\ 0 & 0 & \lambda w \\ \lambda w & \lambda w & 0\end{pmatrix}
\]
=
\[
\begin{gathered}
0\op{1} + 0\op{2} + E_{0}\op{3} \\ 
+\lambda w(\op{1}{3} + \op{2}{3} + \op{3}{1} + \op{3}{2})
\end{gathered}
\]

$|1\rangle,|2\rangle, |3\rangle$ unperturbed eigenvectors.

\end{document}


$\overline{H}^{(n)}$















