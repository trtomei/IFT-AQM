%!TEX TS-program = xelatex
%!TEX encoding = UTF-8 Unicode

\documentclass[12pt]{article}
\usepackage{geometry}                % See geometry.pdf to learn the layout options. There are lots.
\geometry{a4paper,top=2cm}
\usepackage[parfill]{parskip}    % Activate to begin paragraphs with an empty line rather than an indent
\usepackage{graphicx}
\usepackage{amsmath}
\usepackage{amssymb}
\usepackage{mathtools}
\usepackage{physics}
\newcommand{\be}{\begin{equation}}
\newcommand{\ee}{\end{equation}}
\usepackage[thicklines]{cancel}
\usepackage[colorlinks=true,citecolor=blue,linkcolor=blue,urlcolor=blue]{hyperref}
\usepackage{booktabs}
\usepackage{csquotes}
\usepackage{qcircuit}
\usepackage{circledsteps}
\usepackage{nicefrac}
\usepackage{fontspec,xltxtra,xunicode}
\usepackage{xcolor}
\usepackage{simplewick}
\defaultfontfeatures{Mapping=tex-text}

\newcommand{\polv}{\ensuremath{\updownarrow}}
\newcommand{\polh}{\ensuremath{\leftrightarrow}}
\newcommand{\poldr}{\rotatebox[origin=c]{45}{\ensuremath{\leftrightarrow}}}
\newcommand{\poldl}{\rotatebox[origin=c]{-45}{\ensuremath{\leftrightarrow}}}
\newcommand{\bigzero}{\mbox{\normalfont\Large\bfseries 0}}
\newcommand{\vecrp}{\ensuremath{\vec{r}^{\,\prime}}}
\newcommand{\vecnr}{\ensuremath{\vec{\nabla}_{\!r}}}

\title{Advanced Quantum Mechanics\\Class 12}
%\author{The Author}
\date{September 15, 2022}                                           % Activate to display a given date or no date

\setcounter{section}{5}
\setcounter{subsection}{13}

\begin{document}
\maketitle

%%% 01 OKAY

\subsection{Quantum cryptography}

We will discuss quantum key distribution (QKD),
particularly the so-called \emph{BB84} protocol,
using as example photon polarization.
$\ket{x} \to 1,\,\ket{y} \to 0$: message in binary code
(sequence of \emph{ones} and \emph{zeros}),
\textit{e.g.} $100100 \to xyyxyy$.
Alice: sender, Bob: receiver --
uses \textit{e.g.}, birefringent
plate to separate the
photons of vertical and
horizontal polarizations,
to reconstruct the message.

\emph{Not} good: a spy (Eve, for `` eavesdropper'') collects the
photons, analyses them, and resends
other photons in the same sequence
to Bob: Bob has no way of telling
that it was eve that sent the photons.
\emph{QKD}: uses the features of incompatibility of two
different bases of linear polarization.

\begin{enumerate}
\item Alice sends to Bob photons in four different
different polarization states:
%%% 02 OK
\begin{itemize}
\item polarization along $Ox (\updownarrow)$ and $Oy (\leftrightarrow)$,
\item polarization along axes rotated by $45^\circ$ $Ox^\prime (\poldl)$ and $Oy\prime (\poldr)$,
\item attribute bit 1 to states $\updownarrow$ and $\poldl$,
      attribute bit 0 to states $\leftrightarrow$ and $\poldr$.
\end{itemize}
\item Bob uses analysers oriented as $+$ or $\times$. 
Bob records 1 if photon has
polarization $\updownarrow$ or $\poldl$, and 0 if
photon has polarization $\leftrightarrow$ or $\poldr$.
\item After recording $N \gg 1$ photons, Bob
announces publicly the sequence of 
analyzers he used, but he \emph{does not}
announce his results (\textit{i.e.} his sequence of
ones and zeros).
\item Alice then compares what she has sent
him and announces publicly those
%%% 03 OK
polarizers she used that are compatible
with Bob's analysers, That is, Alice
sends Bob a list of possible orientations
that Bob should have gotten her bits.
\item Example of a possible communication
between Alice and Bob:

\begin{tabular}{cccccccccc}
\toprule
Alice's polarizers & \polv & \polh & \poldr & \polv & \poldr & \poldr & \poldl & \polv & \poldl \\
Sequence of bits & 1 & 0 & 0 & 1 & 0 & 0 & 1 & 1 & 1\\
Bob's analyzers & + & $\times$ & + & + & $\times$ & $\times$ & + & + & $\times$\\
Bob's results & 1 & 1 & 0 & 1 & 0 & 0  & 1 & 1 & 1\\
\midrule
Alice's list & \\
of compatible & Y & N & N & Y & Y & Y & N & Y & Y\\
YES (Y), NO (N) & \\
\midrule
retain then the &\\
compatible bits & 1 & -- & -- & 1 & 0 & 0 & -- & 1 & 1\\ 
\bottomrule
\end{tabular}
\[
\boxed{\text{Key} = 110011}
\]
\end{enumerate}
Both, Alice and Bob will agree on a key at the end
of the process \(\rightarrow\) they can use the key to \textit{e.g.}
encrypt messages.

%%% 04 OK
Can Eve know the key? \emph{NO!}
Suppose Alice sends \polv; if Eve detects it 
using $\times$, she (Eve) will make a mistake
supposing the photon had polarization \poldr or \poldl.
If Eve sends to Bob \poldr, 50\% of
the cases Bob will obtain the wrong result.
Eve has a 50\% chance of getting it
right, as she can use + or $\times$.
Alice and Bob 
%
can compare a random subset of their shared key and
%
will register 
a difference in 25\% of cases and
conclude the message was intercepted.

%%% A discussão sobre spin-1/2 ficou jogada aqui...

%%% 05 OK

\subsection{Quantum information, quantum computation}

(A glimpse of)

Theory of processing and transmitting of information
using quantum mechanics features: among others,
\emph{superposition} and \emph{entanglement}.

Classical information theory: basic unit of infomation is
the bit \(\leftarrow\) it takes two values: 0,1

Quantum information theory: basic unit is the qubit
e.g. spin \(1 / 2\).
\be
\text { Conventional: }
\left\{
\begin{aligned}  
&|+\rangle \equiv|0\rangle \\ 
&|-\rangle \equiv|1\rangle 
\end{aligned}
\right.
\ee

In contrast to a classical system (which exists
in states 0,1 only), a quantum system can
exist in \emph{linear superpositions}:
\be
|\varphi\rangle=\lambda|0\rangle+\mu|1\rangle
\ee
\fbox{qubit}: any two-level system,
not necessarily spin-1/2 particles.

%%% 06 OK

A feature of linear superposition:
- as seen in QKD example, it allows Alice and
Bob to know when an eavesdropper (Eve)
intercepts the exchanged message $\rightarrow$
this is a consequence of the
\emph{no cloning theorem}.

Recall that one cannot get everything about a
state \(|x\rangle\) in a single measurement:
\begin{itemize}
\item measurement of an observable \(A\) gives
one of the eigenvalues of \(\hat{A}(\leftarrow\) represents $A$,
independently of the state \(|\chi\rangle\) on which
$A$ is measured.
\item to reconstruct \(|x\rangle\) (or, more generally \(\hat{\rho}\),
several values (likely of several other observables)
must be obtained.
This requires a statistics over
many identically prepared
systems (some \(|x\rangle\), or \(\hat{\rho}\)).
\end{itemize}

One way out: if ``cloning'' would be possible:
%%% 07 OK
\begin{itemize}
\item make the system in the unknown state \(|x\rangle\)
interact with many other systems previously
prepared in states \(|\varphi\rangle\) -- the ``target state'' 
to obtain many copies
of the initial state,
through the usage of a ``copy machine''(CM):
\be
\begin{aligned}
|x\rangle \otimes
\underbrace{|\varphi\rangle \otimes|\varphi\rangle \cdots \otimes|\varphi\rangle}_%
{n}
&\stackrel{\text{CM}}{\longrightarrow}&
\underbrace{|x\rangle \otimes|x\rangle \otimes|x\rangle \cdots \otimes|x\rangle}_%
{n+1}\\
\underbrace{\hspace{10em}} & & \underbrace{\hspace{10em}}\\
\ket{\text{in} (x)} \quad \quad \quad & \longrightarrow & \ket{\text{out} (x)} \quad \quad \quad 
\end{aligned}
\ee
%
\item  This would allow determination of $|x\rangle$ without even 
measuring it $\rightarrow$ one could even measure the
$n$ copies and leave the original
untouched (the $n$ copies provide the
required statistics).
\item
to preserve the QM unitarity \(\rightarrow\) the copy machine
must operate unitarily:
\[
\boxed{\ket{\text{out} (x)} =  \hat{U} \ket{\text{in} (x)}}
\]
but this \emph{is impossible}.
\end{itemize}
Consider one copy:
%%% 08 OK
\be
\begin{aligned}
|\text{in}(x)\rangle
&=|x\rangle \otimes|\varphi\rangle,\langle x | x\rangle=1=\langle\varphi | \varphi\rangle\\
|\text{out}(x)\rangle
&=|x\rangle \otimes|x\rangle
\end{aligned}
\ee
This does not preserve the scalar product, because:
if you take two states to be cloned, 
\(\left|x_{1}\right\rangle\) and \(\left|x_{2}\right\rangle\):
\be
\begin{aligned}
|\text{in}(x_{1})\rangle
&=|x_{1}\rangle \otimes|\varphi\rangle \stackrel{\hat{U}}{\longrightarrow}|\text{out}(x_{1})\rangle=|x_{1}\rangle \otimes|x_{1}\rangle \\ 
|\text{in}(x_{2})\rangle
&=|x_{2}\rangle \otimes|\varphi\rangle \stackrel{\hat{U}}{\longrightarrow}|\text{out}(x_{2})\rangle=|x_{2}\rangle \otimes|x_{2}\rangle
\end{aligned}
\ee
Suppose $\bra{\varphi}\ket{\varphi}| = 1$, then
\be
\begin{aligned}
\langle\text{in} (x_{1}) | \text{in}(x_{2})\rangle
&=\langle x_{1} | x_{2}\rangle \\ 
\langle\text{out}(x_{1}) | \text{out}(x_{2})\rangle
&=(\langle x_{1} | x_{2}\rangle)^{2}
\end{aligned}
\ee
%
\emph{BUT}, in general $(\bra{x_1}\ket{x_2})^2 \neq \bra{x_1}\ket{x_2}$.
Equality happens only:
\begin{itemize}
\item \emph{either} $\bra{x_1}\ket{x_2} = 1$, but from Cauchy-Schwarz
$\left|\left\langle x_{1} | x_{2}\right\rangle\right|^{2} \leqslant \sqrt{\left\langle x_{1} | x_{1}\right\rangle\left\langle x_{2} | x_{2}\right\rangle}$
\item \emph{or} $\bra{x_1}\ket{x_2} = 0$ $\Rightarrow$ orthogonal states can be cloned.
\end{itemize}

\emph{QKD:} if one uses only the orthogonal states,
\(\ket{\uparrow}=|0\rangle\) and \(\ket{\downarrow}=|1\rangle\), not good, they
can be cloned. Need superpositions, which are not clonable.

%%% 09 OK

\emph{Example:} A generic qubit, \(|x\rangle=\lambda|0\rangle+\mu|1\rangle\):
\(|\text{in}(x)\rangle=|x\rangle \otimes|\varphi\rangle \stackrel{\hat{U}}{\longrightarrow}|\text{out}(x)\rangle=\hat{U}|\text{in}(x)\rangle\)
Evaluate this in two ways:
\begin{enumerate}
\item
\be
\begin{aligned}
|\text{out}(x)\rangle 
&=\hat{U}((\lambda|0\rangle+\mu|1\rangle) \otimes|\varphi\rangle) \\ 
&=(\lambda|0\rangle+\mu|1\rangle) \otimes (\lambda|0\rangle+\mu|1\rangle) \\ 
&=\lambda^{2}|00\rangle+\lambda \mu(|01\rangle+|10\rangle)+\mu^{2}|11\rangle 
\end{aligned}
\label{eq:g7}
\ee
%
\item
\be
\begin{aligned}
|\text{out}(x)\rangle 
&=\hat{U}((\lambda|0\rangle+\mu|1\rangle) \otimes|x\rangle) \\ 
&=\hat{U}(\lambda|0\rangle \otimes|\varphi\rangle+\mu|1\rangle \otimes|x\rangle) \\ 
&=\lambda \hat{U}(|0\rangle \otimes|\varphi\rangle)+\mu \hat{U}(|1\rangle \otimes|\varphi\rangle) \\ 
&=\lambda|00\rangle+\mu|11\rangle \end{aligned}
\label{eq:g8}
\ee
\end{enumerate}
But Eqs.~\eqref{eq:g7} and \eqref{eq:g8} cannot be equal for
generic \(\lambda\) and \(\mu \Rightarrow \hat{U}\) cannot be
unitary (linear).

How about cloning of a state in contact with
degrees of freedom of the cloning device?
$\Rightarrow$ Also not possible: Le~Bellac, French edition.

%%% 10 OK

\[
\begin{aligned}
\left|x_{1}\right\rangle,\left|x_{2}\right\rangle \in H_{1} 
&:\text { states to be cloned } \\ 
|\varphi\rangle \in H_{2}
&:\text { target state, on which we want to copy }\left|x_{1}\right\rangle \text { and }\left|x_{2}\right\rangle\\
|m\rangle \in H_3
&: \text { state of the machine (ancilla) }
\end{aligned}
\]
Cloning in $H_1 \otimes H_2 \otimes H_3$:
\[
\begin{aligned}
\left|\text{in}\left(x_{i}\right)\right\rangle
&=\left|x_{i}\right\rangle \otimes|\varphi\rangle \otimes|m\rangle, i=1,2\\
\left|\text{out}\left(x_{i}\right)\right\rangle
&=\left|x_{i}\right\rangle \otimes\left|x_{i}\right\rangle \otimes\left|m\left(x_{i}\right)\right\rangle
\end{aligned}
\]
\textit{i.e.}, the copy machine gets modified after the copy. Then,
using that $\bra{\varphi}\ket{\varphi} = 1 = \bra{m}\ket{m}$,
we again do the calculation two ways:
\begin{enumerate}
\item
\be
\begin{aligned}
\left\langle\text{out}\left(x_{1}\right) \mid \text{out}\left(x_{2}\right)\right\rangle
&=\left\langle\text{in}\left(x_{1}\right)
\left|\hat{U}^{\dagger}\hat{U}\right| 
\text{in}\left(x_{2}\right)\right\rangle \\ 
&=\left\langle x_{1} \otimes \varphi \otimes m\left|\hat{U}^{\dagger} \hat{U}\right| x_{2} \otimes \varphi \otimes m\right\rangle\\
&=\left\langle x_{1} | x_{2}\right\rangle
\end{aligned}
\ee
%
\item
\be
\begin{aligned}
\left\langle\text{out}\left(x_{1}\right) \mid \text{out}\left(x_{2}\right)\right\rangle 
&=\left\langle x_{1} \otimes x_{1} \otimes m\left(x_{1}\right) \mid x_{2} \otimes x_{2} \otimes m\left(x_{2}\right)\right\rangle \\ 
&=\left(\left\langle x_{1} \mid x_{2}\right\rangle\right)^{2}\left(\left\langle m\left(x_{1}\right)\right| m\left(x_{2}\right\rangle\right)^{2} \end{aligned}
\ee
\end{enumerate}
Again, for non-orthogonal states:
\[
1=\left|\left\langle x_{1} \mid x_{2}\right\rangle\right|\cdot\left|\left\langle m\left(x_{1}\right) \mid m\left(x_{2}\right)\right\rangle\right|
\]
and once again the Cauchy-Schwarz inequality shows that it is not possible unless
$\ket{x_2} \sim \ket{x_1}$, $\ket{m(x_2)} \sim \ket{m(x_1)}$.

%%% 11 OK

No cloning theorem guarantees that there is no
faster-than-light communication (Fig 6.1 of Le~Bellac): 
if Bob could clone his spin-1/2,
\textit{i.e.} make copies of it, he could reconstruct
his state and immediately know Alice's
axis (Alice's message), no matter how far apart
their laboratories are.


\emph{See problem in list of exercises.}

\end{document}