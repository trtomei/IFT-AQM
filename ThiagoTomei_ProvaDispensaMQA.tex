%!TEX TS-program = xelatex
%!TEX encoding = UTF-8 Unicode

\documentclass[12pt]{article}
\usepackage{geometry}                % See geometry.pdf to learn the layout options. There are lots.
\geometry{a4paper,top=2cm}
\usepackage[parfill]{parskip}    % Activate to begin paragraphs with an empty line rather than an indent
\usepackage{amsmath}
\usepackage{amssymb}
\usepackage{polyglossia}
\setdefaultlanguage{portuguese}
%\usepackage{brazil}{babel}

\title{Mecânica Quântica Avançada}
\date{}

\begin{document}
\maketitle
\vspace*{-4em}

\textbf{Exercício 1:} considere o sistema de dois níveis, os quais chamaremos\\ 
$|+\rangle = \begin{pmatrix}1\\0\end{pmatrix}$ e $|-\rangle = \begin{pmatrix}0\\1\end{pmatrix}$, regido por uma hamiltoniana que tem a expressão geral:
%
\[
\hat{H}(t) = -\frac{1}{2} \hbar\left(\begin{array}{cc}\omega_{A} & \omega_{1} e^{i \omega t} \\ \omega_{1} e^{-i \omega t} & \omega_{B}\end{array}\right)
\]

\textbf{(a):} considere primeiro o caso particular em que $\omega = 0$, \textit{i.e.} o caso estático -- a hamiltoniana não depende do tempo.
Como é o caso estático, use a seguinte parametrização:
\[
\omega_1 = \frac{2A}{\hbar},\quad
\omega_A = -\frac{2}{\hbar}(E_0 + d),\quad
\omega_B = -\frac{2}{\hbar}(E_0 - d),\quad
\]
e calcule os autovalores de energia.
Esboce o gráfico dos níveis de energia como função de $d/A$, considerando tanto os limites $d/A \gg 1$ e $d/A \ll 1$.

\textbf{(b):} considere agora o caso particular em que $\omega \neq 0$, mas $\omega_A = \omega_0$ e $\omega_B = -\omega_0$.
Considerando que a solução geral para o sistema de dois níveis pode ser dada por
\[
|\psi(t)\rangle = c_+(t)|+\rangle + c_-(t)|-\rangle
\]
com $c_\pm(t)$ coeficientes complexos arbitrários,
resolva a equação de Schrödinger para calcular esses coeficientes.
Pode ser útil definir a quantidade $\delta = \omega - \omega_0$.



\textbf{Exercício 2:} considere a descrição do espalhamento Thomson em termos de \emph{fótons}, \textit{i.e.} o processo
\[
\gamma + P \to \gamma^\prime + P^\prime
\]
onde $P$ é a partícula que está sendo atingida pelo fóton, com massa $M$, carga $Q$ e momento $\vec{p}$.
Uma parte fundamental dessa descrição é, então, a quantização do campo eletromagnético.
Nessa descrição, podemos considerar que o operador quântico correspondente ao potencial vetor $\vec{A}$ é dado por
\[
\hat{\vec{A}}(\vec{r},t) = 
\sqrt{\frac{1}{L^{3}}} \sum_{\lambda} \sqrt{\frac{2 \pi \hbar c}{k}} \vec{\epsilon}_{\lambda}\left[\hat{a}_{\lambda} e^{i\left(\vec{k} \cdot \vec{r}-\omega_{\lambda} t\right)}+\hat{a}_{\lambda}^{\dagger} e^{-i\left(\vec{k} \cdot \vec{r}-\omega_{\lambda} t\right)}\right]
\]
onde todas as ondas planas estão definidas em caixas quadradas de volume $L^3$.
O índice $\lambda$ representa conjuntamente o vetor de onda do fóton $\vec{k}$ e o seu estado de polarização $\alpha$
(e portanto $\alpha=1,2$ apenas).
O vetor $\vec{\epsilon}_{\lambda}$ é o vetor de polarização do campo (e portanto $\vec{\epsilon}_{\lambda} \cdot \vec{k} = 0$).
Então temos
\[
\lambda \equiv\{\vec{k}, \alpha\}, \quad-\lambda \equiv\{-\vec{k}, \alpha\}, \quad \mathrm{e} \quad \omega_{k} \equiv \omega_{\lambda}=\omega_{-\lambda}
\]

\textbf{(a):} considerando o referencial em que a partícula inicial está em repouso, escreva as condições de conservação de energia e momento para esse espalhamento. Use a aproximação não-relativística para a energia da partícula, mas não esqueça o termo de energia de repouso!

\textbf{(b):} o espalhamento Thomson é de baixa energia, \textit{i.e.} a energia do fóton é muito menor que a massa da partícula.
O que isto implica para este espalhamento?

\textbf{(c):} escreva os estados inicial $|i\rangle$ e final $|f\rangle$ do sistema fóton--partícula, 
em termos da função de onda da partícula e do estado de vácuo do campo eletromagnético.

\textbf{(d):} a hamiltoniana que rege o processo é:
\[
\hat{H} = \frac{\hat{\vec{p}}^{\,2}}{2M}
+ \hat{H}_{\text{rad}} 
\underbrace{- \frac{Q}{Mc} \hat{\vec{A}} \cdot \hat{\vec{p}} 
- \frac{Q^2}{2Mc^2} \hat{\vec{A}}^{\,2}}%
_{\hat{H}_{\text{int}}}
\]
e uma parte fundamental do estudo é o cálculo do elemento de matriz
\[
\langle f|\hat{H}_{\text{int}}|i\rangle
\]

Realize este cálculo.

\end{document}