%!TEX TS-program = xelatex
%!TEX encoding = UTF-8 Unicode

\documentclass[12pt]{article}
\usepackage{geometry}                % See geometry.pdf to learn the layout options. There are lots.
\geometry{a4paper,top=2cm}
\usepackage[parfill]{parskip}    % Activate to begin paragraphs with an empty line rather than an indent
\usepackage{graphicx}
\usepackage{amsmath}
\usepackage{amssymb}
\usepackage{mathtools}
\usepackage{physics}
\newcommand{\be}{\begin{equation}}
\newcommand{\ee}{\end{equation}}
\usepackage[thicklines]{cancel}
\usepackage[colorlinks=true,citecolor=blue,linkcolor=blue,urlcolor=blue]{hyperref}
\usepackage{booktabs}
\usepackage{csquotes}
\usepackage{qcircuit}
\usepackage{circledsteps}
\usepackage{nicefrac}
\usepackage{fontspec,xltxtra,xunicode}
\usepackage{xcolor}
\usepackage{simplewick}
\defaultfontfeatures{Mapping=tex-text}

\newcommand{\polv}{\ensuremath{\updownarrow}}
\newcommand{\polh}{\ensuremath{\leftrightarrow}}
\newcommand{\poldr}{\rotatebox[origin=c]{45}{\ensuremath{\leftrightarrow}}}
\newcommand{\poldl}{\rotatebox[origin=c]{-45}{\ensuremath{\leftrightarrow}}}
\newcommand{\bigzero}{\mbox{\normalfont\Large\bfseries 0}}
\newcommand{\vecrp}{\ensuremath{\vec{r}^{\,\prime}}}
\newcommand{\vecnr}{\ensuremath{\vec{\nabla}_{\!r}}}

\title{Advanced Quantum Mechanics\\Class 26}
%\author{The Author}
\date{June 15, 2023}                                           % Activate to display a given date or no date

\begin{document}
\maketitle

\setcounter{section}{5}

%%% 01 OKAY

\section{Many-particle systems}

\begin{itemize}
\item Identical particles
\item Fock space (second quantization)
\end{itemize}

\subsection{Identical particles}

Inexistency of \emph{intrinsic} properties that allow us to
identify them individually.

\begin{itemize}
\item This feature has profound consequences in QM
\item Not so much in classical mechanics: it is always possible
to follow the trajectory $(\vec{r}, \vec{p})$ of a particle in phase space;
if a label is given to a particle at $t=0$, say $\left(\vec{r}_{1}(0), \vec{p}_{1}(0)\right)$,
nothing in principle forbids us to follow it closely
until its final destination $\left(\vec{r}_{1}(t), \vec{p}_{1}(t)\right)$.
\end{itemize}

Same with two particles:
\be
\begin{aligned}
(\vec{r}_1(0), \vec{p}_1(0))\\
(\vec{r}_2(0), \vec{p}_2(0))
\end{aligned}
\xrightarrow{\text{evolution}}
\begin{aligned}
(\vec{r}_1(t), \vec{p}_1(t))\\
(\vec{r}_2(t), \vec{p}_2(t))
\end{aligned}
\ee
Exchange of labels $1 \leftrightarrow 2$, leads to a final state
that is indistinguishable from the one in (1), because
%%% 02 OKAY
the equations of motion, from a Hamiltonian $H(1,2)$
or Lagrangian $L(1,2)$, are symmetric under
$1 \leftrightarrow 2$, because $H(1,2)=H(2,1)$ (same for $L(1,2)$).
$\Rightarrow$ It \emph{must be} $H(1,2)=H(2,1)$; 
otherwise the nonsymmetry
would provide a dynamical criterion
to distinguish the two particles.

Let us analise such a situation in QM. Consider two
\emph{different} particles, $a$ and $b$, prepared in some
state $|\Psi\rangle$ at some distant time. Long after that time,
when the particles are not in interaction anymore,
they will be tested by two well-separated detectors $D_{1}$
and $D_{2}: D_{1}$ tests for energy $E_{1}$ and $D_{2}$ for energy $E_{2}$.

$\Rightarrow$ The probability of $D_{1}$ finding in $|\Psi\rangle$ particle $a$
in the state $|a\rangle=\left|E_{1}\right\rangle$ and $D_{2}$ particle $b$ in the
state $|b\rangle=\left|E_{2}\right\rangle$, that is, the probability of
finding in $|\Psi\rangle$ the state $|a \otimes b\rangle=\left|E_{1} \otimes E_{2}\right\rangle$
is given by
%%% 03 OKAY
\be
P_{\Psi \rightarrow[E_{1}, E_{2}]}=|\langle 
\overset{\substack{a\\\downarrow}}{E_{1}}\otimes 
\overset{\substack{b\\\downarrow}}{E_{2}} | \Psi\rangle|^{2}
\ee

The opposite configuration, $D_1$ tests $|b\rangle=\left|E_{1}\right\rangle$ and
$D_{2}$ tests $|a\rangle=\left|E_{2}\right\rangle$ leads to a \emph{different} probability
\be
P_{\Psi \rightarrow[E_{2}, E_{1}]}=|\langle 
\overset{\substack{a\\\downarrow}}{E_{2}}\otimes 
\overset{\substack{b\\\downarrow}}{E_{1}} | \Psi\rangle|^{2}
\ee

That these probabilities are indeed in general different
can be seen in an extreme example: $D_{1}$ cannot detect
neutrinos, only $D_{2}$, but one of the particles is a neutrino;
therefore one of the results could be zero.

Now, when the particles are identical, it makes no physical
sense to write \textit{e.g.} $\left|E_{1} \otimes E_{2}\right\rangle$, because one cannot tell
whether it is particle 1 that has energy $E_{1}$, as it could
well be that it is particle 2 that has that energy and
the state should be written as $\left|E_{2} \otimes E_{1}\right\rangle$. But they
correspond to the same physical situation: the two states
must differ by a phase $\Rightarrow$
%%% 04 OKAY
\be
\left.
\begin{aligned}
\ket{a\otimes b} &= e^{i\theta_{ab}}\ket{b \otimes a}\\
\ket{b\otimes a} &= e^{i\theta_{ba}}\ket{a \otimes b}
\end{aligned}
\right\}
e^{i(\theta_{ab} + \theta_{ba})}=1
\label{eq:g4}
\ee
Let us define two new states
\be
\begin{aligned}
\ket{a\otimes b}^\prime &= e^{-i\theta_{ab}/2}\ket{a \otimes b}\\
\ket{b\otimes a}^\prime &= e^{-i\theta_{ba}/2}\ket{b \otimes a}
\end{aligned}
\ee
Now, using Eq.~\eqref{eq:g4}, one obtains
\be
\begin{gathered}
\ket{b\otimes a}^\prime = e^{-i\theta_{ba}/2}\ket{b \otimes a} =  e^{-i\theta_{ba}/2} e^{i\theta_{ba}} \ket{a\otimes b}\\
= e^{i\theta_{ba}/2} \ket{a \otimes b} = e^{i\theta_{ba}/2}  e^{i\theta_{ab}/2} \ket{a\otimes b}^\prime 
= e^{i\left(\theta_{a b}+\theta_{b a}\right) / 2} \ket{a\otimes b}^\prime 
\end{gathered}
\ee
But, from Eq.~\eqref{eq:g4}
\[
\left[
e^{i(\theta_{ab}+\theta_{ba})/2}
\right]^2 = e^{i(\theta_{ab}+\theta_{ba})} = 1 \Rightarrow
\]
\be
e^{i(\theta_{ab}+\theta_{ba})/2} = \pm 1 \Rightarrow
\ket{b\otimes a}^\prime = \pm \ket{a\otimes b}^\prime
\ee
Indistinguibility demands that states must be
either symmetric or antisymmetric cunder
the exchange $a \leftrightarrow b$ $\Rightarrow$
%%% 05 OKAY
\be
\text{amplitudes } \bra{a\otimes b}\ket{\Psi} = \pm \bra{b\otimes a}\ket{\Psi}
\ee
More importantly: this exchange symmetry is a
characteristic of the two particles, must be valid
for any state:
\begin{itemize}
\item suppose that for one state $\ket{\Phi_1}$ we have
\be
\bra{a\otimes b}\ket{\Phi_1} = + \bra{b\otimes a}\ket{\Phi_1}
\label{eq:g9}
\ee
and for another state, $\ket{\Phi_2}$, we have
\be
\bra{a\otimes b}\ket{\Phi_2} = - \bra{b\otimes a}\ket{\Phi_2}
\label{eq:g10}
\ee
\item the linearity of QM means that $\ket{\Psi}$ can be
written as
\be
\ket{\Psi} = c_1 \ket{\Phi_1} + c_2 \ket{\Phi_2} 
\ee
which implies
\be
\bra{a\otimes b}\ket{\Psi} = c_1 \bra{a\otimes b}\ket{\Phi_1} + c_2 \bra{a\otimes b}\ket{\Phi_2} 
\ee
$\Rightarrow$ \emph{not} acceptable.
It would nevertheless be acceptable if \textit{e.g.}
$c_{1}=0 \Rightarrow$ no transitions $|\Psi\rangle \rightarrow\left|\Phi_{1}\right\rangle$,
$\left|\Phi_{1}\right\rangle$ does not belong to the $H$ of the two particles.
\end{itemize}

%%% 06 OKAY

\emph{Experimental fact:}
\begin{itemize}
\item half-integer spin particles $\rightarrow$ antisymmetric amplitudes $\Rightarrow$ Fermions.
\item integer spin particles $\rightarrow$ symmetric amplitudes $\Rightarrow$ Bosons.
\end{itemize}

\emph{Two-particles states}: convenient to use labels 1 and 2
for the particles, exchange symmetry
in the indices 1 and 2
\be
\ket{a\otimes b}_{\substack{B\\F}} = \frac{1}{\sqrt{2}}
\left[
\ket{a_1 \otimes b_2} \pm \ket{a_2 \otimes b_1} 
\right]
\label{eq:g13}
\ee
Coordinate representation (bosons):
\be
\begin{aligned}
\Psi_{a b}^{B}\left(\vec{r}_{1}, \vec{r}_{2}\right)
&=\left\langle\vec{r}_{1}, \vec{r}_{2} \mid a \otimes b\right\rangle_{B}\\
&=\frac{1}{\sqrt{2}}\left[\varphi_{a}\left(\vec{r}_{1}\right) \varphi_{b}\left(\vec{r}_{2}\right)+\varphi_{a}\left(\vec{r}_{2}\right) \varphi_{b}\left(\vec{r}_{1}\right)\right]
\end{aligned}
\label{eq:g14}
\ee
%
Fermions, need to include an extra coordinate, \emph{spin}
\be
\begin{gathered}
|\vec{r}\rangle \rightarrow|\vec{r} s\rangle,\quad|a\rangle \rightarrow| a, s m\rangle,\quad s=\pm1/2\\
\langle\vec{r} s | a, s m\rangle=\varphi_{a}(\vec{r}) \chi_{m}(s)
\end{gathered}
\ee
with
\[
\chi_{m}(s)=\delta_{m s}\left\{
\begin{pmatrix}1 \\ 0\end{pmatrix} \text{ and }
\begin{pmatrix}0 \\ 1\end{pmatrix}\right.
\]
%%% 07
\be
\begin{aligned}
&\Psi_{a m, b m^{\prime}}^{F}\left(\vec{r}_{1} s_{1}, \vec{r}_{2} s_{2}\right) = \\
&=\frac{1}{\sqrt{2}}
\left[\varphi_{a}\left(\vec{r}_{1}\right) \chi_{m}\left(s_{1}\right) \varphi_{b}\left(\vec{r}_{2}\right) \chi_{m^{\prime}}\left(s_{2}\right)\right.
\left.-\varphi_{a}\left(\vec{r}_{2}\right) \chi_{m}\left(s_{2}\right) \varphi_{b}\left(\vec{r}_{1}\right) \chi_{m^{\prime}}\left(s_{1}\right)\right]\\
&=
\begin{vmatrix}
\varphi_{a}\left(\vec{r}_{1}\right) \chi_{m}\left(s_{1}\right)  & 
\varphi_{a}\left(\vec{r}_{2}\right) \chi_{m}\left(s_{2}\right) \\
\varphi_{b}\left(\vec{r}_{1}\right) \chi_{m^{\prime}}\left(s_{1}\right)&
\varphi_{b}\left(\vec{r}_{2}\right) \chi_{m^{\prime}}\left(s_{2}\right)
\end{vmatrix}
\rightarrow\text{ Slater determinant}
\end{aligned}
\ee

In many situations one can write the amplitudes
as a product
\be
\begin{aligned}
\underbrace{\Psi_{a m, b m^{\prime}}\left(\vec{r}_{1} s_{1}, \vec{r}_{2} s_{2}\right)}%
&=
\underbrace{\Phi_{ab}\left(\vec{r}_{1}, \vec{r}_{2}\right)} & \times &%
\underbrace{\chi_{mm^\prime}\left(s_{1}, s_{2}\right)}\\
\text{antisymmetric}\quad & \quad\quad \substack{\text{symm}\\\\\text{antisymm}} 
&\substack{\times\\\text{or}\\\times}& \quad \substack{\text{antisymm}\\\\\text{symm}} 
\end{aligned}
\ee
Spin States: 2 Spin-1/2, example $S=0$, $M=0$
\be
\begin{aligned}
\left|m m^{\prime}\right\rangle_{A} 
&=\frac{1}{\sqrt{2}}
\left[\ket{+_{1} \otimes-_{2}}-
      \ket{+_{2} \otimes-_{1}}\right] \\
&=\frac{1}{\sqrt{2}}
\left[\ket{+_{1} \otimes-_{2}}-
      \ket{-_{1} \otimes+_{2}}\right]
\end{aligned}
\ee
Pauli principle: no two particles in the same state.
This generalizes to $N$ particles:
for bosons you use the permanent,
for fermions you use the determinant.

%%% 08 OKAY

\subsection{Fock space -- second quantization}

In the above discussion, the Hilbert space of $N$ identical
particles was realized as the symmetric or antisymmetric 
sectors of the $N$ one-particle Hilbert
spaces.

The number $N$ enters as a kinematic
constraint (ingredient), associated \textit{a priori}
with the system under study; that is,
we have considered:
\begin{itemize}
\item \emph{OR} the space of one particle $\rightarrow N=1$
\item \emph{OR} the space of two particles $\rightarrow N=2$
\[\vdots\]
\end{itemize}

This works well when the number of particles is
conserved; when this \emph{is not} the case (like \textit{e.g.} in open
systems, relativistic systems), it is necessary to change
the framework:
Fock-space methods, in which the number of
particles is a \emph{dynamical} variable.

%%% 09 OKAY

Although the formalism was originally developed having
in mind relativistic quantum fields, it finds natural
applicability in nonrelativistic many-body problems in
atomic, nuclear and particle physics.

In this framework, instead of using or a space of
one particle, OR the space of two particles, OR $\ldots$, one
considers simultaneously:
\begin{itemize}
\item the space of one particle $\rightarrow N=1$
\item \emph{AND} the space of two particles $\rightarrow N=2$
\item \emph{AND} the space of two particles $\rightarrow N=2$
\[\vdots\]
\end{itemize}
That is: one replaces the logic of the \emph{OR} by
the logic of the \emph{AND}.

Let us introduce the concepts in involved neglecting for
the moment spin indices and work in coordinate
space. We follow the presentation in the book of
L.~S.~Brown, \href{https://www.cambridge.org/core/books/quantum-field-theory/276D2EE40D4929469FED764BFF57186D}{Quantum Field Theory}.

%%% 10 OKAY

Recall that for two identical particles, we arrived
at Eqs.~\eqref{eq:g13} and \eqref{eq:g14}:
\be
\Psi\left(\vec{r}_{1}, \vec{r}_{2}\right)=\left\langle\vec{r}_{1}, \vec{r}_{2} \mid \Psi\right\rangle
=
\begin{cases}
+\Psi\left(\vec{r}_{2}, \vec{r}_{1}\right)=\phantom{-}\left\langle\vec{r}_{2}, \vec{r}_{1} \mid \Psi\right\rangle,\text{ bosons}\\
-\Psi\left(\vec{r}_{2}, \vec{r}_{1}\right)=-\left\langle\vec{r}_{2}, \vec{r}_{1} \mid \Psi\right\rangle,\text{ fermions}
\end{cases}
\ee
which follow from Eqs.~\eqref{eq:g9} and \eqref{eq:g10}:
\be
\left|\vec{r}_{1}, \vec{r}_{2}\right\rangle=\pm\left|\vec{r}_{2}, \vec{r}_{1}\right\rangle\to\text{ upper: Boson, lower: Fermion}
\ee
Orthonormality is given as
\be
\left\langle\vec{r}_{1}, \vec{r}_{2} \mid \vec{r}_{1}^{\prime}, \vec{r}_{2}^{\prime}\right\rangle=\delta\left(\vec{r}_{1}-\vec{r}_{1}^{\prime}\right) \delta\left(\vec{r}_{2}-\vec{r}_{2}^{\prime}\right) \pm \delta\left(\vec{r}_{1}-\vec{r}_{2}^{\prime}\right) \delta\left(\vec{r}_{2}-\vec{r}_{1}^{\prime}\right)
\ee
and completeness
\be
\int d^{3} r_{1} d^{3} r_{2}\left|\vec{r}_{1} \vec{r}_{2}\right\rangle\left\langle\vec{r}_{1} \vec{r}_{2}\right|=1
\ee
For $N$ particles, this generalizes to:
\begin{gather}
\left|\vec{r}_{1} \ldots \vec{r}_{a} \ldots \vec{r}_{b} \ldots \vec{r}_{N}\right\rangle=\pm\left|\vec{r}_{1} \ldots \vec{r}_{b} \ldots \vec{r}_{a} \ldots \vec{r}_{N}\right\rangle\\
%
\left\langle\vec{r}_{1} \ldots \vec{r}_{N} \mid \vec{r}_{1}^{\,\prime} \ldots \vec{r}_{N}^{\,\prime}\right\rangle=\sum_{p}(\pm 1)^{P} \prod_{\alpha=1}^{N} \delta\left(\vec{r}_{a}-\vec{r}_{\pi(a)}^{\,\prime}\right)
\end{gather}
where
\be
P = \begin{cases}
\begin{gathered}
\text{0 for an even permutation}\\
\text{1 for an odd permutation}\\
\end{gathered}
\end{cases}
\ee
$\pi(a)$: permutation (associated with index $a$)
and the completeness relation is
\be
\int d^{3} r_{1} \ldots d^{3} r_{N}\left|\vec{r}_{1} \ldots \vec{r}_{N}\right\rangle\left\langle\vec{r}_{1} \ldots \vec{r}_{N}\right|=1
\ee

%%% 11 OKAY

The idea is to consider all the states $\left\{|0\rangle,|\vec{r}\rangle,\left|\vec{r}_{1} \vec{r}_{2}\right\rangle \ldots\right\}$
simultaneously. Here, $|0\rangle$ is the state with zero particles

\emph{Note that}: there is neither new physics, nor new
postulates; rather, it is an extension of
of the mathematics of QM used so far.

The space of states $\left\{|0\rangle,|\vec{r}\rangle,\left|\vec{r}_{1} \vec{r}_{2}\right\rangle, \ldots\right\}$ is known
as the \emph{Fock space}. Scalar products in this space
are defined as:
\begin{align}
&\langle 0 | 0\rangle=1, 
\quad\langle 0 | \vec{r}\rangle=0\\
&\langle\vec{r} | 0\rangle=0,
\quad\left\langle\vec{r} | \vec{r}^{\,\prime}\right\rangle=\delta\left(\vec{r}-\vec{r}^{\,\prime}\right)\\
&\quad\quad\vdots\nonumber\\
&\left\langle\vec{r}_{1} \ldots \vec{r}_{N} | \vec{r}_{1} \ldots \vec{r}_{N^{\prime}}\right\rangle=\delta_{NN^{\prime}}\left\langle\vec{r}_{1} \ldots \vec{r}_{N} | \vec{r}_{1} \ldots \vec{r}_{N}\right\rangle\label{eq:g29}
\end{align}

Let us define the operator
\be
\hat{\psi}^{\dagger}(\vec{r})=\sum_{N=0}^{\infty} \frac{1}{N !} \int d^{3} r_{1} \ldots d^{3} r_{N}\left|\vec{r} \vec{r}_{1} \ldots \vec{r}_{N}\right\rangle\left\langle\vec{r}_{1} \ldots \vec{r}_{N}\right|
\label{eq:g30}
\ee
and consider its action on the states $|0\rangle,\left|\vec{r}_{1}\right\rangle, \ldots$:

%%% 12 OKAY
\be
\begin{aligned}
\hat{\psi}^{\dagger}(\vec{r})\ket{0}
&=\sum_{N=0}^{\infty} \frac{1}{N !} \int d^{3} r_{1}^\prime \ldots d^{3} r_{N}^\prime \left|\vec{r}\, \vec{r}_{1}^{\,\prime} \ldots \vec{r}_{N}^{\,\prime}\right\rangle
\underbrace{\left\langle\vec{r}_{1}^{\,\prime} \ldots \vec{r}_{N}^{\,\prime}|0\right\rangle}%
_{\text{Eq:~\eqref{eq:g29}}:\,\delta_{N0}\bra{0}\ket{0}}\\
&=\ket{\vec{r}} \to \text{1-particle state}
\end{aligned}
\ee
%
\be
\begin{aligned}
\hat{\psi}^{\dagger}(\vec{r})\ket{\vec{r}_1}
&=\sum_{N=0}^{\infty} \ldots
\underbrace{\left\langle\vec{r}_{1}^{\,\prime} \ldots \vec{r}_{N}^{\,\prime}|\vec{r}_1\right\rangle}%
_{\delta_{N1}\bra{\vec{r}_{1}^{\,\prime}}\ket{\vec{r}_{1}}}
= \int d^{3} r_{1}^\prime \ket{\vec{r}\, \vec{r}_{1}^{\,\prime}}\bra{\vec{r}_{1}^{\,\prime}}\ket{\vec{r}_{1}}\\
&= \int d^{3} r_{1}^\prime \ket{\vec{r}\, \vec{r}_{1}^{\,\prime}}\delta(\vec{r}_{1}^{\,\prime}-\vec{r}_{1})
= \ket{\vec{r}\,\vec{r}_{1}} \to \text{2-particle state}
\end{aligned}
\ee
and in general
\be
\begin{aligned}
\hat{\psi}^{\dagger}(\vec{r})\ket{\vec{r}_1\ldots\vec{r}_N}
&=\sum_{N=0}^{\infty} \frac{1}{N !} \int d^{3} r_{1}^\prime \ldots d^{3} r_{N}^\prime \left|\vec{r}\, \vec{r}_{1}^{\,\prime} \ldots \vec{r}_{N}^{\,\prime}\right\rangle\\
&\hspace{6em}\times\langle\vec{r}_{1}^{\,\prime} \ldots \vec{r}_{N}^{\,\prime}\ket{\vec{r}_1\ldots\vec{r}_N}\\%
&= \text{ (Exercise) }\ket{\vec{r}\,\vec{r}_1\ldots\vec{r}_N} \to \text{(N+1)-particle state}
\end{aligned}
\ee
so
\[
\boxed{
\begin{aligned}
\hat{\psi}^{\dagger}(\vec{r}):
&\text{ increases the number of}\\
&\text{ particles by one unit}
\end{aligned}
}\rightarrow \text{It is a \emph{creation} operator.}
\]

%% 13 OKAY
\mbox{All of the many-particle states can be constructed
from the zero-particle state $\ket{0}$:}
\begin{align}
\ket{\vec{r}} &= \hat{\psi}^\dagger(\vec{r})\ket{0}\\
\ket{\vec{r}_1,\vec{r}_2} &= \hat{\psi}^\dagger(\vec{r}_1)\ket{\vec{r}_2} = 
\hat{\psi}^\dagger(\vec{r}_1)\hat{\psi}^\dagger(\vec{r}_2)\ket{0}\\
\vdots\nonumber\\
\ket{\vec{r}_1,\ldots,\vec{r}_N} &= \hat{\psi}^\dagger(\vec{r}_1)\ldots\hat{\psi}^\dagger(\vec{r}_N)\ket{0}
\end{align}

The adjoint of $\hat{\psi}^\dagger(\vec{r})$, $\hat{\psi}(\vec{r})$ is
\be
\hat{\psi}(\vec{r})=\sum_{N=0}^{\infty} \frac{1}{N !} \int d^{3} r_{1} \ldots d^{3} r_{N}\left|\vec{r}_{1} \ldots \vec{r}_{N}\right\rangle\left\langle\vec{r}\,\vec{r}_{1} \ldots \vec{r}_{N}\right|
\label{eq:g37}
\ee
The action of $\hat{\psi}(\vec{r})$ on the $\ket{0}$, $\ket{\vec{r}_1}$, \ldots is:
\be
\hat{\psi}(\vec{r})\ket{0} = \sum_{N=0}^{\infty} \frac{1}{N !} \int d^{3} r_{1}\ldots \underbrace{\bra{\vec{r}\,\vec{r}_1\ldots\vec{r}_N}\ket{0}}_{=0} = 0
\ee
%
\be
\begin{aligned}
\hat{\psi}(\vec{r})\ket{\vec{r}_1}
& =\sum_{N=0}^{\infty} \frac{1}{N !} \int d^{3} r_{1}^\prime\ldots
\ket{\vec{r}_1^{\,\prime}\ldots\vec{r}_N^{\,\prime}}
\underbrace{\bra{\vec{r}\,\vec{r}_1^{\,\prime}\ldots\vec{r}_N^{\,\prime}}\ket{\vec{r}_1}}_{\delta_{N0}}\\
&= \ket{0}\bra{\vec{r}}\ket{\vec{r}_1} = \delta\left(\vec{r}-\vec{r}_{1}\right)\ket{0}
\end{aligned}
\ee
%%% 14 OKAY
\be
\begin{aligned}
\hat{\psi}(\vec{r})\ket{\vec{r}_1,\vec{r}_2}
& =\sum_{N=0}^{\infty} \frac{1}{N !} \int d^{3} r_{1}\ldots
\ket{\vec{r}_1^{\,\prime}\ldots\vec{r}_N^{\,\prime}}
\underbrace{\bra{\vec{r}\,\vec{r}_1^{\,\prime}\ldots\vec{r}_N^{\,\prime}}\ket{\vec{r}_1\vec{r}_2}}_{N=1\text{ only contributes}}\\
&= \int d^{3} r_{1}^\prime \ket{\vec{r}_1^{\,\prime}} \,\, \times
\underbrace{\bra{\vec{r}\,\vec{r}_1^{\,\prime}}\ket{\vec{r}_1\vec{r}_2}}%
_{\mathrlap{\delta(\vec{r}-\vec{r}_1)\delta(\vec{r}_1^{\,\prime}-\vec{r}_2) \pm \delta(\vec{r}-\vec{r}_2)\delta(\vec{r}_1^{\,\prime}-\vec{r}_1)}\hspace{5em}}\\
&= \delta(\vec{r}-\vec{r}_1)\ket{\vec{r}_2}\pm\delta(\vec{r}-\vec{r}_2)\ket{\vec{r}_1}
\end{aligned}
\label{eq:g40}
\ee
and finally
\be
\begin{aligned}
\hat{\psi}(\vec{r})
&=      \delta(\vec{r}-\vec{r}_1) \ket{\vec{r}_2\vec{r}_3\ldots\vec{r}_N}
+ (\pm) \delta(\vec{r}-\vec{r}_2) \ket{\vec{r}_2\vec{1}_3\ldots\vec{r}_N}\\
&+      \delta(\vec{r}-\vec{r}_3) \ket{\vec{r}_1\vec{1}_2\ldots\vec{r}_N} + \ldots\to
\text{(Exercise)}\\
&=\sum_{a=1}^N(\pm1)^{a+1}\delta(\vec{r}-\vec{r}_a)\ket{\vec{r}_1\ldots\vec{r}_{a-1}\vec{r}_{a+1}\ldots\vec{r}_N} 
\end{aligned}
\label{eq:g41}
\ee
\[
\boxed{
\begin{aligned}
\hat{\psi}(\vec{r}):
&\text{ decreases the number of}\\
&\text{ particles by one unit}
\end{aligned}
}
\rightarrow \text{It is an \emph{annihilation} operator.}
\]
%%% 15 OKAY
Let us now consider the action of two operators $\hat{\psi}(\vec{r})$ and 
$\hat{\psi}^{\dagger}(\vec{r})$ on an $N$-particle state. 
Start with two creation operators $\psi^\dagger\psi^\dagger$:
\be
\begin{aligned} 
\hat{\psi}^{\dagger}(\vec{r}) \hat{\psi}^{\dagger}(\vec{r}^{\,\prime})\left|\vec{r}_{1} \ldots \vec{r}_{N}\right\rangle 
&=\hat{\psi}^{\dagger}(\vec{r})\left|\vec{r}^{\,\prime} \vec{r}_{1} \ldots \vec{r}_{N}\right\rangle \\ 
&=\left|\vec{r}\,\vec{r}^{\,\prime} \vec{r}_{1} \ldots \vec{r}_{N}\right\rangle \\ 
&=\pm\left|\vec{r}^{\,\prime}\,\vec{r}\,\vec{r}_{1} \ldots \vec{r}_{N}\right\rangle \\ 
&=\pm \hat{\psi}^\dagger\left(\vec{r}^{\,\prime}\right) \hat{\psi}^\dagger(\vec{r})\left|\vec{r}_{1} \ldots \vec{r}_{N}\right\rangle 
\end{aligned}
\ee
which, in view of the completeness of the state, implies
\be
\hat{\psi}^{\dagger}(\vec{r}) \hat{\psi}^{\dagger}(\vec{r}^{\,\prime}) \mp \hat{\psi}^{\dagger}(\vec{r}^{\,\prime}) \hat{\psi}^{\dagger}(\vec{r}) = 0
\ee
so
\be
\left[\hat{\psi}^{\dagger}\left(\vec{r}\right), \hat{\psi}^{\dagger}\left(\vec{r}^{\,\prime}\right)\right]_{\mp}=0
\ee

Consider now the action of $\hat{\psi} \hat{\psi}^{\dagger}$
\be
\begin{aligned} 
\hat{\psi}(\vec{r}) \hat{\psi}^{\dagger}(\vec{r}^{\,\prime})
\left|\vec{r}_{1} \ldots \vec{r}_{N}\right\rangle 
&=\hat{\psi}(\vec{r})\left|\vec{r}^{\,\prime}\,\vec{r}_{1} \ldots \vec{r}_{N}\right\rangle 
\to\text{Eq.~\eqref{eq:g41}}\\ 
&=\delta\left(\vec{r}, \vec{r}^{\,\prime}\right)\left|\vec{r}_{1} \ldots \vec{r}_{N}\right\rangle \\ 
&+(\pm) \sum_{a=1}^{N}(\pm1)^{a+1} \delta\left(\vec{r}-\vec{r}_{a}\right)
\ket{\vec{r}^{\,\prime}\vec{r}_1\ldots\vec{r}_{a-1}\vec{r}_{a+1}\ldots\vec{r}_N}
\end{aligned}
\label{eq:g45}
\ee
%%% 16 OKAY
The action of $\hat{\psi}^{\dagger} \hat{\psi}$  is given by:
\be
\begin{aligned} 
\hat{\psi}^{\dagger}(\vec{r}^{\,\prime}) \hat{\psi}(\vec{r}) 
\left|\vec{r}_{1} \ldots \vec{r}_{N}\right\rangle 
&=\hat{\psi}^{\dagger}(\vec{r}^{\,\prime}) 
   \sum_{a=1}^{N}(\pm1)^{a+1} \delta\left(\vec{r}-\vec{r}_{a}\right)
\ket{\vec{r}_1\ldots\vec{r}_{a-1}\vec{r}_{a+1}\ldots\vec{r}_N}\\
&= \sum_{a=1}^{N}(\pm1)^{a+1} \delta\left(\vec{r}-\vec{r}_{a}\right)
\ket{\vec{r}^{\,\prime}\,\vec{r}_1\ldots\vec{r}_{a-1}\vec{r}_{a+1}\ldots\vec{r}_N}
\end{aligned}
\label{eq:g46}
\ee
Subtracting \eqref{eq:g46} from \eqref{eq:g45} for the case of Bosons and
adding those equations for the Fermion case:
\setcounter{equation}{47}
\be
\left[
\hat{\psi}(\vec{r}),\hat{\psi}^{\dagger}(\vec{r}^{\,\prime})
\right]_\mp
\ket{\vec{r}_1\ldots\vec{r}_N} = \delta(\vec{r} - \vec{r}^{\,\prime}) 
\ket{\vec{r}_1\ldots\vec{r}_N}
\ee
so
\be
\left[
\hat{\psi}(\vec{r}),\hat{\psi}^{\dagger}(\vec{r}^{\,\prime})
\right]_\mp = \delta(\vec{r} - \vec{r}^{\,\prime})
\ee
One can also easily make the calculation for $\psi\psi$. In summary:
\begin{gather}
\ket{\vec{r}_1\ldots\vec{r}_N} = \hat{\psi}^{\dagger}(\vec{r}_1)\ldots\hat{\psi}^{\dagger}(\vec{r}_N) \ket{0}\label{eq:g50}\\
\hat{\psi}(\vec{r})\ket{0} = 0\label{eq:g51}\\
\left[
\hat{\psi}(\vec{r}),\hat{\psi}(\vec{r}^{\,\prime})
\right]_\mp 
= 0 = 
\left[
\hat{\psi}^{\dagger}(\vec{r}),\hat{\psi}^{\dagger}(\vec{r}^{\,\prime})
\right]_\mp\label{eq:g52}\\
\left[
\hat{\psi}(\vec{r}),\hat{\psi}^{\dagger}(\vec{r}^{\,\prime})
\right]_\mp = \delta(\vec{r} - \vec{r}^{\,\prime})\label{eq:g53}
\end{gather}
%%% 17
\setcounter{equation}{54}
A traditional alternative notation for $[\hat{A},\hat{B}]_\pm$:
\be
\begin{aligned}
[\hat{A} \hat{B}]_- \equiv \, [ \,\hat{A}, \hat{B} \,]   &=\hat{A} \hat{B}-\hat{B} \hat{A}, \text{ the usual commutator}\\
[\hat{A} \hat{B}]_+ \equiv \{\hat{A}, \hat{B}\} &=\hat{A} \hat{B}-\hat{B} \hat{A}, \text{ the anti-commutator}
\end{aligned}
\ee
From now on, we will not need anymore to use
the definition of the operators $\hat{\psi}^{\dagger}$ and $\hat{\psi}$, given by
Eqs.~\eqref{eq:g30} and \eqref{eq:g37}: all that is needed is encoded
by Eqs.~\eqref{eq:g50}--\eqref{eq:g53}.
$\Rightarrow$
Properties of states (and observables, to
be discussed shortly ahead) follow from the
(anti) commutation relations of $\hat{\psi}$ and $\hat{\psi}^\dagger$.

\emph{Example:}
\be
\begin{gathered}
\hat{\psi}(\vec{r}) \ket{\vec{r}_1\vec{r}_2} = 
\hat{\psi}(\vec{r}) \hat{\psi}^\dagger(\vec{r}_1) \hat{\psi}^\dagger(\vec{r}_2) \ket{0}\\
= \left(
[\hat{\psi}(\vec{r}),\hat{\psi}^\dagger(\vec{r}_1)] + 
(\pm)\hat{\psi}^\dagger(\vec{r}_1)\hat{\psi}(\vec{r})
\right)\hat{\psi}^\dagger(\vec{r}_2) \ket{0}\\
= \delta(\vec{r}-\vec{r}_1)\ket{\vec{r}_2} + (\pm)\delta(\vec{r}-\vec{r}_2)\ket{\vec{r}_1}
+ \hat{\psi}^\dagger(\vec{r}_1)\hat{\psi}^\dagger(\vec{r}_2)\underbrace{\hat{\psi}(\vec{r})\ket{0}}_{=0} \Rightarrow\\
\hat{\psi}(\vec{r}) \ket{\vec{r}_1\vec{r}_2} = 
\delta(\vec{r}-\vec{r}_1)\ket{\vec{r}_2} \pm \delta(\vec{r}-\vec{r}_2)\ket{\vec{r}_1}
\end{gathered}
\ee
$\to$ same result as that obtained in Eq.~\eqref{eq:g40}
by using the definition of $\hat{\psi}$.











\end{document}



































