% XeLaTeX can use any Mac OS X font. See the setromanfont command below.
% Input to XeLaTeX is full Unicode, so Unicode characters can be typed directly into the source.

% The next lines tell TeXShop to typeset with xelatex, and to open and save the source with Unicode encoding.

%!TEX TS-program = xelatex
%!TEX encoding = UTF-8 Unicode

\documentclass[12pt]{article}
\usepackage{geometry}                % See geometry.pdf to learn the layout options. There are lots.
\geometry{letterpaper}                   % ... or a4paper or a5paper or ... 
%\geometry{landscape}                % Activate for for rotated page geometry
%\usepackage[parfill]{parskip}    % Activate to begin paragraphs with an empty line rather than an indent
\usepackage{graphicx}
\usepackage{amssymb}

% Will Robertson's fontspec.sty can be used to simplify font choices.
% To experiment, open /Applications/Font Book to examine the fonts provided on Mac OS X,
% and change "Hoefler Text" to any of these choices.

\usepackage{fontspec,xltxtra,xunicode}
\defaultfontfeatures{Mapping=tex-text}
%\setromanfont[Mapping=tex-text]{Computer Modern}
%\setsansfont[Scale=MatchLowercase,Mapping=tex-text]{Gill Sans}
%\setmonofont[Scale=MatchLowercase]{Andale Mono}

\title{Brief Article}
\author{The Author}
%\date{}                                           % Activate to display a given date or no date

\begin{document}
\maketitle

{Postulates of quantum physics}

Here: generalization of previous results for
the photon polarization and spin $\frac{1}{2}$
-- give the general conceptual framework
but does not provide the tools for
\emph{solving specific problems}.

This requires modelling (simplifications, approximations),
heuristic arguments \ldots \emph{not} derivable within the general framework.

The discussion will be divided into three parts:
\begin{enumerate}
\item State vectors and physical properties
\item Time evolution
\item Approximations and modelling (short discussion)
\end{enumerate}

\section{State vectors and physical properties}

\subsection{The superposition principle}

\emph{Postulate I:} the space of states

$|\varphi\rangle$: state vector, element of a complex Hilbert
space $H$ (space of states). It defines completely the properties of a quantum system.

Convenience: choose $\|\varphi \|^{2}=\langle\varphi \mid \varphi\rangle=1$

In general, description in terms of vectors in a $H$ implies the superposition principle (linearity): 
if $|\varphi\rangle$, $|\chi\rangle$ in $H$, the vector $|\psi\rangle$:
%
\begin{equation}
|\psi\rangle=\frac{\lambda|\varphi\rangle+\mu|\chi\rangle}{\| \lambda|\varphi\rangle+\mu|\chi\rangle \|^{2}}
\end{equation}
%
for $\lambda$ and $\mu$ complex numbers, is also in $H$ and represents a physical state.

\emph{Postulate II:} probability amplitudes and probabilities

\end{document}  