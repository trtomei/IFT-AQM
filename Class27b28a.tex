%!TEX TS-program = xelatex
%!TEX encoding = UTF-8 Unicode

\documentclass[12pt]{article}
\usepackage{geometry}                % See geometry.pdf to learn the layout options. There are lots.
\geometry{a4paper,top=2cm}
\usepackage[parfill]{parskip}    % Activate to begin paragraphs with an empty line rather than an indent
\usepackage{graphicx}
\usepackage{amsmath}
\usepackage{amssymb}
\usepackage{mathtools}
\usepackage{physics}
\newcommand{\be}{\begin{equation}}
\newcommand{\ee}{\end{equation}}
\usepackage[thicklines]{cancel}
\usepackage[colorlinks=true,citecolor=blue,linkcolor=blue,urlcolor=blue]{hyperref}
\usepackage{booktabs}
\usepackage{csquotes}
\usepackage{qcircuit}
\usepackage{circledsteps}
\usepackage{nicefrac}
\usepackage{fontspec,xltxtra,xunicode}
\usepackage{xcolor}
\usepackage{simplewick}
\defaultfontfeatures{Mapping=tex-text}

\newcommand{\polv}{\ensuremath{\updownarrow}}
\newcommand{\polh}{\ensuremath{\leftrightarrow}}
\newcommand{\poldr}{\rotatebox[origin=c]{45}{\ensuremath{\leftrightarrow}}}
\newcommand{\poldl}{\rotatebox[origin=c]{-45}{\ensuremath{\leftrightarrow}}}
\newcommand{\bigzero}{\mbox{\normalfont\Large\bfseries 0}}
\newcommand{\vecrp}{\ensuremath{\vec{r}^{\,\prime}}}
\newcommand{\vecnr}{\ensuremath{\vec{\nabla}_{\!r}}}

\title{Advanced Quantum Mechanics\\Class 27 (b) + 28 (a)}
%\author{The Author}
\date{June 22, 2023}                                           % Activate to display a given date or no date

\begin{document}
\maketitle

\setcounter{section}{6}
\setcounter{subsection}{5}

%%% 14 OKAY

\subsection{Heisenberg representation}
\[
\boxed{
i \hbar \frac{\partial \hat{O}(t)}{\partial t}=[\hat{O}(t), \hat{H}]
}
\]

The ``equation of motion'' of the field operator $\hat{\psi}(\vec{r}, t)$ is:
\setcounter{equation}{88}
\be
i \hbar \frac{\partial}{\partial t} \hat{\psi}(\vec{r}, t)=[\hat{\psi}(\vec{r}, t), \hat{H}]
\ee
the solution is ($\hat{V}_2$ time-independent)
\be
\hat{\psi}(\vec{r}, t)=e^{i / \hbar \hat{H} t} \hat{\psi}(\vec{r}) e^{-i / \hbar \hat{H} t}
\ee
%
This solution then leads us to find out that the \emph{equal-time} commutation relations are given by:
\be
\left[\hat{\psi}^\dagger(\vec{r}, t), \hat{\psi}^\dagger(\vecrp, t)\right]_{\mp} = 0 =
\left[\hat{\psi}(\vec{r}, t), \hat{\psi}(\vecrp, t)\right]_{\mp}
\ee
\be
\left[\hat{\psi}(\vec{r}, t), \hat{\psi}^\dagger\left(\vecrp, t\right)\right]_{\mp}=\delta\left(\vec{r}-\vecrp\right)
\ee
Notice that for \emph{different times}, in general, 
one doesn't really have simple commutation relations.
It is possible to find them, but it entails
\emph{solving the problem in question}.
This makes sense -- different-time commutation relations are related to
``creating a particle and destroying it later'', for instance;
so they are related to the \emph{dynamics} of the problem.

Since $\hat{H}\ket{0}=0$ $\Rightarrow$ is time-independent.
Let us now obtain the Schrödinger equation. For any
time $t$, we have that:
\be
\hat{\psi}^\dagger(\vec{r}, t)|0\rangle=
e^{i / \hbar \hat{H} t} \hat{\psi}^\dagger(\vec{r})
\underbrace{e^{-i / \hbar \hat{H} t} | 0\rangle}_{\ket{0}}  =
e^{i / \hbar \hat{H} t} \hat{\psi}^\dagger(\vec{r})| 0\rangle = 
e^{i / \hbar \hat{H} t}|\vec{r}\rangle=|\vec{r}, t\rangle_{H}
\ee
(we are using the subscript $H$ to remind ourselves that these are kets in the Heisenberg representation)
Clearly:
\be
\hat{\psi}^{\dagger}\left(\vec{r}_{1}, t\right) \ldots 
\hat{\psi}^{\dagger}\left(\vec{r}_{N}, t\right)|0\rangle=
\left|\vec{r}_{1} \ldots \vec{r}_{N}, t\right\rangle_H
\ee
%%% 15 OKAY
Acting with $i\hbar \partial/\partial t$ on this last equation, we obtain:
\be
\begin{gathered}
i\hbar \frac{\partial}{\partial t} \ket{\vec{r}_{1} \ldots \vec{r}_{N}; t}_H = 
i\hbar \frac{\partial}{\partial t}
\hat{\psi}^\dagger(\vec{r}_1, t)\ldots\hat{\psi}^\dagger(\vec{r}_N, t)\ket{0}\\
= i\hbar \frac{\partial}{\partial t}
\left(
e^{i / \hbar \hat{H} t} \hat{\psi}^\dagger(\vec{r}_1) 
  \cancel{e^{-i / \hbar \hat{H} t}} \cancel{e^{i / \hbar \hat{H} t}} \hat{\psi}^\dagger(\vec{r}_2)\ldots\psi^\dagger(\vec{r}_N)\ket{0}\right)\\
= i\hbar \frac{\partial}{\partial t} 
e^{i / \hbar \hat{H} t} 
\hat{\psi}^\dagger(\vec{r}_1)\ldots\hat{\psi}^\dagger(\vec{r}_N)\ket{0}\\
=-\hat{H} e^{i / \hbar \hat{H} t} \hat{\psi}^\dagger(\vec{r}_1)\ldots\hat{\psi}^\dagger(\vec{r}_N)\ket{0} = 
-\hat{H} \ket{\vec{r}_1\ldots\vec{r}_N;t}_H \Rightarrow\\
i\hbar \frac{\partial}{\partial t} \ket{\vec{r}_{1} \ldots \vec{r}_{N}; t}_H = -\hat{H} \ket{\vec{r}_1\ldots\vec{r}_N;t}_H
\end{gathered}
\ee
From here, one clearly gets the Schrödinger equation
\be
\begin{aligned}
i \hbar \frac{\partial}{\partial t}\left\langle\vec{r}_{1} \ldots \vec{r}_{N} ; t | \Psi\right\rangle
&=\left\langle\vec{r}_{1} \ldots \vec{r}_{N} ; t|H| \Psi\right\rangle\Rightarrow\\
i \hbar \frac{\partial}{\partial t} \Psi(\vec{r}_{1} \ldots \vec{r}_{N} ; t)
&=H(\vec{r}_{1} \ldots \vec{r}_{N}) \Psi(\vec{r}_{1} \ldots \vec{r}_{N} ; t)
\end{aligned} 
\ee

Because (anti) commutation relations involve the
singular Dirac deltas, it is convenient to change basis
$\rightarrow$ discrete basis: system in a box or self-bound systems
naturally provide discrete basis.
\be
\{\phi_k(\vec{r})\}:\text{ basis, $k$ discrete quantum numbers}
\ee
($k$ may be energy, orbital angular momentum, \ldots)

%%% 16 OKAY
Orthonormality and completeness:
\begin{gather}
\int d^{3} r \phi_{k}^{*}(\vec{r}) \phi_{k^{\prime}}(\vec{r})=\delta_{k k^{\prime}}
\label{eq:g98}\\
\sum_{k} \phi_{k}(\vec{r}) \phi_{k}^{*}\left(\vec{r}^{\,\prime}\right)=\delta\left(\vec{r}-\vec{r}^{\,\prime}\right)
\label{eq:g99}
\end{gather}
Use these to expand $\hat{\psi}(\vec{r})$ and $\hat{\psi}^\dagger(\vec{r})$ in terms of
new operators $\hat{a}_{k}$ and $\hat{a}^\dagger_{k}$.
\emph{Important:} these operators \emph{DO NOT} depend on 
the coordinate ($\vec{r}$) anymore, 
but they depend on the discrete index $k$:
\begin{align}
\hat{\psi}(\vec{r})&=\sum_{k} \phi_{k}(\vec{r}) \hat{a}_{k}\\
\hat{\psi}^\dagger(\vec{r})&=\sum_{k} \phi_{k}^*(\vec{r}) \hat{a}^\dagger_{k}
\end{align}
%
These can be inverted by making use of Eq.~\eqref{eq:g98}:
\[
\int d^{3} r \phi_{k^\prime}^{*}(\vec{r}) \hat{\psi}(\vec{r})=\sum_{k} \hat{a}_{k}  
\underbrace{\int d^{3} r\phi^*_{k^\prime}(\vec{r}) \phi_{k}(\vec{r})}_{\delta_{k^\prime k}}
\]
which leads to
\be
\hat{a}_{k}=\int d^{3} r \phi_{k}^{*}(\vec{r}) \hat{\psi}(\vec{r})
\ee
Similarly:
\be
\hat{a}^\dagger_{k}=\int d^{3} r \phi_{k}(\vec{r}) \hat{\psi}^{\dagger}(\vec{r})
\ee
Since $\psi(\vec{r})$ destroys the vacuum:
\be
\hat{\psi}(\vec{r})|0\rangle=0 \quad \longrightarrow \quad \hat{a}_{k}|0\rangle=0
\label{eq:g104}
\ee
%%% 17 OKAY
From the (anti) commutation relations of $\hat{\psi}$ and $\hat{\psi}^\dagger$,
one obtains very easily:
\be
\Big[\hat{a}_{k}, \hat{a}_{k^{\prime}}\Big]_{\mp}=0=\left[\hat{a}_{k,}^{\dagger}, \hat{a}_{k^{\prime}}^{\dagger}\right]_{\mp}
\label{eq:g105}
\ee
and
\be
\left[\hat{a}_{k}, \hat{a}_{k^{\prime}}^{\dagger}\right]_{\mp}=\delta_{k k^{\prime}}
\label{eq:g106}
\ee
In a similar fashion, but now using the completeness
of the $\phi_{k}(\vec{r})$, Eq.~\eqref{eq:g99}, one obtains:
\be
\hat{N}=\int d^{3}r \hat{\psi}^\dagger(\vec{r}) \hat{\psi}(\vec{r})=\sum_{k} \hat{N}_{k}
\ee
where
\be
\hat{N_{k}}=\hat{a}_{k}^{\dagger} \hat{a}_{k}
\ee
\emph{Note that}: The algebra of the $\hat{a}$ and $a^{\dagger}$ operators
in Eqs.~\eqref{eq:g105} and \eqref{eq:g106} is the same of the ladder
operators of the harmonic oscillator, but one
oscillator for each $k$
$\Rightarrow$
one can then use all we know from
the harmonic oscillator!

%%% 18 OKAY
Let $\ket{N_{k}}$ be the eigenvector of the number operator
$\hat{N}_{k}$. From Eq.~\eqref{eq:g104}, $a_{k}\ket{0}=0$, and from what we
know from the harmonic oscillator:
\be
\hat{N}_{k}\left|N_{k}\right\rangle=N_{k}\left|N_{k}\right\rangle, \quad N_{k}=0,1,2, \ldots
\ee
and
\be
\begin{aligned}
\hat{a}_{k}\left|N_{k}\right\rangle          &=\sqrt{N_{k}}\left|N_{k}-1\right\rangle\\
\hat{a}_{k}^{\dagger}\left|N_{k}\right\rangle&=\sqrt{N_{k}+1}\left|N_{k}+1\right\rangle
\end{aligned}
\ee
\emph{Important}: these relations are valid for both
bosons and \emph{fermions} (Exercise: check this)
\emph{but}, for the case of fermions, because of
\be
\left\{\hat{a}_{k}^{\dagger}, \hat{a}_{k}^{\dagger}\right\} = 0 \,\therefore\left(a_{k}^{\dagger}\right)^{2}=0
\ee
one has
\be
\ket{k} = a_k^\dagger\ket{0} \,\therefore a_k^\dagger\ket{k} = 0
\ee
Again, the Pauli principle
-- no two particles in a given state $k$.

%%% 19 OKAY

\subsection{Inclusion of spin}

Extra coordinate $s$: $\hat{\psi}(\vec{r}) \rightarrow \hat{\psi}_{m}(\vec{r})$, and $m$ refers to a projection of $\vec{s}$ on some axis ($\equiv \hat{z}$)
\[
\hat{\psi}_{m}(\vec{r}) \ket{0} = \ket{\vec{r},sm}
\]
and, in the general case, $s \to m$.

\setcounter{equation}{113}
(Anti) commutation relations:
\begin{gather}
 [\hat{\psi}_{m}(\vec{r}), \hat{\psi}_{m^\prime}(\vecrp)]_\mp = 0 
=[\hat{\psi}_{m}^\dagger(\vec{r}), \hat{\psi}_{m^\prime}^\dagger(\vecrp)]_\mp\\
 [\hat{\psi}_{m}(\vec{r}), \hat{\psi}_{m^\prime}^\dagger(\vecrp)]_\mp =
 \delta_{mm^\prime}\delta(\vec{r}-\vecrp)
\end{gather}

\emph{Observables}

\emph{Density:}
\be
\hat{\rho}(\vec{r}) = \sum_m \hat{\rho}_m(\vec{r}),\quad
\hat{\rho}_m(\vec{r}) = \hat{\psi}_{m}^\dagger(\vec{r})\hat{\psi}_{m}(\vec{r})
\ee
from which one obtains the number operator
\be
\hat{N} = \sum_m \hat{N}_m,\quad
\hat{N}_m = \int d^3r \hat{\psi}_{m}^\dagger(\vec{r})\hat{\psi}_{m}(\vec{r})
\ee

\emph{Spin operator:}

Recall that in the first-quantization formulation, the 
spin of $N$ particles is the vector sum
\be
\hat{\vec{S}} = \sum_{\alpha=1}^N \hat{\vec{S}}_\alpha 
\ee
%%% 20 OKAY
In Fock space, use the sandwich rule: the matrix element
of $\vec{S}$ of first quantization is sandwiched between $\hat{\psi}^\dagger$ and $\hat{\psi}$:
\be
\hat{\vec{S}}=\sum_{mm^\prime} \int d^3 r \hat{\psi}_{m^{\prime}}^{\dagger}(\vec{r}) \vec{S}_{m^{\prime} m} \hat{\psi}_{m}(\vec{r})
\ee
where we use already that $\chi_m(s) = \delta_{ms}$ and
\be
\vec{S}_{m^\prime m} = \bra{sm^\prime}\vec{S}\ket{sm}
\ee
The $\hat{H}_2$ Hamiltonian, %Eq.~\eqref{eq:g83}, 
Eq.~(83), 
for spin-independent
$V_1$ and $V_2$, that is
\be
\begin{gathered}
\left\langle m^{\prime}\left|V_{1}\right| m\right\rangle \sim \delta_{m^{\prime} m}\\
\left\langle m_{1}^{\prime} m_{2}^{\prime}\left|V_{2}\right| m_{1} m_{2}\right\rangle \sim \delta_{m_{1}^{\prime} m_{1}} \delta_{m_{2}^{\prime} m_{2}}
\end{gathered}
\ee
is given by
\be
\begin{gathered}
\hat{H}_2 = \sum_{m} \int d^{3} r \hat{\psi}_{m}^{\dagger}(\vec{r})\left[-\frac{\hbar^{2}}{2 M} \vecnr^{2}+V_{1}(\vec{r})\right] \hat{\psi}_{m}(\vec{r})\\
+ \sum_{m,m^\prime}
\hat{\psi}_{m}^{\dagger}(\vec{r}) \hat{\psi}_{m^\prime}^{\dagger}(\vecrp)
V_2(\vec{r},\vecrp)
\hat{\psi}_{m^\prime}(\vecrp)\hat{\psi}_{m}(\vec{r})
\end{gathered}
\ee

Consider now a spin-spin interaction $V_{SS}\sim\vec{S}^{(1)} \cdot \vec{S}^{(2)}$ which in
first-quantization has matrix elements
\be
\begin{gathered}
\bra{m_1^\prime m_2^\prime}\vec{S}^{(1)} \cdot \vec{S}^{(2)}\ket{m_1 m_2}
= \vec{S}_{m_1^\prime m_1} \cdot \vec{S}_{m_2^\prime m_2}\\
= \sum_{i=1}^3 \left(S_i\right)_{m_1^\prime m_1}\left(S_i\right)_{m_2^\prime m_2}
\rightarrow \left(S_i\right)_{m_1^\prime m_1}\left(S_i\right)_{m_2^\prime m_2}\\
\text{(sum over repeated indices)}
\end{gathered}
\ee
%%% 21
according to the sandwich rule is given by
\be
\begin{gathered}
\hat{V}_{SS} =\frac{1}{2} \sum_{m_1m_1^\prime}\sum_{m_2m_2^\prime} \int d^3r\,d^3r^\prime
\hat{\psi}_{m_1^\prime}^{\dagger}(\vec{r}) \hat{\psi}_{m_2^\prime}^{\dagger}(\vecrp)
V_2(\vec{r},\vecrp)\\
\times 
\left(\vec{S}_{m_1^\prime m_1} \cdot \vec{S}_{m_2^\prime m_2}\right)
\hat{\psi}_{m_2}(\vecrp)\hat{\psi}_{m_1}(\vec{r})
\end{gathered}
\ee

For generic $V_1$ and $V_2$ spin-dependent potential,
the Hamiltonian is given as

\be
\begin{gathered}
\hat{H}_2 = \sum_{mm^\prime} \int d^3r
\hat{\psi}_{m^{\prime}}^{\dagger}(\vec{r})
\left(
-\delta_{mm^\prime}\frac{\hbar^2}{2M} + 
[V_1(\vec{r},s)]_{m^\prime m}
\right)
\hat{\psi}_{m}(\vec{r})\\
+\frac{1}{2} \sum_{m_1m_1^\prime}\sum_{m_2m_2^\prime} \int d^3r\,d^3r^\prime
\hat{\psi}_{m_1^\prime}^{\dagger}(\vec{r}) \hat{\psi}_{m_2^\prime}^{\dagger}(\vecrp)
\times\\
[V_2(\vec{r},\vecrp,s)]_{m_1m_1^\prime m_2m_2^\prime}
\times \hat{\psi}_{m_2}(\vecrp)\hat{\psi}_{m_1}(\vec{r})
\end{gathered}
\label{eq:g125}
\ee

\emph{Example:} two-body tensor force
\be
S_{12}=\frac{1}{\left|\vec{r}_{1}-\vec{r}_{2}\right|^{2}} 
\left(\vec{S}^{(1)} \cdot\left(\vec{r}_{1}-\vec{r}_{2}\right)\right)
\left(\vec{S}^{(2)} \cdot\left(\vec{r}_{1}-\vec{r}_{2}\right)\right)
-\frac{1}{3} \vec{S}^{(1)} \cdot \vec{S}^{(2)}
\ee
Matrix elements in first-quantized notations, defining
the relative distance as $\vec{z} = \vec{r}_1 - \vec{r}_2$:
\be
\bra{m_1^\prime m_2^\prime} S_{12} \ket{m_1m_2} =
\frac{S_i^{(1)} z_i\,S_j^{(2)} z_j}{z^2} - \frac{1}{3} [S_i^{(1)}]_{m_1^\prime m_1} [S_i^{(2)}]_{m_2^\prime m_2}
\ee
%%% 22 OKAY
In Fock space, this reads
\be
\begin{gathered}
\hat{V}_T =
\frac{1}{2} \sum_{m_1m_1^\prime}\sum_{m_2m_2^\prime} \int d^3r\,d^3r^\prime
\hat{\psi}_{m_1^\prime}^{\dagger}(\vec{r}) \hat{\psi}_{m_2^\prime}^{\dagger}(\vecrp)\times\\
\bigg[
\frac{1}{|\vec{r}-\vecrp|^2} (S_i)_{m_1^\prime m_1} (S_j)_{m_2^\prime m_2}
(\vec{r}-\vecrp)^i (\vec{r}-\vecrp)^j\\
- \frac{1}{3} (S_i)_{m_1^\prime m_1} (S_i)_{m_2^\prime m_2}
\bigg]
\hat{\psi}_{m_2}(\vecrp)\hat{\psi}_{m_1}(\vec{r})
\end{gathered}
\ee

%%% Adiantando um pouco da Aula 28 aqui:
%%% 01 OKAY

\emph{Change of basis}: $\hat{\psi},\hat{\psi}^\dagger \to \hat{a},\hat{a}^\dagger$
\be
\begin{aligned}
\hat{\psi}_{m}(\vec{r})=\sum_{k} \phi_{k m}(\vec{r}) \hat{a}_{k m} \\ 
\hat{\psi}^\dagger_{m}(\vec{r})=\sum_{k} \phi_{k m}^{*}(\vec{r}) \hat{a}^\dagger_{k m}\end{aligned}
\\
\quad\left\{
\begin{aligned}
&k\text{ may be a wave vector }\vec{k}\\
&k\text{ may be } n\to\text{ energy}
\end{aligned}
\right.
\label{eq:g129}
\ee
with orthogonality and completeness given by
\begin{align}
\int d^3 r\, \phi_{k, m}^{*}(\vec{r}) \phi_{k^{\prime} m^{\prime}}(\vec{r})&=\delta_{m m^{\prime}} \delta_{k, k^{\prime}}\\
\sum_{m} \sum_{k} \phi_{k m}(\vec{r}) \phi_{k m}^{*}\left(\vecrp\right)&=\delta\left(\vec{r}-\vecrp\right)
\end{align}
The inverse relations are using the orthogonality relation.

%%% 02 OKAY
The $\phi_{k m}(\vec{r})$ can sometimes be written as
\be
\phi_{km}(\vec{r})=\varphi_{k}(\vec{r}) \chi_{m}(s) 
\ee
where $s$ is the spin (not necessarily $\nicefrac{1}{2}$).
Suppose one chooses the $\phi_{k m}(\vec{r})$ to be the eigenstates
of the single-particle Hamiltonian $\hat{H}_{1}$, \textit{i.e.}
\be
\left[-\frac{\hbar^{2}}{2 M} \nabla_{\!r}^{2}+V_{1}(\vec{r})\right] \phi_{k m}(\vec{r})=\varepsilon_{k} \phi_{k m}(\vec{r})
\ee
where we consider $V_1(\vec{r})$ spin-independent, for simplicity.
Then, using Eqs.~\eqref{eq:g129} into the first term of Eq.~\eqref{eq:g125}, one obtains
\be
\begin{gathered}
\hat{H}_1 = \sum_{mm^\prime} \int d^3r\, \sum_{kk^\prime}
\phi_{k^\prime m^\prime}^{*}(\vec{r})
\left[
-\frac{\hbar^{2}}{2 M} \delta_{m^{\prime} m} \vecnr^{2}+V_{1}(\vec{r}) \delta_{m^{\prime} m}
\right]
\phi_{km}(\vec{r}) \hat{a}^\dagger_{k^\prime m^\prime} \hat{a}_{k m}\\
%
\sum_{m} \int d^3r\, \sum_{kk^\prime}
\phi_{k^\prime m}^{*}(\vec{r})
\underbrace{\left[
-\frac{\hbar^{2}}{2 M} \vecnr^{2}+V_{1}(\vec{r}) \phi_{km}(\vec{r})
\right]}%
_{\varepsilon_k \phi_{km}(\vec{r})}
\hat{a}^\dagger_{k^\prime m} \hat{a}_{k m}\\
%
\sum_{m} \sum_{kk^\prime} \varepsilon_k 
\underbrace{\int d^3r\,\phi_{k^\prime m}^{*}(\vec{r})\phi_{km}(\vec{r})}%
_{\delta_{kk^\prime}}
\hat{a}^\dagger_{k^\prime m} \hat{a}_{k m}
\end{gathered}
\ee
so finally
\be
\hat{H}_{1}=
\sum_{m} \sum_{k} \varepsilon_{k} \hat{a}_{k m}^{\dagger} \hat{a}_{k m}=\sum_{m} \sum_{k} \varepsilon_{k} \hat{N}_{km}
\ee
where we define $\hat{N}_{km} \equiv \hat{a}_{k m}^{\dagger} \hat{a}_{k m}$

%%% 03 OKAY
Let us express the two-body interaction in terms 
of the $\hat{a}$ and $\hat{a}^\dagger$:
\be
\begin{gathered}
\hat{V}_2 = \nicefrac{1}{2}
\sum_{m_1m_1^\prime}\sum_{m_2m_2^\prime} 
\int d^3r\,d^3r^\prime
\sum_{k_1k_1^\prime}\sum_{k_2k_2^\prime}
\phi^*_{k_1^\prime m_1^\prime}(\vec{r})
\phi^*_{k_2^\prime m_2^\prime}(\vecrp)\\
\times
[V_2(\vec{r},\vecrp)]_{m_1^\prime m_2^\prime, m_1 m_2}
\phi_{k_2 m_2}(\vecrp)
\phi_{k_1 m_1}(\vec{r})
\times 
\hat{a}_{k_1^\prime m_1^\prime}^{\dagger} \hat{a}_{k_2^\prime m_2^\prime}^{\dagger} 
\hat{a}_{k_2 m_2} \hat{a}_{k_1 m_1}\\
= \frac{1}{2}
\sum_{m_1m_1^\prime}\sum_{m_2m_2^\prime} 
\sum_{k_1k_1^\prime}\sum_{k_2k_2^\prime}
V_2(k_1^\prime m_1^\prime, k_2^\prime m_2^\prime, k_1 m_1, k_2 m_2)\\
\times
\hat{a}_{k_1^\prime m_1^\prime}^{\dagger} \hat{a}_{k_2^\prime m_2^\prime}^{\dagger}
\hat{a}_{k_2 m_2} \hat{a}_{k_1 m_1}
\end{gathered}
\ee
where 
\be
\begin{gathered}
V_2(k_1^\prime m_1^\prime, k_2^\prime m_2^\prime, k_1 m_1, k_2 m_2)
= 
\int d^3r\,d^3r^\prime
\phi^*_{k_1^\prime m_1^\prime}(\vec{r})
\phi^*_{k_2^\prime m_2^\prime}(\vecrp)\times\\
[V_2(\vec{r},\vecrp)]_{m_1^\prime m_2^\prime, m_1 m_2}
\phi_{k_2 m_2}(\vecrp)
\phi_{k_1 m_1}(\vec{r})
\end{gathered}
\ee

\end{document}
