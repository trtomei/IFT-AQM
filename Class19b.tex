%!TEX TS-program = xelatex
%!TEX encoding = UTF-8 Unicode

\documentclass[12pt]{article}
\usepackage{geometry}                % See geometry.pdf to learn the layout options. There are lots.
\geometry{a4paper,top=2cm}
\usepackage[parfill]{parskip}    % Activate to begin paragraphs with an empty line rather than an indent
\usepackage{graphicx}
\usepackage{amsmath}
\usepackage{amssymb}
\usepackage{mathtools}
\usepackage{physics}
\newcommand{\be}{\begin{equation}}
\newcommand{\ee}{\end{equation}}
\usepackage[thicklines]{cancel}
\usepackage[colorlinks=true,citecolor=blue,linkcolor=blue,urlcolor=blue]{hyperref}
\usepackage{booktabs}
\usepackage{csquotes}
\usepackage{qcircuit}
\usepackage{circledsteps}
\usepackage{nicefrac}
\usepackage{fontspec,xltxtra,xunicode}
\usepackage{xcolor}
\usepackage{simplewick}
\defaultfontfeatures{Mapping=tex-text}

\newcommand{\polv}{\ensuremath{\updownarrow}}
\newcommand{\polh}{\ensuremath{\leftrightarrow}}
\newcommand{\poldr}{\rotatebox[origin=c]{45}{\ensuremath{\leftrightarrow}}}
\newcommand{\poldl}{\rotatebox[origin=c]{-45}{\ensuremath{\leftrightarrow}}}
\newcommand{\bigzero}{\mbox{\normalfont\Large\bfseries 0}}
\newcommand{\vecrp}{\ensuremath{\vec{r}^{\,\prime}}}
\newcommand{\vecnr}{\ensuremath{\vec{\nabla}_{\!r}}}

\title{Advanced Quantum Mechanics\\Class 19 (b)}
%\author{The Author}
\date{May 04, 2023}                                           % Activate to display a given date or no date

\setcounter{section}{4}
\setcounter{subsection}{8}
\setcounter{equation}{155}

\begin{document}
\maketitle

%%% 15 OKAY

\subsection{Spherical tensors (continued)}

Reminder: \emph{spherical tensors} are operators that transform in a well-defined way by rotations (via the Wigner $D$-matrix).
Alternatively, they can be defined by their commutation relations with the $\hat{J}_0$ and $\hat{J}_\pm$
operators.

\begin{small}
\setcounter{equation}{74}
\be
\bra{\tau^\prime,j^\prime m^\prime}
\ket{\tau,j m} = \delta_{\tau\tau^\prime}\delta_{jj^\prime}\delta_{mm^\prime},\,
\sum_{m=-j}^{+j}\op{\tau,j m} = \mathbf{1}
\label{eq:g76}
\ee
\setcounter{equation}{133}
\be
D_{m_{1} m_{1}^{\prime}}^{\left(j_{1}\right)} D_{m_{2} m_{2}^{\prime}}^{\left(j_{2}\right)}=\sum_{m, m^{\prime}=-j}^{+j} C_{m_{1} m_{1}^{\prime} ; j m}^{j_{1} j_{2}} C_{m_{2} m_{2}^{\prime} ; j m^{\prime}}^{j_{1} j_{2}} D_{m m^{\prime}}^{(j)}
\label{eq:g134}
\ee
\be
\setcounter{equation}{137}
[\hat{\vec{J}},\hat{S}\,] = 0
\ee
\be
\setcounter{equation}{142}
[\hat{J_{i}}, \hat{V_{j}}]=i \hbar \varepsilon_{i j k} V_{k}
\ee
\setcounter{equation}{151}
\begin{align} 
{\left[\hat{J}_{0}, \hat{T}_{q}^{(k)}\right] } &=\hbar q \hat{T}_{q}^{(k)} \label{eq:g152}\\ 
{\left[\hat{J}_{\pm}, \hat{T}_{q}^{(k)}\right] } &=\hbar \sqrt{k(k+1)-q(q \pm 1)} \hat{T}_{q \pm 1}^{(k)}  \label{eq:g153}
\end{align}
\end{small}

\setcounter{equation}{155}

\par\noindent\rule{\textwidth}{1.0pt}

\emph{Examples}

1) Scalar product of two vector operators \(\hat{\vec{V}}\) and \(\hat{\vec{U}}\) :
\be
\begin{aligned} 
{\left[\hat{J}_{i}, \hat{\vec{U}} \cdot \hat{\vec{V}}\right] } 
&=\hat{U}_{l}\left[\hat{J}_{i}, \hat{V}_{l}\right]+
\left[\hat{J}_{i},\hat{U}_{l}\right] \hat{V}_{l}\\
&= \hat{U}_{l} i \hbar \varepsilon_{i l k} \hat{V}_{k} + i \hbar \varepsilon_{i l k} \hat{U}_{k} \hat{V}_{l} \\ 
& = i \hbar \varepsilon_{i l k} (\hat{U}_{l} \hat{V}_{k}+ 
\underbrace{\hat{U}_{k} \hat{V}_{l}}_{k\leftrightarrow l})=0
\end{aligned}
\ee
$\vec{U}\cdot\vec{V}$ is a \emph{scalar}. What kind of $\hat{T}_{kq}$ is it?
From Eq.~\eqref{eq:g152}: \emph{$q=0$}, from Eq.~\eqref{eq:g153}: \emph{$k=0$} $\Rightarrow$
$\vec{U}\cdot\vec{V}$ is a $T_0^{(0)}$ spherical tensor.

%%% 16 OKAY

2) \emph{Vector} $\hat{V}_i$:

\[\left[\hat{J}_{i}, \hat{V}_{j}\right]=i \hbar \varepsilon_{i j k} \hat{V}_{k},\]
so we calculate
\be
\left[\hat{J}_{0}, \hat{V}_{z}\right]=0 \therefore q=0 \therefore \hat{V}_{0} \equiv \hat{V}_{z}
\label{eq:g157}
\ee
and
\be
\left[\hat{J}_{0}, \hat{V}_{x}\right]= i \hbar \hat{V}_{y},\quad
\left[\hat{J}_{0}, \hat{V}_{y}\right]=-i \hbar \hat{V}_{x}
\ee
then, summing these two
\be
\left[\hat{J}_{0}, \left(\hat{V}_{x} \pm i \hat{V}_{y}\right)\right]=\pm \hbar\left(\hat{V}_{x} \pm i \hat{V}_{y}\right)
\ee
Defining:
\be
\hat{V}_{+}=\hat{V}_{1}=-\frac{1}{\sqrt{2}}\left(\hat{V}_{x}+i \hat{V}_{y}\right), \quad \hat{V}_{-}=V_{-1}=\frac{1}{\sqrt{2}}\left(\hat{V}_{x}-i \hat{V}_{y}\right)
\label{eq:g160}
\ee
one obtains
\be
\left[\hat{J}_{0}, \hat{V}_{\pm}\right]=\left[J_{0}, V_{\substack{\phantom{-}1\\-1}}\right]=\pm \hbar \hat{V}_{\pm}=\pm \hbar \hat{V}_{\substack{\phantom{-}1\\-1}}
\ee
Comparing \eqref{eq:g157} and \eqref{eq:g160} with \eqref{eq:g152}: $\boxed{q=0,\pm 1}$.
On the other hand, the commutation relations
\[
\left[\hat{J}_{x}, \hat{V}_{z}\right]=-i \hbar \hat{V}_{y},\left[\hat{J}_{y}, \hat{V}_{z}\right]=i \hbar \hat{V}_{x}
\]
are equivalent to
\be
\left[\hat{J}_\pm, \, \hat{V}_{0}\right] = 
\left[\hat{J}_{x} \pm i \hat{J}_{y}, \, \hat{V}_{0}\right]=\hbar \sqrt{2} \hat{V}_{\pm}
\ee
so, remembering that \emph{$k=1$} from Eq.~\eqref{eq:g153}
\be
\boxed{\hat{V}_{0}=\hat{T}_{10}}
\ee

%%% 17 OKAY

Let us check whether $\hat{V}_{\pm}=\hat{T}_{1 \pm 1}$:
\be
\begin{aligned} 
{\left[\hat{J}_{\pm}, \hat{V}_{\pm 1}\right] } 
&=\mp \frac{1}{\sqrt{2}}\left[\hat{J}_{x} \pm i \hat{J}_{y}, \hat{V}_{x} \pm i \hat{V}_{y}\right] \\ 
&=\mp \frac{1}{\sqrt{2}}\left(\pm i\left[\hat{J}_{x}, \hat{V}_{y}\right] \pm i\left[\hat{J}_{y}, \hat{V}_{x}\right]\right)\\
&=-\frac{i}{\sqrt{2}}\left(i \hbar \hat{V}_{0}-i \hbar \hat{V}_{0}\right)=0 
\end{aligned}
\ee
which, by comparing with \eqref{eq:g153}, confirms that \emph{$k=1$}.
\be
\boxed{
\begin{aligned} 
\vec{V}=\left(\hat{V}_{1},\right.&\left.\hat{V}_{-1}, V_{0}\right): \text {rank 1 spherical tensor } \\ 
& \hat{T}_{q}^{(1)}\,, q=-1,0,1 
\end{aligned}
}
\ee


\subsection{Products of spherical tensors} 
Products of spherical tensors arise often in applications.
They can be reduced ($\equiv$ ``recoupled'') by using
a procedure similar to that used for coupling
angular momentum.

Let \(T_{q_{1}}^{\left(k_{1}\right)}\) and \(S_{q_{2}}^{\left(k_{2}\right)}\) be spherical tensors. In general,
their product is not a spherical tensor. However, the
following linear combination is:
%%% 18 OKAY
\be
\left[T^{(k_1)} S^{(k_2)}\right]_{q}^{(k)} \equiv \sum_{q_{1}
^{\prime} q_{2}^{\prime}} C_{q_{1}^{\prime} q_{2}^{\prime} ; k q}^{k_{1} k_{2}}\,T_{q_{1}^{\prime}}^{\left(k_{1}\right)} S_{q_{2}^{\prime}}^{\left(k_{2}\right)}
\label{eq:g166}
\ee
Let us see how one is led to this. The product
\(T_{q_{1}}^{\left(k_{1}\right)} S_{q_{2}}^{\left(k_{2}\right)}\) transforms under rotation as:
\[
\begin{gathered}
T_{q_{1}}^{\left(k_{1}\right)^{\prime}} S_{q_{2}}^{\left(k_{2}\right)^{\prime}}=\sum_{q_{1}^{\prime}} D_{q_{1}^{\prime} q_{1}}^{\left(k_{1}\right)}[R] T_{q_{1}^{\prime}}^{\left(k_{1}\right)} \sum_{q_{2}^{\prime}} D_{q_{2}^{\prime} q_{2}}^{\left(k_{2}\right)}[R] S_{q_{2}^{\prime}}^{\left(k_{2}\right)}\\
=\sum_{q_{1}^{\prime} q_{2}^{\prime}} T_{q_{1}^{\prime}}^{\left(k_{1}\right)} S_{q_{2}^{\prime}}^{\left(k_{2}\right)} D_{q_{1}^{\prime} q_{1}}^{\left(k_{1}\right)}[R] D_{q_{2}^{\prime} q_{2}}^{\left(k_{2}\right)}[R],\,\text{ use \eqref{eq:g134}}\\
=\sum_{q_{1}^{\prime} q_{2}^{\prime}} T_{q_{1}^{\prime}}^{\left(k_{1}\right)} S_{q_{2}^{\prime}}^{\left(k_{2}\right)} \sum_{q^{\prime} q^{\prime \prime}} C_{q_{1}^{\prime} q_{2}^{\prime} ; k q^{\prime}}^{k_{1} k_{2}} C_{q_{1} q_{2} ; k q^{\prime \prime}}^{k_{1} k_{2}} D_{q^{\prime} q^{\prime \prime}}^{(k)}
\end{gathered}
\]
Multiply this by \(C_{q_{1} q_{2} ; k q}^{k_{1} k_{2}}\) and sum over \(q_{1} q_{2}\):
\be
\begin{gathered}
\sum_{q_{1} q_{2}} C_{q_{1} q_{2} ; k q}^{k_{1} k_{2}} 
%
T_{q_{1}}^{\left(k_{1}\right)^{\prime}}
S_{q_{2}}^{\left(k_{2}\right)^{\prime}}=\\
=\sum_{q_{1}^{\prime} q_{2}^{\prime}} T_{q_{1}^{\prime}}^{\left(k_{1}\right)} S_{q_{2}^{\prime}}^{\left(k_{2}\right)} \sum_{q^{\prime} q^{\prime\prime}} C_{q_{1}^{\prime} q_{2}^{\prime}; k q}^{k_{1} k_{2}} D_{q^{\prime} q^{\prime \prime}}^{(k)}
\times \sum_{q_{1} q_{2}} C_{q_{1} q_{2} ; k q}^{k_{1} k_{2}} C_{q_{1} q_{2} ; k q^{\prime\prime}}^{k_{1} k_{2}} \rightarrow \delta_{q^{\prime\prime}q}\\
=\sum_{q_{1}^{\prime} q_{2}^{\prime}} T_{q_{1}^{\prime}}^{\left(k_{1}\right)} S_{q_{2}^{\prime}}^{\left(k_{2}\right)} \sum_{q^{\prime}} C_{q_{1}^{\prime} q_{2}^{\prime} ; k q^{\prime}}^{k_{1} k_{2}} D_{q^{\prime} q}^{(k)}
=\sum_{q^{\prime}} D_{q^{\prime} q}^{(k)} \sum_{q_{1}^{\prime} q_{2}^{\prime}} C_{q_{1}^{\prime} q_{2}^{\prime} ; k q}^{k_{1} k_{2}} T_{q_{1}^{\prime}}^{\left(k_{1}\right)} S_{q_{2}^{\prime}}^{\left(k_{2}\right)}
\end{gathered}
\ee
%%% 19 OKAY
From the definition in \eqref{eq:g166}, one can write this as
\be
\left[T^{\left(k_{1}\right)} S^{\left(k_{2}\right)}\right]_{q}^{(k)}=\sum_{q^{\prime}=-k}^{+k}\left[T^{\left(k_{1}\right)} S^{\left(k_{2}\right)}\right]_{k q^{\prime}} D_{q^{\prime} q}^{(k)}
\label{eq:g168}
\ee
\emph{That is}: the product $[T^{(k_1)} S^{(k_2)}]_{q}^{(k)}$ is indeed a
spherical tensor of rank $k$ and components $-k \leqslant q \leqslant k$, where
$|k_1-k_2| \leqslant k \leqslant k_1 + k_2$.

One can invert Eq.~\eqref{eq:g168}; \textit{i.e.} one can write the
product $T_{q_1}^{(k_1)} S_{q_2}^{(k_2)}$ in terms of the coupled tensor:
\be
T_{q_{1}}^{\left(k_{1}\right)} S_{q_{2}}^{\left(k_{2}\right)}=\sum_{k q}\left[T^{\left(k_{1}\right)} S^{\left(k_{2}\right)}\right]_{q}^{(k)} C_{q_{1} q_{2} ; k q}^{k_{1} k_{2}}
\ee

\clearpage
\emph{Example:} Sakurai \& Napolitano, Problem 3.30. Write
\[
T_{q}^{(k)}=\sum_{q_{1}=-1}^{1} \sum_{q_{2}=-1}^{1}\left\langle 11 ; q_{1} q_{2} \mid 11 ; k q\right\rangle U_{q_{1}}^{(1)} V_{q_{2}}^{(1)}
\]
where \(U_{0}^{(1)}=U_{z}, U_{\pm 1}^{(1)}=\mp\left(U_{x} \pm i U_{y}\right) / \sqrt{2}\), and similarly for \(V_{q}^{(1)}\).

\[
\begin{aligned} 
T_{+1}^{(1)} &=\langle 11 ; 01 \mid 11 ; 11\rangle U_{0}^{(1)} V_{+1}^{(1)}+\langle 11 ; 10 \mid 11 ; 11\rangle U_{+1}^{(1)} V_{0}^{(1)} \\ &=\frac{1}{\sqrt{2}}\left[-U_{0}^{(1)} V_{+1}^{(1)}+U_{+1}^{(1)} V_{0}^{(1)}\right] \\ &=\frac{1}{2}\left[U_{z}\left(V_{x}+i V_{y}\right)-\left(U_{x}+i U_{y}\right) V_{z}\right]=\frac{1}{2}\left[U_{z} V_{x}-U_{x} V_{z}+i\left(U_{z} V_{y}-U_{y} V_{z}\right)\right] \\ 
%
T_{0}^{(1)} &=\langle 11 ;-1,1 \mid 11 ; 10\rangle U_{-1}^{(1)} V_{+1}^{(1)}+\langle 11 ; 00 \mid 11 ; 10\rangle U_{0}^{(1)} V_{0}^{(1)}\langle 11 ; 1,-1 \mid 11 ; 10\rangle U_{+1}^{(1)} V_{-1}^{(1)} \\ &=\frac{1}{\sqrt{2}}\left[-U_{-1}^{(1)} V_{+1}^{(1)}+U_{+1}^{(1)} V_{-1}^{(1)}\right] \\ &=\frac{1}{2 \sqrt{2}}\left[\left(U_{x}-i U_{y}\right)\left(V_{x}+i V_{y}\right)-\left(U_{x}+i U_{y}\right)\left(V_{x}-i V_{y}\right)\right]=\frac{i}{\sqrt{2}}\left[U_{x} V_{y}-U_{y} V_{x}\right] \\
%
T_{-1}^{(1)} &=\langle 11 ;-1,0 \mid 11 ; 1,-1\rangle U_{-1}^{(1)} V_{0}^{(1)}+\langle 11 ; 0,-1 \mid 11 ; 1,-1\rangle U_{0}^{(1)} V_{-1}^{(1)} \\ &=\frac{1}{\sqrt{2}}\left[-U_{-1}^{(1)} V_{0}^{(1)}+U_{0}^{(1)} V_{-1}^{(1)}\right] \\ &=\frac{1}{2}\left[-\left(U_{x}-i U_{y}\right) V_{z}+U_{z}\left(V_{x}-i V_{y}\right)\right]=\frac{1}{2}\left[U_{z} V_{x}-U_{x} V_{z}+i\left(U_{y} V_{z}-U_{z} V_{y}\right)\right] \\ 
%
T_{+2}^{(2)} &=U_{+1}^{(1)} V_{+1}^{(1)}=\frac{1}{2}\left[\left(U_{x}+i U_{y}\right)\left(V_{x}+i V_{y}\right)\right]=\frac{1}{2}\left[U_{x} V_{x}-U_{y} V_{y}+i\left(U_{y} V_{x}+U_{x} V_{y}\right)\right] \\ 
%
T_{+1}^{(2)} &=\langle 11 ; 01 \mid 11 ; 21\rangle U_{0}^{(1)} V_{+1}^{(1)}+\langle 11 ; 10 \mid 11 ; 21\rangle U_{+1}^{(1)} V_{0}^{(1)} \\ &=\frac{1}{\sqrt{2}}\left[U_{0}^{(1)} V_{+1}^{(1)}+U_{+1}^{(1)} V_{0}^{(1)}\right] \\ &=-\frac{1}{2}\left[U_{z}\left(V_{x}+i V_{y}\right)+\left(U_{x}+i U_{y}\right) V_{z}\right]\\
%
T_{0}^{(2)} &=\langle 11 ;-1,1 \mid 11 ; 20\rangle U_{-1}^{(1)} V_{+1}^{(1)}+\langle 11 ; 00 \mid 11 ; 20\rangle U_{0}^{(1)} V_{0}^{(1)}\langle 11 ; 1,-1 \mid 11 ; 20\rangle U_{+1}^{(1)} V_{-1}^{(1)} \\ &=\frac{1}{\sqrt{6}}\left[U_{-1}^{(1)} V_{+1}^{(1)}+2 U_{0}^{(1)} V_{0}^{(1)}+U_{+1}^{(1)} V_{-1}^{(1)}\right] \\ &=\frac{1}{2 \sqrt{6}}\left[-\left(U_{x}-i U_{y}\right)\left(V_{x}+i V_{y}\right)+4 U_{z} V_{z}-\left(U_{x}+i U_{y}\right)\left(V_{x}-i V_{y}\right)\right] \\ &=-\frac{1}{\sqrt{6}}\left[U_{x} V_{x}+U_{y} V_{y}+2 U_{z} V_{z}\right] \\ 
%
T_{-1}^{(2)} &=\langle 11 ;-1,0 \mid 11 ; 2,-1\rangle U_{-1}^{(1)} V_{0}^{(1)}+\langle 11 ; 0,-1 \mid 11 ; 2,-1\rangle U_{0}^{(1)} V_{-1}^{(1)} \\ &=\frac{1}{\sqrt{2}}\left[U_{-1}^{(1)} V_{0}^{(1)}+U_{0}^{(1)} V_{-1}^{(1)}\right] \\ &=\frac{1}{2}\left[\left(U_{x}-i U_{y}\right) V_{z}+U_{z}\left(V_{x}-i V_{y}\right)\right]=\frac{1}{2}\left[U_{z} V_{x}+U_{x} V_{z}-i\left(U_{y} V_{z}+U_{z} V_{y}\right)\right] \\ 
%
T_{-2}^{(2)} &=U_{-1}^{(1)} V_{-1}^{(1)}=\frac{1}{2}\left[\left(U_{x}-i U_{y}\right)\left(V_{x}-i V_{y}\right)\right]=\frac{1}{2}\left[U_{x} V_{x}-U_{y} V_{y}-i\left(U_{y} V_{x}+U_{x} V_{y}\right)\right] \end{aligned}
\]

\clearpage


%%% Acabou o 19, vai pro 22 OKAY

These results point to a generic result, regarding a generic $\hat{T}_{ij}$:
\begin{itemize}
\item In general, a generic rank-2 tensor \(\hat{T}_{i j}\)
is a more complicated object, for it can
in general be decomposed into a sum of
tensors of different ranks.
\end{itemize}
To see this, let us consider the previous example,
in that the cartesian components of \(\vec{U}\) and \(\vec{V}\) were used:
\setcounter{equation}{183}
\be
\hat{T}_{ij}=\hat{U}_{i} \hat{V}_{j}
\ee
This product can be rewritten as:
%%% 23 OKAY
\be
\begin{aligned}
\hat{T}_{i j}
&=\hat{U}_{i} \hat{V}_{j}=\frac{1}{3}(\hat{\vec{U}} \cdot \hat{\vec{V}}) \delta_{i j}\\
&+\frac{1}{2}\left(\hat{U}_{i} \hat{V}_{j}-\hat{U}_{j} \hat{V}_{i}\right) \leftarrow \text{ antisymmetric in }(ij)\\
&+\left[\underbrace{\frac{1}{2}\left(\hat{U}_{i} \hat{V}_{j}+\hat{U}_{j} \hat{V}_{i}\right)}-\frac{1}{3}(\hat{\vec{U}} \cdot \hat{\vec{V}}) \delta_{i j}\right]\\
&\quad\quad\text{symmetric in }(ij)
\label{eq:g185}
\end{aligned}
\ee
The first component, $\hat{\vec{U}} \cdot \hat{\vec{V}}$, is clearly a scalar under
rotation and is a rank-0 spherical tensor: $\frac{1}{3}\vec{U}\cdot\vec{V} = -\frac{1}{\sqrt{3}}Z_0^{0}$.
That is, up to a normalization, $T_0^{0} = Z_0^{0}$. The last
two terms in %\textcolor{red}
{\eqref{eq:g185}} are traceless. The antisymmetric
term is formed by the components of the cross-%
product ($\vec{U}\times\vec{V}$): $(\hat{\vec{U}}\times\hat{\vec{V}})_k = \varepsilon_{ijk}\hat{U}_i\hat{V}_j = \varepsilon_{ijk} \frac{1}{2} (\hat{U}_i\hat{V}_j-\hat{U}_j\hat{V}_i)$
and is, up to a normalization, $T_q^{(1)} = Z_q^{(1)}$. And so is
the last term, it has $q-3-1=5$ independent
components: $T_q^{(2)} \sim Z_q^{(2)}$.

Of course not all $\hat{T}_{ij}$ can be written as a $\hat{U}_j\hat{V}_j$. But
any rank-2 tensor given in terms of cartesian
can be decomposed as in the above; namely
\be
\hat{T}_{i j}=\hat{E} \delta_{i j}+\hat{A}_{i j}+\hat{S}_{i j}
\ee
where
%%% 24 OKAY
\be
\hat{E}=\frac{1}{3} \sum_{i} \hat{T}_{ii}=\frac{1}{3} \Tr \hat{T}
\ee
%
\be
\hat{A}_{i j}=\frac{1}{2}\left(\hat{T}_{i j}-\hat{T}_{j i}\right) \leftarrow \text{ antisymmetric in }(ij)
\ee
%
\be
\hat{S}_{i j}=\frac{1}{2}\left(\hat{T}_{i j}+\hat{T}_{j i}\right)-\frac{1}{3} \delta_{i j} \sum_{k} \hat{T}_{kk} \leftarrow \text{ symmetric in }(ij)
\ee
By now, it is clear that \(\hat{E}\) is scalar, \(\hat{A}_{i j}\) will
take part as components of \(\hat{T}_{q}^{(1)}\) and \(\hat{S}_{i j}\) in \(\hat{T}_{q}^{(2)}\).

\emph{Exercise:} Consider a potential energy given in terms
of the cartesian components of the position vector as
\be
\hat{V}(x, y, z)=V_{0}\,\hat{x} \hat{y}
\ee
Show that this can be written is terms of spherical
tensors as
\be
\hat{V}(x, y, z)=-\frac{i}{2}\left(\hat{V}_{2}^{(2)}-\hat{V}_{-2}^{(2)}\right)
\ee
where
\be
\hat{V}_{\pm2}^{(2)} = \frac{1}{2}\left(\hat{x}^2-\hat{y}^2\pm2i\hat{x}\hat{y}\right)
\ee
Now we are equipped to prove one of
the most important results in physics.

%%% 25 OKAY

\subsection{Wigner-Eckart theorem}

Consider \(S_{q}^{(k)}\) being the spherical tensor \(\left|\tau, j m\right\rangle\)
in Eq.~\eqref{eq:g168}:

\be
\left[\hat{T}^{(k_{1})}|\tau, j\rangle\right]_{q}^{(k)}=\sum_{q_{1} m} \hat{T}_{q_{1}}^{(k_{1})}|\tau, j m\rangle C_{q_{1} m ; k q}^{k_{1} j}
\label{eq:g193}
\ee
where $\left[\hat{T}^{(k_{1})}|\tau, j\rangle\right]_{q}^{(k)}$ is an
operator acting on an eigenstate \(|\tau, j m\rangle\) and produces a
state that transforms as \(|k q\rangle\); therefore
\[
{\hat{\vec{J}}}^{\,2}|k q\rangle=k(k+1) \hbar^{2}|k q\rangle, \quad \hat{J}_{0}|k q\rangle=\hbar q|k q\rangle
\]
Let us take the scalar product of \eqref{eq:g193} with \(\left|\tau^{\prime}, j^{\prime} m^{\prime}\right\rangle\),
multiply by \(C_{q^{\prime} m^{\prime \prime} ; k q}^{k_{1}j}\) and sum over \(k q\) :
\[
\begin{aligned} 
& \sum_{k q}\left\langle\tau^{\prime}, j^{\prime} m^{\prime}\right|\left[\hat{T}^{\left(k_{1}\right)}|\tau, j\rangle\right]_{q}^{(k)} C_{q^{\prime} m^{\prime \prime} ; k q}^{k_{1} j}=\\
=& \sum_{q_{1} m}\left\langle\tau^{\prime}, j^{\prime} m^{\prime}\left|\hat{T}_{q_{1}}^{\left(k_{1}\right)}\right| \tau, j m\right\rangle \underbrace{\sum_{k q} C_{q^{\prime} m^{\prime \prime} ; k q}^{k_{1} j} C_{q_{1}, m ; k q}^{k_{1} j}}%
_{\delta_{q^{\prime} q_{1}}\delta_{m^{\prime \prime} m}}
\end{aligned}
\]
therefore
\[
\left\langle\tau^{\prime}, j^{\prime} m^{\prime}\right|\left[T^{\left(k_{1}\right)}|\tau, j\rangle\right]_{m^{\prime}}^{\left(j^{\prime}\right)} C_{q^{\prime} m^{\prime \prime} ; j^{\prime} m^{\prime}}^{k_{1} j}=\left\langle\tau^{\prime}, j^{\prime} m^{\prime}\left|T_{q^{\prime}}^{\left(k_{1}\right)}\right| \tau, j m^{\prime \prime}\right\rangle
\]
Let's make $m^{\prime \prime} \rightarrow m, q^{\prime} \rightarrow q, k_{1} \rightarrow k$ $\Rightarrow$ to lighten the notation.
\be
\underbrace{\left\langle\tau^{\prime}, j^{\prime} m^{\prime}\right|\left[T^{(k)}|\tau, j\rangle\right]_{m^{\prime}}^{\left(j^{\prime}\right)}}%
_{\Circled{1}} 
C_{q m ; j^{\prime} m^{\prime}}^{k j} =\left\langle\tau^{\prime}, j^{1} m^{\prime}\left|T_{q}^{(k)}\right| \tau, j m\right\rangle
\ee
this $\Circled{1}$ term is independent of $m^\prime$ $\Rightarrow$ this will be shown shortly ahead.

%%% 26 OKAY

This matrix element $\Circled{1}$ is commonly written as
\be
\left\langle\tau^{\prime}, j^{\prime} m^{\prime}\right|\left[T^{(k)}|\tau, j\rangle\right]_{m^{\prime}}^{\left(j^{\prime}\right)}=\frac{\left\langle\tau^{\prime}, j^{\prime}\left\|T^{(k)}\right\| \tau, j\right\rangle}{\sqrt{2 j^{\prime}+1}}
\ee
where the double bar term ( $\langle\|\cdots\|\rangle$ ) is called the \emph{reduced} matrix element.
Therefore, one obtains that:
\be
\left\langle\tau^{\prime}, j^{\prime} m^{\prime}\left|T_{q}^{(k)}\right| \tau, j m\right\rangle=
\frac{C_{q m ; j^{\prime} m^{\prime}}^{k j} }{\sqrt{2 j^{\prime}+1}}\left\langle\tau^{\prime}, j^{\prime} \| T^{(k)}|| \tau, j\right\rangle
\label{eq:g196}
\ee
with
\be
|j-k| \leqslant j^\prime \leqslant j+k, m^\prime=m+q
\ee
Eq.~\eqref{eq:g196} is the expression of the Wigner-Eckart theorem:
the matrix element of a spherical tensor \(T_q^{(k)}\) in
the states \(|\tau, j m\rangle\) and \(\left|\tau^{\prime}, j^{\prime} m^{\prime}\right\rangle\) can be factorized
into the product of a matrix element that \emph{does not}
depend on the projections \(m, m^{\prime}\) and \(q\) (the so-called
reduced matrix element) and a kinematical factor
given by a CG coeficient and a normalization
factor.

To conclude, let us show that the expression \(\left\langle\tau^{\prime}, j^{\prime} m^{\prime}\right|\left[\hat{T}^{\left(k_{1}\right)}|\tau, j\rangle\right]_{m^{\prime}}^{\left(j^{\prime}\right)}\)
\emph{does not} depend on \(m^{\prime}\):
\begin{itemize}
\item first, since \(\left[T_{q}^{(k)}|\tau, j\rangle\right]_{m^{\prime}}^{\left(j^{\prime}\right)} \sim\left|j^{\prime} m^{\prime}\right\rangle\), one can write (via Eq.~(75)):
\be
\left[T_{q}^{(k)}|\tau, j\rangle\right]_{m^{\prime}\pm1}^{\left(j^{\prime}\right)}
=
\frac{\hat{J}_{\pm}[\hat{T}^{(k)}\ket{\tau,j}]_{m^\prime}^{(j^\prime)}}
{\hbar\sqrt{j^\prime(j^\prime+1)-m^\prime(m^\prime\pm1)}}
\ee

%%% 27 OKAY
\item using the result \(\hat{J}_{\mp} \hat{J}_{\pm}=\hat{J}^{2}-\hat{J}_{0}\left(\hat{J}_{0} \pm 1\right)\), one has
\[
\begin{aligned}
\bra{\tau^\prime,j^\prime m^\prime\pm1} \left[\hat{T}^{(k)}\ket{\tau,j}\right]^{j^\prime}_{m^\prime\pm1}
&=\frac{\bra{\tau^\prime,j^\prime m^\prime} \left[\hat{J}^{2}-\hat{J}_{0}\left(\hat{J}_{0} \pm 1\right)\right] \left[\hat{T}^{(k)}\ket{\tau,j}\right]^{j^\prime}_{m^\prime}}%
{\hbar^2[j^{\prime}\left(j^{\prime}+1\right)-m^{\prime}\left(m^{\prime}\pm1\right)]}\\
&=\bra{\tau^\prime,j^\prime m^\prime}\left[\hat{T}^{(k)}\ket{\tau,j}\right]^{j^\prime}_{m^\prime}
\end{aligned}
\]
But the left-hand side is the matrix element for $m^\prime \pm 1$, 
while the right-hand side is the matrix element for $m^\prime$.
Therefore the matrix element does not depend on $m^\prime$.
\end{itemize}
\end{document}




















