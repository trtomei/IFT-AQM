%!TEX TS-program = xelatex
%!TEX encoding = UTF-8 Unicode

\documentclass[12pt]{article}
\usepackage{geometry}                % See geometry.pdf to learn the layout options. There are lots.
\geometry{a4paper,top=2cm}
\usepackage[parfill]{parskip}    % Activate to begin paragraphs with an empty line rather than an indent
\usepackage{graphicx}
\usepackage{amsmath}
\usepackage{amssymb}
\usepackage{mathtools}
\usepackage{physics}
\newcommand{\be}{\begin{equation}}
\newcommand{\ee}{\end{equation}}
\usepackage[thicklines]{cancel}
\usepackage[colorlinks=true,citecolor=blue,linkcolor=blue,urlcolor=blue]{hyperref}
\usepackage{booktabs}
\usepackage{csquotes}
\usepackage{qcircuit}
\usepackage{circledsteps}
\usepackage{nicefrac}
\usepackage{fontspec,xltxtra,xunicode}
\usepackage{xcolor}
\usepackage{simplewick}
\defaultfontfeatures{Mapping=tex-text}

\newcommand{\polv}{\ensuremath{\updownarrow}}
\newcommand{\polh}{\ensuremath{\leftrightarrow}}
\newcommand{\poldr}{\rotatebox[origin=c]{45}{\ensuremath{\leftrightarrow}}}
\newcommand{\poldl}{\rotatebox[origin=c]{-45}{\ensuremath{\leftrightarrow}}}
\newcommand{\bigzero}{\mbox{\normalfont\Large\bfseries 0}}
\newcommand{\vecrp}{\ensuremath{\vec{r}^{\,\prime}}}
\newcommand{\vecnr}{\ensuremath{\vec{\nabla}_{\!r}}}

\title{Advanced Quantum Mechanics\\Class 02}
%\author{The Author}
\date{August 11, 2022}                                           % Activate to display a given date or no date

\begin{document}
\maketitle

\setcounter{section}{3}
\setcounter{equation}{15}
\setcounter{subsection}{1}

%%% 1 OK

\subsection{Heisenberg inequalities}

Let $\mathcal{A}$ and $\mathcal{B}$ be incompatible properties,
\[
[\hat{A}, \hat{B}] \neq 0
\]

Suppose measurement of $\mathcal{A}$ gives $a$. Immediately
measuring $\mathcal{B}$ will give $b$ (and suppose $b$ is nondegenerate). 
In general, $b \neq a$, i.e.
it is not possible in general to find states for which $\mathcal{A}$ and $\mathcal{B}$ are known exactly. 

%%% 2 OK

This, with
\[
p(x \rightarrow b)=| \bra{b}\ket{a} |^{2}
\]
implies that there will be a \emph{dispersion} in values of A and B.

\emph{Definition:}

\begin{enumerate}
\item Suppose the system is in state $|\varphi\rangle$. The
\emph{dispersions} in $\mathcal{A}$ and $\mathcal{B}$ are defined,
assuming 
$\hat{A}=\hat{A}^{\dagger}$ and $\hat{B}=\hat{B}^{\dagger}$,
as
\begin{equation}
\left(\Delta_{\varphi} \hat{A}\right)^{2}=\left\langle\hat{A}^{2}\right\rangle_{\varphi}-\left(\langle\hat{A}\rangle_{\varphi}\right)^{2}=\left\langle\left(\hat{A}-\langle\hat{A}\rangle_{\varphi}\right)^{2}\right\rangle
\end{equation}
and likewise for $\hat{B}$
%
\item Suppose:
\begin{equation}
[\hat{A}, \hat{B}]=i \hat{C}
\end{equation}
then $\hat{C} = \hat{C}^{\dagger}$:
\begin{equation}
\begin{gathered}
[\hat{A}, \hat{B}]^{\dagger}=\left[\hat{B}^{\dagger},\hat{A}^{\dagger}\right]=[\hat{B}, \hat{A}]=-[\hat{A}, \hat{B}] = - i\hat{C}\\ 
[\hat{A}, \hat{B}]^{\dagger} = (i \hat{C})^{\dagger} = -i \hat{C}^{\dagger} 
\therefore \hat{C}=\hat{C}^{\dagger} 
\end{gathered}
\end{equation}
%
\item Define
\begin{equation}
\begin{array}{l}\hat{A}_{0}^{\varphi}=\hat{A}-\langle\hat{A}\rangle_{\varphi} \\ \hat{B}_{0}^{\varphi}=\hat{B}-\langle\hat{B}\rangle_{\varphi}\end{array}
\end{equation}
which implies
\begin{equation}
\left[\hat{A}_{0}^{\varphi}, \hat{B}_{0}^{\varphi}\right]=[\hat{A}, \hat{B}]=i \hat{C}
\end{equation}
%
\item Consider the \emph{squared norm} of the state
\begin{equation}
\left(\hat{A}_{0}^{\varphi}+i \lambda \hat{B}_{0}^{\varphi}\right)|\varphi\rangle, \lambda \textrm { real }
\end{equation}
which is necessarily positive:
%%% 3 OK
\begin{equation}
\|\left(\hat{A}_{0}^{\varphi}+i \lambda \hat{B}_{0}^{\varphi}\right)|\varphi\rangle \|^{2} \geqslant 0
\label{eq:g23}
\end{equation}
%
Then, it can be shown (exercise) that
\begin{equation}
\lambda^{2}\left\langle\hat{B}_{0}^{\varphi^2}\right\rangle_{\varphi}-
\lambda\langle\hat{C}\rangle_{\varphi}+
\left\langle\hat{A}_{0}^{\varphi^{2}}\right\rangle_{\varphi} \geqslant 0
\label{eq:g24}
\end{equation}
and that must not have a real root, then discriminant negative:
\begin{equation}
\langle\hat{C}\rangle_{\varphi}^{2}-4
\left\langle\hat{A}_{0}^{\varphi^{2}}\right\rangle_{\varphi}
\left\langle\hat{B}_{0}^{\varphi^{2}}\right\rangle_{\varphi} \leqslant 0
\label{eq:g25}
\end{equation}
%
\end{enumerate}

But:
\begin{equation}
\left\langle\left(\hat{A}_{0}^{\varphi}\right)^{2}\right\rangle_{\varphi}
=\left(\Delta_{\varphi} \hat{A}\right)^{2}
\quad,\quad
\left\langle\left(\hat{B}_{0}^{\varphi}\right)^{2}\right\rangle_{\varphi}
=\left(\Delta_{\varphi} \hat{B}\right)^{2}
\end{equation}
which leads to the \emph{Heisenberg inequality}:
\begin{equation}
\boxed{
\left(\Delta_{\varphi} \hat{A}\right)\left(\Delta_{\varphi} \hat{B}\right) \geqslant \frac{1}{2}\left|\langle\hat{C}\rangle_{\varphi}\right|
}
\label{eq:heisenberineq}
\end{equation}

Equality in Eq.~\eqref{eq:g23} $\to$ Eq.~\eqref{eq:g25}: solve for $\lambda$ and replace in Eq.~\eqref{eq:g23},
then find Eq.~\eqref{eq:heisenberineq} with equality.

\emph{Exercise:} 
$\Delta_\varphi \hat{A}=0 \textrm{ iff } \hat{A}|\varphi\rangle=a|\varphi\rangle$; 
for finite-dimension $H$ that implies 
$\langle\hat{C}\rangle_{\varphi}=0$.

Correct interpretation of Eq.~\eqref{eq:heisenberineq}:
$N$ measurements of $\mathcal{A}$, $\mathcal{B}$ and $\mathcal{C}$ (measured in different
experiments, of course), all prepared in same $|\varphi\rangle$,
%%% 4 OK
one obtains (ideally) accurate experimental values for 
$\Delta_{\varphi} \hat{A}$ and 
$\Delta_{\varphi} \hat{B}$ and
$\langle\hat{C}\rangle_{\varphi}$,
which then obey Eq.~\eqref{eq:heisenberineq}. 
Notice that this is in \emph{no way} related to experimental errors.
If errors are $\delta A$ and $\delta B$, for an accurate
determination of the dispersion, one needs
\begin{equation}
\delta A \ll \Delta_{\varphi} \hat{A} \textrm { and } \delta B \ll \Delta_{\varphi} \hat{B}.
\end{equation}




\subsection{Time evolution}

\emph{Postulate IV:} the Schrödinger equation
\begin{equation}
\boxed{
i \hbar \frac{d|\varphi(t)\rangle}{d t} = 
\hat{H}(t) |\varphi(t)\rangle
}
\label{eq:schrodinger}
\end{equation}
where $\hat{H} = \hat{H}^{\dagger}$ is the Hamiltonian.

Remarks:
\begin{enumerate}
\item Valid for a \emph{closed system}, i.e. a system that is \emph{not part} of a larger system.
\item Valid for a system interacting with a classical 
system $\to$ \emph{not} necessarily isolated (e.g. spin-$\frac{1}{2}$
%%% 5 OK
under the influence of a magnetic field);
\emph{not valid} when interacting with a \emph{quantized}
e.m field $\to$ $H = H_\textrm{atom} + H_\textrm{field}$. 
\end{enumerate}

Notice that Eq.~\eqref{eq:schrodinger} is 1\textsuperscript{st} order in time, and evolution is deterministic: given \mbox{$|\varphi(t=0)\rangle$}, $|\varphi(t)\rangle$ is predictable with certainty. Evolution is reversible.

\bigskip

\emph{Assumption:} duration of measurement much shorter
than ``characteristic time of the evolution''.

Time evolution is unitary, since $\hat{H} = \hat{H}^{\dagger}$ $\to$ norm is conserved under the evolution.
\begin{equation}
\begin{aligned} \frac{d}{d t} \||\varphi(t)\rangle \|^{2} &=\frac{d}{d t}\langle\varphi(t) | \varphi(t)\rangle \\ &=\left\langle\frac{d \varphi}{d t} | \varphi\right\rangle+\left\langle\varphi | \frac{d \varphi}{d t}\right\rangle\\
%%% 6 OK
&=\left\langle\varphi \left|\left(-\frac{i}{\hbar} \hat{H}\right)^{\dagger}\right|\varphi\right\rangle+\left\langle\varphi\left|\left(-\frac{i}{\hbar} \hat{H}\right)\right| \varphi\right\rangle
=0
\end{aligned}
\end{equation}

Alternative, assuming \emph{conservation of probability}.
If $|\varphi\rangle$ is expanded as
\begin{equation}
|\varphi(t)\rangle=\sum_{n, r}|n, r\rangle
\underbrace{\langle n, r| \varphi(t)\rangle}_{C_{n, r}(t)}
=\sum_{n, r} C_{n, r}(t)|n, r\rangle
\end{equation}
the $C_{n, r}(t)$ satisfy
\begin{equation}
\begin{aligned} \frac{d}{d t}\| |\varphi(t)\rangle \|^{2} &=\frac{d}{d t} \sum_{n, r}\left|C_{n, r}(t)\right|^{2}=\frac{d}{d t} \sum_{n} 
\underbrace{\sum_r \left|C_{n, r}(t)\right|^{2}}_{p\left(a_{n}, t\right)} \\ 
&=\frac{d}{d t}\left(\sum_{n} p_{n}\left(a_{n}, t\right)\right)=0 
\end{aligned}
\end{equation}

Matrix form of the evolution, for the practical calculations: 
let $\{|\alpha\rangle\}$ be an arbitrary basis of $H$. So
\[
i\hbar \frac{d}{d t} \underbrace{\langle\alpha | \varphi(t)\rangle}_{C_{\alpha}(t)}=
\langle\alpha|\hat{H}(t)| \varphi\rangle=
\sum_\beta
\underbrace{\langle\alpha|\hat{H}|\beta\rangle}_{H_{\alpha \beta}(t)}
\underbrace{\langle\beta|\varphi(t)\rangle}_{C_\beta(t)}
\]
which leads to
\begin{equation}
\boxed{i \hbar \dot{C}_{\alpha}(t)=\sum_{\beta} H_{\alpha \beta}(t) C_{\beta}(t)}
\end{equation}

%%% 7 OK

\subsection{Evolution operator}

Instead of a \emph{differential} evolution equation,
Eq.~\eqref{eq:schrodinger}, one can postulate an integral equation
for an evolution operator $\hat{U}(t,t_0)$.

\emph{Postulate IV$\,^\prime$:} the state vector $|\varphi(t)\rangle$ at time $t$
is derived from the state $|\varphi(t_0)\rangle$ at
time $t_0$ by applying a unitary operator $\hat{U}(t, t_0)$ on $|\varphi(t_0)\rangle$:
\begin{equation}
\boxed{
|\varphi(t)\rangle = \hat{U}(t, t_0)|\varphi(t_0)\rangle
}
\label{eq:postulate4prime}
\end{equation} 
and $\hat{U}(t, t_0)$ is the \emph{evolution operator}. 
Of course, it still needs a dynamical equation $\to$ see below.

Notice that $\hat{U} \hat{U}^{\dagger} = \hat{U}^{\dagger}\hat{U} = I \to$ at any $t$, norm is conserved:
\begin{equation}
\begin{aligned}
\langle\varphi(t) | \varphi(t)\rangle 
&=\left\langle\hat{U} \varphi\left(t_{0}\right) | \hat{U} \varphi\left(t_{0}\right)\right\rangle \\ 
&=\left\langle\varphi\left(t_{0}\right)\left|\hat{U}^{\dagger} \hat{U}\right| \varphi\left(t_{0}\right)\right\rangle \\ 
&=\left\langle\varphi\left(t_{0}\right) | \varphi\left(t_{0}\right)\right\rangle 
\end{aligned}
\end{equation}
Also, from conservation of norm: $\hat{U} \hat{U}^{\dagger} = I$, but we need some caution for infinite dimension $H$.

%%% 8 OK

\subsection{\texorpdfstring{Properties of $\hat{U}$}{Properties of U}}

\begin{enumerate}
\item group property
\begin{equation}
\hat{U}\left(t, t_{1}\right) \hat{U}\left(t_{1}, t_{0}\right)=\hat{U}\left(t, t_{0}\right)
\label{eq:Ugroup}
\end{equation}
Indeed:
\[
|\varphi(t)\rangle=
\hat{U}\left(t, t_{1}\right)\underbrace{|\varphi\left(t_{1}\right)\rangle}%
_{\hat{U}\left(t_{1}, t_{0}\right)\left|\varphi\left(t_{0}\right)\right\rangle}
=\hat{U}\left(t, t_{1}\right) \hat{U}\left(t_{1}, t_{0}\right)\left|\varphi\left(t_{0}\right)\right\rangle
\]
but also 
\[
|\varphi(t)\rangle = \hat{U}\left(t, t_{0}\right)\left|\varphi\left(t_{0}\right)\right\rangle
\]
leading to Eq.~\eqref{eq:Ugroup}.
%
\item identity: 
\begin{equation}
\hat{U}(t_0,t_0) = I
\end{equation}
%
\item inverse:
\begin{equation}
\hat{U}^{-1}\left(t, t_{0}\right)=\hat{U}\left(t_{0}, t\right)
\end{equation}
Since $t_1$ is arbitrary in Eq.~\eqref{eq:Ugroup}:
\[
|\varphi(t)\rangle=\hat{U}\left(t, t_{0}\right)\left|\varphi\left(t_{0}\right)\right\rangle,
\]
and
\[
\left|\varphi\left(t_{0}\right)\right\rangle=\hat{U}\left(t_{0}, t\right)|\varphi(t)\rangle
\]
but
\[
|\varphi(t_0)\rangle = \underbrace{\hat{U}\left(t, t_{0}\right) \hat{U}\left(t_{0}, t\right)}%
_{I}%
|\varphi(t)\rangle
\]
so
\[
\hat{U}^{-1}\left(t, t_{0}\right) I=\hat{U}\left(t_{0}, t\right) \rightarrow \hat{U}^{-1}\left(t, t_{0}\right)=\hat{U}\left(t_{0}, t\right)
\]
\end{enumerate}


\subsection{\texorpdfstring{Dynamical equation for $\hat{U}$}{Dynamical equation for U}}

Postulates IV and IV$^\prime$ are not independent, one needs
one of them only. What is the dynamical equation?
%%% 9 OK
We postulate it is
\begin{equation}
\boxed{
\hat{H}\left(t^{\prime}\right) \hat{U}\left(t^{\prime}, t\right)=i \hbar \frac{d}{d t^{\prime}} \hat{U}\left(t^{\prime}, t\right)
}
\label{eq:dynEqU}
\end{equation}

Is this correct? Differentiate Eq.~\eqref{eq:postulate4prime} (make $t \to t^\prime, t_0\to t)$
\begin{equation}
i \hbar \frac{d}{d t^\prime}|\varphi(t^\prime)\rangle =
i \hbar\left(\frac{d}{d t^\prime} \hat{U}\left(t^{\prime},t\right) \right) |\varphi(t)\rangle	
\label{eq:g40}
\end{equation}
but again, by Eq.~\eqref{eq:postulate4prime} itself, we have
\begin{equation}
\hat{H}\left(t^{\prime}\right)\left|\varphi\left(t^{\prime}\right)\right\rangle=
\hat{H}(t^\prime) \hat{U}\left(t^{\prime}, t\right)|\varphi(t)\rangle
\label{eq:g41}
\end{equation}
The Schrödinger equation says that the the left-hand sides of Eqs.~\ref{eq:g40} and \ref{eq:g41} are the same.
So, to be consistent with it, the right-hand sides should also be equal:
\[
\hat{H}(t^\prime) \hat{U}\left(t^{\prime}, t\right)|\varphi(t)\rangle = 
i \hbar\left(\frac{d}{d t^\prime} \hat{U}\left(t^{\prime},t\right) \right) |\varphi(t)\rangle
\]
Since $|\varphi(t)\rangle$ is arbitrary, Eq.~\eqref{eq:dynEqU} follows.


Way back from Eq.~\eqref{eq:dynEqU} to Schrödinger equation. Take the limit on the left-hand side of Eq.~\eqref{eq:dynEqU}:
\[
\lim _{t^{\prime} \rightarrow t} 
\hat{H} (t^{\prime}) \hat{U}(t^{\prime}, t)=\hat{H}(t) \hat{U}(t, t)=\hat{H}(t)
\]
and therefore it reduces to
\begin{equation}
\hat{H}(t)=\left.i \hbar \frac{d}{d t^{\prime}} \hat{U}\left(t^{\prime}, t\right)\right|_{t^{\prime}=t}
\label{eq:g42}
\end{equation}
Again, differentiate Eq.~\eqref{eq:postulate4prime} with respect to $t^\prime$, and take the limit $t^\prime \to t$:
\begin{equation}
i \hbar \frac{d}{d t^\prime}|\varphi(t^\prime)\rangle = 
\underbrace{i \hbar \lim _{t^{\prime} \rightarrow t} \frac{d}{d t^{\prime}} \hat{U}\left(t^{\prime}, t\right)}%
_{\hat{H}(t)}%
|\varphi(t^\prime)\rangle
\end{equation}
therefore
\[
i \hbar \frac{d}{d t}|\varphi(t)\rangle=\hat{H}(t)|\varphi(t)\rangle
\]
as we said.

Equation~\eqref{eq:g41} is not easy to integrate in general,
when $\hat{H}(t) \neq$ constant in time.
%%% 10 OK
When $\hat{H}(t)$ is constant:
\begin{equation}
i \hbar \frac{d U\left(t, t_{0}\right)}{d t}=\hat{H} \hat{U}\left(t, t_{0}\right)
\to
\hat{U}\left(t, t_{0}\right)=e^{-i \hat{H}\left(t-t_{0}\right) / t}
\label{eq:g44}
\end{equation}
as can be easily verified.

When $\hat{H}(t)$ is not constant, it can be proved (left as exercise) that:
\begin{equation}
\hat{U}\left(t, t_{0}\right)=\mathcal{T} e^{-i} \int_{t_{0}}^{t} d t^{\prime} \hat{H}\left(t^{\prime}\right)
\end{equation} 
where $\mathcal{T}$ is the \emph{time ordering}.
\begin{equation}
\begin{gathered}
\mathcal{T} e^{-i \int_{t_{0}}^{t} d t^{\prime} \cdots} = 1 
-i / \hbar \int_{t_{0}}^{t} dt^{\prime} \hat{H}\left(t^{\prime}\right)\\
+
\frac{1}{2 !}\left(\frac{i}{\hbar}\right)^{2} \int_{t_{0}}^{t} d t^{\prime} \int_{t_{0}}^{t} d t^{\prime \prime} \, \mathcal{T}\left[\hat{H}\left(t^{\prime}\right) H\left(t^{\prime \prime}\right)\right]+\cdots
\end{gathered}
\end{equation}

\subsection{Stationary states}

Suppose a system completely isolated from 
external influences $\to$ description \emph{cannot} depend
on the time origin, only time differences matter.
\begin{equation}
\hat{U}\left(t, t_{0}\right)=\hat{U}\left(t-t_{0}\right)
\end{equation}

%%% 11 OK

Therefore, Eq.~\eqref{eq:g42} implies:
\begin{equation}
\begin{aligned} 
\hat{H}\left(t_{0}\right) 
&=\left.i \hbar \frac{d}{d t} \hat{U}\left(t, t_{0}\right)\right|_{t=t_{0}}
\quad\textrm{call }\tau=t-t_{0} \\ 
&=\left.i \hbar \frac{d}{d t} \hat{U}\left(t-t_{0}\right)\right|_{t=t_{0}}
=\left.\underbrace{\frac{i}{h} \frac{d}{d \tau} \hat{U}(\tau)}_{\textrm{independent of }t_0}\right|_{\tau=0}
\end{aligned}
\end{equation}
So, $\hat{H}$ is independent of time.
In this case, Eq.~\eqref{eq:g44} applies:
\begin{equation}
\hat{U}\left(t, t_{0}\right)=e^{-i / \hbar\left(t-t_{0}\right) \hat{H}}
\end{equation}
where $\hat{U}\left(t, t_{0}\right)$ implements a time translation $t-t_0$
on the state vector.
For $t-t_0 = \varepsilon$ infinitesimal
\begin{equation}
\hat{U}\left(t, t_{0}\right)=1-i / \hbar \cdot \varepsilon \hat{H}
\end{equation}
meaning that $\hat{H}$ is the \emph{generator} of infinitesimal time translations.
And for an isolated system, the most
general definition of a Hamiltonian is
precisely that of an infinitesimal time-translation
generator (more to come when studying symmetry in quantum physics).

%%% 12

Let $|n,r\rangle$ be eigenvector of $\hat{H}$ with eigenvalue 
$E_n$: $\hat{H}|n, r\rangle=E_{n}|n, r\rangle$. Its time evolution
is simply: $\left.|\varphi(t_{0})\right\rangle=|n, r\rangle$ $\to$
\begin{equation}
|\varphi(t)\rangle=e^{-i / \hbar\left(t-t_{0}\right) \hat{H}}|n, r\rangle=e^{-i / \hbar\left(t-t_{0}\right) E_{n}}|n, r\rangle
\end{equation}
The probability of finding $|\varphi(t)\rangle$ in any
state $|\chi\rangle$ is independent of time:
\begin{equation}
\begin{aligned}|\langle x | \varphi(t)\rangle|^{2} &=\left|e^{-i / \hbar\left(t-t_{0}\right) E_{n}}\langle x | n, r\rangle\right|^{2} \\ &=\left|\left.\langle x | n, r\rangle\right|^{2}=\left|\left\langle x | \varphi\left(t_{0}\right)\right\rangle\right|^{2}\right.
\end{aligned}
\end{equation}
Independent of time $\to$ ``stationary'' and we call
$|n,r\rangle$ a \emph{stationary state}.

For the time evolution of expansion coefficients:
\begin{equation}
\left|\varphi\left(t_{0}\right)\right\rangle=\sum_{n, r} C_{n, r}(n, r\rangle
\end{equation}
evolves to
\begin{equation}
| \varphi(t)\rangle=\sum_{n, r}C_{n, r} e^{-i / \hbar\left(t-t_{0}\right) \hat{H}}|n, r\rangle
\end{equation}
where $\left\langle n, r\left|\varphi\left(t_{0}\right)\right\rangle \rightarrow C_{n, r}\left(t_{0}\right)\right.$. So
\begin{equation}
\begin{gathered}
\ket{\varphi(t)} = \sum_{n_r} C_{n, r}(t_0) e^{-i / \hbar\left(t-t_{0}\right)E_n}|n, r\rangle\\
\text{with } C_{n, r}(t) = e^{-i / \hbar\left(t-t_{0}\right)E_n} C_{n, r}(t_0)
\end{gathered}
\end{equation}

\end{document}  