%!TEX TS-program = xelatex
%!TEX encoding = UTF-8 Unicode

\documentclass[12pt]{article}
\usepackage{geometry}                % See geometry.pdf to learn the layout options. There are lots.
\geometry{a4paper,top=1.8cm,bottom=2cm,right=2.7cm}
\usepackage[parfill]{parskip}    % Activate to begin paragraphs with an empty line rather than an indent
\usepackage{graphicx}
\usepackage{amsmath}
\usepackage{amssymb}
\usepackage{mathtools}
\usepackage{physics}
\newcommand{\be}{\begin{equation}}
\newcommand{\ee}{\end{equation}}
\usepackage[thicklines]{cancel}
\usepackage{url}
\usepackage{booktabs}

\usepackage{fontspec,xltxtra,xunicode}
\defaultfontfeatures{Mapping=tex-text}

\newcommand{\polv}{\ensuremath{\updownarrow}}
\newcommand{\polh}{\ensuremath{\leftrightarrow}}
\newcommand{\poldr}{\rotatebox[origin=c]{45}{\ensuremath{\leftrightarrow}}}
\newcommand{\poldl}{\rotatebox[origin=c]{-45}{\ensuremath{\leftrightarrow}}}

\title{Advanced Quantum Mechanics\\Exercises Set 5\vspace{-0.5em}}
%\author{The Author}
\date{Due date: June 6\textsuperscript{th}, 2023}

\begin{document}
\maketitle


\textbf{Exercise 1.} If \(S_{q}^{(k)}\) and \(T_{q}^{(k)}\) are two spherical tensor operators of rank \(k\) prove that the con%
tracted operator
\[
\sum_{q=-k}^{k}(-1)^{q} S_{q}^{(k)} T_{-q}^{(k)}
\]
is a scalar operator. For \(k=1\), show that this scalar operator is just the scalar product \(\vec{S} \cdot \vec{T}\).


\textbf{Exercise 2.} Wigner-Eckart theorem \textit{ad nauseam}: let $\ket{j,m}$ be the eigentstates of $\hat{\vec{J}}^2$ and $\hat{J}_0 = \hat{J}_z$,
and let
\[
\chi=\langle 2,2|\hat{x} \hat{y}| 0,0\rangle
\] 
Using the Wigner-Eckart theorem, calculate as a function of $\chi$ the matrix elements:
\[
\langle 2, m|\hat{\mathcal{O}}| 0,0\rangle
\]
for $\mathcal{O} = $
a) $\hat{x}\hat{y}$ b) $\hat{x} \hat{z}$, c) $\hat{y} \hat{z}$, d) $\hat{x}^{2}$ e) $\hat{y}^{2}$ and f) $\hat{z}^{2}$.
Note that in principle one could evaluate the matrix elements
directly, but the point of the exercise is to employ the Wigner-Eckart theorem over and over.


\textbf{Exercise 3.} Consider the two-particle operator
\[
\hat{S}(1,2)=2\left[3 \frac{\left(\vec{S}_{1} \cdot \vec{r}\right)\left(\vec{S}_{2} \cdot \vec{r}\right)}{r^{2}}-\vec{S}_{1} \cdot \vec{S}_{2}\right]
\]
where $r = |\vec{r}|$ and $\hat{r} = \vec{r}/r$. Rewrite this operator in terms of the spherical components of $\vec{S}_{1}$, $\vec{S}_{2}$ and $\vec{r}$.


\textbf{Exercise 4.} In this exercise you will derive a very useful expression for matrix elements of
vector operators in the basis of the total angular momentum operator. We have seen that the
Wigner-Eckart theorem asserts that the matrix element of a component $T_q^{(k)}$ of a spherical
tensor $T$ in the basis $\ket{\tau,jm}$, where $j$ and $m$ refer to the eigenvalues of the angular momentum
$\hat{\vec{J}}^2$ and $\hat{J}_0$ operators and $\tau$ stands for the eigenvalues of all compatible observables other than
angular momentum, is given by
\[
\left\langle\tau^{\prime}, j^{\prime} m^{\prime}\left|T_{q}^{(k)}\right| \tau, j m\right\rangle=\frac{C_{q m ; j^{\prime} m^{\prime}}^{k j}}{\sqrt{2 j^{\prime}+1}}\left\langle\tau^{\prime}, j^{\prime}\left\|T^{(k)}\right\| \tau, j\right\rangle
\]
where \(C_{q m ; j^{\prime} m^{\prime}}^{k j}\) are the Clebsh-Gordan coefficients.

a) Use the fact that \(C_{0 j ; j j}^{1 j}=\sqrt{j /(j+1)}\) and determine the reduced matrix element of the
operator \(\hat{\vec{J}}\), i.e. \(\left\langle\tau^{\prime}, j^{\prime}\|J\| \tau, j\right\rangle\).

b) Use the Wigner-Eckart theorem to show that for any vector operator $\hat{\vec{V}}$:
\[
\left\langle\tau^{\prime}, j^{\prime} m^{\prime}\left|V_{q}\right| \tau, j m\right\rangle=\frac{\left\langle\tau^{\prime}, j^{\prime}\|V\| \tau, j\right\rangle}{\left\langle\tau^{\prime}, j^{\prime}\|j\| \tau, j\right\rangle}\left\langle\tau^{\prime}, j^{\prime} m^{\prime}\left|J_{q}\right| \tau, j m\right\rangle
\]

c) Next, show that for \(j^{\prime}=j\), the ratio of the reduced matrix elements in item b) can be
eliminated in favour of the matrix element \(\langle\tau^{\prime}, j m|\vec{J} \cdot \vec{V}| \tau, j m\rangle\), so that it can be written as
\[
\left\langle\tau^{\prime}, j m^{\prime}\left|V_{q}\right| \tau, j m\right\rangle=\frac{\left\langle\tau^{\prime}, j m|\vec{J} \cdot \vec{V}| \tau, j m\right\rangle}{j(j+1) \hbar^{2}}\left\langle j m^{\prime}\left|J_{q}\right| j m\right\rangle
\]


\textbf{Exercise 5.} Connection between time-reversal and rotation for spins.

(a) Use the fact that
\[
\hat{\Theta} \hat{\vec{J}} \hat{\Theta}^{-1}=\hat{\vec{J}}_{t}=-\hat{\vec{J}}
\]
 to show that $\hat\Theta|j m\rangle$ equals $|j,-m\rangle$ up to some phase that includes the factor
$(-1)^{m}$. That is, show that $\hat\Theta|j m\rangle=e^{i \delta}(-1)^{m}|j,-m\rangle$, where $\delta$ is independent of $m$.

(b) Using the same phase convention, find the time-reversed state corresponding to ${D}(R)|j m\rangle$,
where $D(R)$ are the Wigner matrices.
Consider using the infinitesimal form ${D}(\hat{\mathbf{n}}, d \phi)$ and then generalize to finite rotations.

(c) From these results, prove that, independent of $\delta$, one finds
\[
{D}_{m^{\prime} m}^{(j) *}(R)=(-1)^{m-m^{\prime}} {D}_{-m^{\prime},-m}^{(j)}(R)
\]

(d) Conclude that we are free to choose $\delta=0$, and $\Theta|j m\rangle=(-1)^{m}|j,-m\rangle=i^{2 m}|j,-m\rangle$.


\end{document}